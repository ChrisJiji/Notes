\documentclass[11pt]{extarticle}
\usepackage{amsmath}
\usepackage{amssymb}
\usepackage{amsthm}
\usepackage{graphicx}
\usepackage{setspace}
\onehalfspacing
\usepackage{fullpage}
\usepackage{titling}
\usepackage{mathrsfs}  
\newtheorem*{remark}{Remark}
\newtheorem{theorem}{Theorem}[section]
\newtheorem{corollary}{Corollary}[theorem]
\newtheorem{lemma}[theorem]{Lemma}
\theoremstyle{definition}
\newtheorem{definition}{Definition}[section]

\title{A special case in the set of all potential boyfriends}
\author{Chris Ji, Matthew Stringer}
\begin{document}

\maketitle
\begin{abstract}
The most fundamental problem in the theory of dating is to determine whether you are an element of the set of potential boyfriends. We currently know that it is not possible to attract a female companion if you are not contained within the set. Many attempts have been made to forcefully insert oneself into the set, all ending in futility. We present an attempt to resolve this problem by performing an exhaustive analysis, to determine whether ‘me’ is contained within the set of potential boyfriends. We also present various coping mechanisms and vices for individuals not contained with the set. This proof was inspired by the erroneous and un-inspired work by Shern Low. 
\end{abstract}


\section*{Introduction}
This paper will explore a special case in the set of potential boyfriends by Chris Ji, aided by Matt Stringer, for students with a reasonably broad background in failed relationships and lives filled with loneliness. It is hard to give precise prerequisites to best understand the presented material, but a graduate level course in disappointing women and a passing acquaintance with broken dreams should suffice. \\
Chris and Matt are no strangers to the world of theoretical relationships. It has been questioned among them whether they belong to the set of potential boyfriends for years. Their earliest work involved attempting to positively identify any element within the set of potential boyfriends, during the spring of 2018 located at the Depressive Institute of Waterloo (colloquially known as University of Waterloo). This work hinged greatly upon decades of prior research conducted by male students with crippling social anxiety, from the 1950’s up until present day. The findings of Chris and Matt were fruitless, as it appeared no students at this institute satisfied the determined conditions necessary to be contained within the set. \\
After returning to their hometown, they attempted to give a seminar on these findings to interested students, faculty, and friends. The audience, keen to learn new material, did not appreciate the trivial findings; it seemed all audience members knew it as fact that no male UWaterloo student would be contained within the set of potential boyfriends.\\
Since then, Chris and Matt have worked frivolously to complete a condensed proof to show the necessary conditions for an element to be considered in the set of potential boyfriends. The aim of this work is to convey the strong and simple line of logic on which the proof rests. It is certainly well within the ability of most lonely students to appreciate the way the building blocks of the proof go together to give the result, even though those blocks may themselves be hard to fathom. If anything, this paper should serve as an inspiration to students to explore the outdoors, interact socially with other human beings, and attempt as hard as they may to one day become an element of the set.

\section*{Notation}
We will denote $X$ as the set of all potential boyfriends in the facebook group subtle asian traits. Clearly, every element of $X$ is a singular (gender neutral) person. I believe people are much more than words, so I will not attempt to describe any $x\in X$. Instead, I will leave every $x$ to be an abstract concept. \\
\theoremstyle{definition}
\begin{definition}
$\mathscr{F}(x)$ is an indicator variable taking a person, $x$, and $\mathscr{F}(x)=1$ when $x$ has had more than 1 girl confess to him., and $0$ otherwise. In short, $\mathscr{F}(x):X\rightarrow\{0,1\}$, where $$\mathscr{F}(x)\begin{cases}1,\quad\text{if confessions}(x)>0\\0,\quad\text{otherwise}\end{cases}$$
\end{definition}

\theoremstyle{definition}
\begin{definition}
We will define $\mathscr{E}$ similarly for the number of girls that have loved $x$: $$\mathscr{E}(x)=\begin{cases}1,\quad\text{if love}(x)>0\\0,\quad\text{otherwise}\end{cases}$$
\end{definition}

\theoremstyle{definition}
\begin{definition}
A function $\mathscr{G}:x\rightarrow\mathbb{N}$ will simply count the potential number of relationships $x$ can have. Note that if $\mathscr{G}(x)=0$, then $x\notin X$, by definition. 
\end{definition}


\begin{theorem}
$\mathscr{F}(x)=0\Leftrightarrow\mathscr{G}(x)=0$
\end{theorem}
\begin{proof}
Since $X$ is the set of all potential boyfriends in subtle asian traits, by definition every $x\in X$ is too beta to ever confess to a girl. If a girl had never confessed to $x$ (ie. $\mathscr{F}(x)=0$), a relationship can never have started, and hence $\mathscr{F}(x)=0\Rightarrow\mathscr{G}(x)=0$. Since $x$ is desperate and ugly and dumb and will never live up to their families dreams, they must accept any confessions from girls for a chance at a fulfilling future. As such, if $\mathscr{G}(x)=0$, then $\mathscr{F}(x)$ clearly has to be 0.
\end{proof}

\begin{theorem}
$\mathscr{E}(x)=0\Leftrightarrow\mathscr{G}(x)=0$
\end{theorem}
\begin{proof}
As a relationship has to have love, if no one has ever loved $x$ (ie $\mathscr{E}(x)=0$), then $x$ will have never had the minimum requirement for a relationship, and so $\mathscr{G}(x)=0$. Similarly, since $\mathscr{G}(x)=0$, no one has ever loved $x$, and so $\mathscr{E}(x)=0$. 
\end{proof}

\begin{lemma}
$x\in X\Leftrightarrow \mathscr{F}(x)\mathscr{E}(x)=1$
\end{lemma}
\begin{proof}
If $x\in X$, then $\mathscr{G}(x)>0$ by definition. Then from the above theorems, we know that $\mathscr{F}(x)$ and $\mathscr{E}(x)$ are both non-zero. As the only non-zero element in both the codomains of $f$ and $g$ are 1, $\mathscr{F}(x)\mathscr{E}(x)=1*1=1$. Similarly, if $\mathscr{F}(x)\mathscr{E}(x)=1$, as above, $\mathscr{F}(x)\neq0\cap \mathscr{E}(x)\neq0\Rightarrow\mathscr{G}(x)\neq0$. As $\mathscr{G}(x)\neq0$, then $x\in X$ by definition.
\end{proof}


\section{The case where x=me}
As I am also a member of subtle asian traits, the only person I really know is me. Hence, I will only do the case where I am $x$. 
\begin{theorem}
When $x=\text{me}$, $x\notin X$. In other words, I am not a potential boyfriend.
\end{theorem}
\begin{proof}
We will first define a few lemmas.

\begin{lemma}
$\mathscr{E}(x)=1$. 
\end{lemma}

\begin{proof}
My mommy loves me $>$:(
\end{proof}


\begin{lemma}
$\mathscr{F}(x)=0$
\end{lemma}

\begin{proof}
We will do a proof by memory. I do not recall any girls confessing to me, and so $\mathscr{F}(x)=0$. 
\end{proof}

Since $\mathscr{F}(x)=1,\mathscr{E}(x)=0$, then $\mathscr{F}(x)\mathscr{E}(x)=0$, and by Lemma 0.3, $x\notin X$, and hence I am not a potential boyfriend.
\end{proof}

\section*{Discussion}
Most effective coping mechanisms and vices
\begin{itemize}
    \item Moderate alcoholism and recreational drug use have both been observed to reduce the stress created from not being contained within the set of potential boyfriends. 
    \item Consuming exorbitant amounts of bubble tea also appears to have a numbing effect on the internal sadness radiating throughout the bodies of non-set elements.
    \item Sunlight and human contact appear to have a positive affect on the psyche, however those not contained within the set often avoid these at all costs.
\end{itemize}

\paragraph{Aside} Theorists have been baffled with further findings, which show that these behaviours are equally performed by both non-elements and elements of the set. There lacks an obvious bridge between performing these activities and becoming an element of the set. Our findings may suggest this bridge to be “not being ugly”, however there lacks a concrete proof.

\section*{References}
\begin{enumerate}
    \item Low S.: When you’re trying to figure out why you don’t have a girlfriend, Subtle Asian Traits (2018)
\end{enumerate}




\end{document}