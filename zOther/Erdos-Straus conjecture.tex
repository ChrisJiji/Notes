\documentclass[10pt,letter]{article}
\usepackage{amsmath}
\usepackage{amssymb}
\usepackage{amsthm}
\usepackage{graphicx}
\usepackage{setspace}
\onehalfspacing
\usepackage{fullpage}
\newtheorem*{remark}{Remark}
\begin{document}

\paragraph{The problem}
For every integer $n\geq2$, there exist positive integers $x,y,z$ such that $\frac{4}{n}=\frac{1}{x}+\frac{1}{y}+\frac{1}{z}$, with $x\leq y\leq z$ 

\paragraph{Not solved} 
\begin{itemize}
    \item primes congruent to 1 mod 4
    \item primes congruent to 1 mod 8
    \item primes congruent to 1 mod 24
    \item primes congruent to 1 mod 840 $(1^2)$ 
    \item primes congruent to 121 mod 840 $(11^2)$
    \item primes congruent to 169 mod 840 $(13^2)$
    \item primes congruent to 289 mod 840 $(17^2)$
    \item primes congruent to 361 mod 840 $(19^2)$
    \item primes congruent to 529 mod 840 $(23^2)$
\end{itemize}

\paragraph{some thoughts}
$$\frac{4}{3m}=\frac{1}{2m}+\frac{1}{2m}+\frac{1}{3m}$$
$$\frac{4}{3m+1}=\frac{1}{\frac{1}{2}(3m+1)}+\frac{1}{3m+1}+\frac{1}{3m+1}$$
$$\frac{4}{3m+2}=\frac{1}{m+1}+\frac{1}{3m+2}+\frac{1}{(m+1)(3m-2)}$$
Also, it has been solved for everything other than 1 mod 4. So we need to solve it for numbers that are 1 mod 3 and 1 mod 4. 

\paragraph{Paper by Monks and Velingker}
\begin{itemize}
    \item In the prime factorization of $x,y,z$, all of the primes appear at least twice 
    \item If $q$ is an integer such that $q,p$ are coprime, then if $q$ divides one of $x,y,z$, then $q$ divides the product of the other two. In particular, $x|yz$, from the result below. Also, there is exactly 1 $p$ in the prime factorization of $z$ and at most one $p$ in the prime factorization of $y$. Then $z|pxy$ and $y|xz$. 
    \item $p\nmid x$ and $p|z$ and $x<p$. 
    \item $p^2\nmid y$ and $p^2\nmid z$
    \item $k=\frac{z}{p}$, then $p\nmid y\Leftrightarrow 4k\cong1\quad(\text{mod }p)$ 
    \item $\text{gcd}(y,z)\neq1$
    \item If $z=p\cdot\text{lcm}(x,y)$, then $p^2+m$ has a divisor congruent to $-p\quad(\text{mod }m)$ where $m=4\cdot\text{gcd}(x,y)$
    \item it is solved for $p\cong3\quad \text{mod 4}$ 
    \item $\left\lceil\frac{p}{4}\right\rceil\leq x\leq\frac{p+1}{3}\leq y$ 
    \item if $p|y$ then $x\leq \left\lfloor\frac{pc}{3c-1}\right\rfloor$ where $c=\left\lceil\sqrt{\left\lceil\frac{p}{4}\right\rceil}\right\rceil$ 
    \item $p\left\lfloor\frac{5+\sqrt{4p-3}}{4}\right\rfloor\leq z\leq\frac{p^2(p+1)\left\lceil\frac{p}{4}\right\rceil}{4\left\lceil\frac{p}{4}\right\rceil-p}$
\end{itemize}
This conjecture is true if and only if for every prime $p$, for every positive integer $n$, $pn$ has two divisors that sum to a multiple of either $4n-p$ or $4n-1$. \\ 


\end{document}