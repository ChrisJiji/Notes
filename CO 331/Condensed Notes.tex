\documentclass[10pt,letter]{article}
\usepackage{amsmath}
\usepackage{amssymb}
\usepackage{amsthm}
\usepackage{graphicx}
\usepackage{setspace}
\onehalfspacing
\usepackage{fullpage}
\newtheorem*{remark}{Remark}
\theoremstyle{plain}
\newtheorem*{theorem*}{Theorem}
\newtheorem{theorem}{Theorem}[section]
\newtheorem{corollary}{Corollary}[theorem]
\newtheorem*{lemma*}{Lemma}
\newtheorem{lemma}[theorem]{Lemma}
\theoremstyle{definition}
\newtheorem{definition}{Definition}[section]
\newtheorem*{definition*}{Definition}

\begin{document}

\paragraph{Goals of Coding Theory}
\begin{enumerate}
    \item High error correcting capability
    \item High information rate 
    \item Efficient encoding and decoding algorithms
\end{enumerate}
\paragraph{q-ary Symmetric Channel} A channel with $q$ letters where the probability of every symbol being correct is $1-p$, and the probability of the symbol changing to any other symbol is $\frac{p}{q-1}$. 
\paragraph{Block Code} A code with $M$ words, all of length $n$ is called an $[n,M]-$code. 
\paragraph{IMLD} decodes to the codeword that has the last hamming distance from the received word.
\paragraph{MED} Maximizes $P(c|r)=P(r|c)\frac{P(c)}{P(r)}$
\paragraph{Sphere Packing Bound}$M\sum_{i=1}^e{n\choose i}(1-q)^i\leq q^n$. $q^n$ is the entire space, $M$ is the number of words, and each sum is the area of each word that can be error corrected. 
\paragraph{Fields}
\begin{itemize}
    \item Addition is associative, commutative
    \item Additive identity, inverse
    \item Multiplication is associative, commutative
    \item Multiplicative identity, inverse
    \item Multiplication distributes over addition
    \item The zero ring is not a field
\end{itemize}
\paragraph{Rings}
A commutative ring has everything as above, but multiplication doesn't need an inverse, and the zero ring is a ring. A ring has everything as as a commuative ring, but multiplication isn't commutative.
\paragraph{Order} The order of a field, $F$ is $|F|$. Note if $F$ has characteristic $p$, then the order is $p^n$ for some $n$. The order of some $\alpha\in GF(q)^*$, then ord$(\alpha)$ is the smallest integer $t$ such that $\alpha^t=1$. 
\paragraph{Characteristic} The characteristic of $F$ is the smallest positive integer $p$ such that $\underbrace{1+\cdots+1}_{p\text{ times}}=0$. 
\paragraph{Subfield} Let $F$ be a field with characteristic $p$. A subfield $E$ of $F$ is $\{0,1,\ldots,\underbrace{1+\cdots+1}_{p\text{ times}}\}\subseteq F$. 
\paragraph{Prime Subfield} Let $F$ be a field with characteristic $p$. The prime subfield of $F$ is $\mathbb{Z}_p$. 
\paragraph{Generator (primitive element)} $\alpha\in GF(q)$ with order $q-1$ is called a generator. 

\end{document}