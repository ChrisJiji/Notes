\documentclass[12pt,letter]{article}
\usepackage{amsmath}
\usepackage{amssymb}
\usepackage{amsthm}
\usepackage{graphicx}
\usepackage{setspace}
\onehalfspacing
\usepackage{fullpage}
\newtheorem*{remark}{Remark}
%\date{\vspace{-5ex}}
\begin{document}
\section*{Module 5: Lists}
You should understand the data definitions for lists, how the
template mirrors the definition, and be able to use the template to
write recursive functions consuming this type of data.\\ 
You should understand box-and-pointer visualization of lists.\\ 
You should understand the additions made to the semantic model of
Beginning Student to handle lists, and be able to do step-by-step
traces on list functions.\\ 
You should understand and use (listof . . . ) notation in contracts.\\ 
You should understand strings, their relationship to characters and
how to convert a string into a list of characters (and vice-versa). \\ 
You should be comfortable with lists of structures, including
understanding the recursive definitions of such data types, and you
should be able to derive and use a template based on such a
definition

\section*{Module 6: Working with recursion}
You should understand the recursive definition of a natural number,
and how it leads to a template for recursive functions that consume
natural numbers.\\ 
You should understand how subsets of the integers greater than or
equal to some bound m, or less than or equal to such a bound, can
be defined recursively, and how this leads to a template for recursive
functions that “count down” or “count up”. You should be able to
write such functions.\\ 
You should understand the principle of insertion sort, and how the
functions involved can be created using the design recipe.
You should be able to use list abbreviations and quote notation for
lists where appropriate. \\ 
You should be able to construct and work with lists that contain lists.\\ 
You should understand the similar uses of structures and fixed-size
lists, and be able to write functions that consume either type of data.\\ 
You should understand the three approaches to designing functions
that consume two lists (or a list and a number, or two numbers) and
know which one is suitable in a given situation.

\section*{Module 7: Types of Recursion}
You should be able to recognize uses of pure structural recursion,
accumulative recursion, and generative recursion.

\section*{Module 8: Trees}
You should be familiar with tree terminology. \\ 
You should understand the data definitions for binary arithmetic
expressions, evolution trees, and binary search trees, understand
how the templates are derived from those definitions, and how to use
the templates to write functions that consume those types of data.\\ 



\section*{Midterm 2 Review}
\textbf{Nested box representations and box-and-pointer representations.}\\ \\ 
\textbf{(list), and '(), and cons notations of lists. :}\\ \\ 
\textbf{Stepper questions with lists and new functions. The stepper questions will still be with the $1^{st}$, $2^{nd}$, last step format.}\\ \\ 
\textbf{Writing template functions for structures and for lists}\\ \\
\textbf{Insertion Sort}\\ \\ 
\textbf{Recursing on a list (sublist)}\\ \\ 
\textbf{Working with 2 dimensional lists (get-cols-range)}\\ \\ 
\textbf{Recursing on a Nat (add)}\\ \\ 
\textbf{Recursing in lockstep (hangman)}\\ \\ 
\textbf{Recursing at different rates (symmetric-difference)}\\ \\ 
\textbf{Recognizing the difference between the different types of recursion}\\ \\
\textbf{Association lists (HeightsAL)}\\ \\ 
\textbf{Accumulative recursion (shortest)}\\ \\ 
\textbf{Trees (count-internal-nodes)}

\end{document}