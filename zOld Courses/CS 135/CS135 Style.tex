\documentclass{article}     % Specifies the document class 

\begin{document}

\section*{1.2} 
Do not copy paste 
\\
Save using file > Save Other > Save Definitions As Text

\section*{2.1 Comments}
Full line comments are double semicolon and a space, in-line comments are a 
single semicolon and a space.

\section*{2.2 Header}
Use the following header for all assignments: \\
;;\\
;; ***************************************************\\
;; Chris Ji (20725415)\\
;; CS 135 Fall 2017\\ 
;; Assignment XX, Problem X\\
;; ***************************************************\\
;;\\

\section*{2.3 Whitespace}
Insert two consecutive blank lines between functions or “function blocks”\\
Insert one blank line before and after the function definition: do not use more than one consecutive
blank line within a function block. \\
If you have many functions and/or large function blocks, you may want to insert a row of symbols (such as the *’s used in
the file header above) to separate your functions. \\
If the question asks you to write more than one function, the file should contain them in the order specified by the
assignment. \\
Helper functions are placed above the assignment function they are
helping. 
For conditional expressions, placing the keyword cond on a line by itself, and
aligning not only the questions but the answers as well can improve readability
(provided that they are short). Each question must appear on a separate line, and long questions/answers
should be placed on separate lines. Marks may be deducted if a cond is too dense
or confusing.\\

\section*{2.4 Variable and Function Identifiers}
It is a Racket convention to use lowercase letters and hyphens\\
The rare exception is when proper names are used\\
Constants should be used to improve your code in the following ways:\\
\indent To improve the readability of your code by avoiding “magic” numbers.\\
\indent To improve flexibility and allow easier updating of special values.\\
\indent To define values for testing and examples.

\section*{3.1 Design Recipe Sample}
;; (sum-of-squares p1 p2) produces the sum of \hfill <--- Purpose\\
;; \indent the squares of p1 and p2\\
;; sum-of-squares: Num Num -> Num \hfill <--- Contract\\
;; Examples: \hfill <--- Examples\\
(check-expect (sum-of-squares 3 4) 25)\\
(check-expect (sum-of-squares 0 2.5) 6.25)\\
(define (sum-of-squares p1 p2) ; \hfill <--- Function Header\\
\\
(+ (* p1 p1) ; \hfill <--- Function Body\\
\indent (* p2 p2)))\\
\\
;; Tests: \hfill <--- Tests\\
(check-expect (sum-of-squares 0 0) 0)\\
(check-expect (sum-of-squares -2 7) 53)\\\\
\textbf{Do these in the following order:}
\begin{itemize}
    \item Draft Purpose
    \item Examples
    \item Function Header
    \item Contract
    \item Finalize Purpose
    \item Body
    \item Tests
\end{itemize}

\section*{3.2 Purpose}
;; (sum-of-squares p1 p2) produces the sum of\\
;; the squares of p1 and p2\\
The purpose statement has two parts: an illustration of how the function is applied, and a brief description of what the function does. \\
The purpose starts with an example of how the function is applied, which
uses the same parameter names used in the function header.
\begin{itemize} 
\item Do not write the word “purpose”.
\item The description must include the names of the parameters in the purpose
to make it clear what they mean and how they relate to what the function
does (choosing meaningful parameter names helps also). Do not include
parameter types and requirements in your purpose statement — the contract
already contains that information.
\item If the description requires more than one line, “indent” the next line of the
purpose 2 or 3 spaces.
\item If your purpose requires more than two or three lines, you should probably
consider re-writing it to make it more condensed. It could also be that you
are describing how your function works, not what it does.
\item If you find the purpose of one of your helper functions is too long or too
complicated, you might want to reconsider your approach by using a different
helper function or perhaps using more than one helper function.
\end{itemize}

\section*{3.3 Contract}
;; function-name: Type1 Type2 ... TypeN -> Type\\
The contract contains the name of the function, the types of the arguments it consumes,
and the type of the value it produces.

\section*{3.4 Examples}
;; Examples:\\
(check-expect (sum-of-squares 3 4) 25)\\
(check-expect (sum-of-squares 0 2.5) 6.25)

\section*{3.5 Function Header and Body}
\textbf{Develop your purpose, contract, and examples \underline{before} you write the code for the function.}\\
(define (sum-of-squares p1 p2)\\
\indent(+ (* p1 p1)\\
\indent\indent(* p2 p2)))\\\\
\textbf{MAKE SURE YOUR FUNCTIONS ARE SPELLED THE SAME WAY AS GIVEN}

\section*{3.6 Tests}
;; Tests:\\
(check-expect (sum-of-squares 0 0) 0)\\
(check-expect (sum-of-squares -2 7) 53)
\begin{itemize}
\item Make sure your tests are testing every part of the code, e.g. if a conditional expression has 3 possible outcomes, all 3 outcomes are tested.
\item Make sure the tests are small, testing small sections of code at a time, to make sure that if it fails, the failing part is easily identifiable
\end{itemize}

\end{document}