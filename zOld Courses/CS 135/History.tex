\documentclass[10pt,letter]{article}
\usepackage{amsmath}
\usepackage{amssymb}
\usepackage{amsthm}
\usepackage{graphicx}
\usepackage{setspace}
\onehalfspacing
\usepackage{fullpage}
\newtheorem*{remark}{Remark}
%\date{\vspace{-5ex}}
\begin{document}

\section*{Charles Babbage}
Difference Engine: An automatic mechanical calculator designed to tabulate polynomial functions using repeated addition. \\
Analytical engine: Programmable using punched cards, 'store' where 'constants' could be held, could perform the 4 basic math functions, and had looping. 

\section*{Ada Augusta Byron}
Assisted Babbage in explaining and promoting his idea, wrote articles describing the operation and use of the analytical engine. 

\section*{David Hilbert}
Formulated the axiomatic treatment of Euclidean geometry. Asked: Is mathematics complete? Is mathematics consistent? Can we rigourously prove or disprove any mathematical statement? Hilbert believed yes. 

\section*{Kurt Godel}
Answered "Any axiom system powerful enough to describe arithmetic on integers is not complete. If it is consistent, its consistency cannot be proved within the system." His proof is basically mapping every formula possible to a natural number, and then at let $n$ represent the formula "The formula represented by $n$ is not provable." Then that formula is true but not provable, contradicting Hilbert's hypothesis. 

\section*{Alonzo Church}
With his student Kleene, he tried to answer Hilbert's question of whether or not every formula can be proven, disproven, or shown to be unprovable. He started lambda calculus. His proof of Hilbert's question was similar to Godel's, and proved that there was no way to tell if two lambda expressions were equivalent. 

\section*{Alan Turing}
Defined a different model of computation that disproved Hilbert. Theorized there could be 2 machines, the first machine could tell whether or not its input would stop, and the second machine uses the first machine to see if its input represents a coded machine that would stop when fed its own description. The second machine would halt iff it doesn't halt, therefore the first machine doesn't exist. 

\section*{John Von Neumann}
Invented RAM, CPU, fetch-execute loops, stored program. His programming language did not have recursion. 

\section*{Grace Murray Hopper}
Wrote txe first compiler, and defined first english-like data processing language. 

\section*{FORTRAN}


\section*{John McCarthy}
Made Lisp


\section*{Lisp}

\section*{Scheme}

\section*{Sussman and Steele}








\end{document}