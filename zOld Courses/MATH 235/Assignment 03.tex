\documentclass[10pt,english]{article}
\usepackage[T1]{fontenc}
\usepackage[latin9]{inputenc}
\usepackage{geometry}
\geometry{verbose,tmargin=1.5in,bmargin=1.5in,lmargin=1.5in,rmargin=1.5in}
\usepackage{amsthm}
\usepackage{amsmath}
\usepackage{amssymb}

\makeatletter
\usepackage{enumitem}
\newlength{\lyxlabelwidth}

\usepackage[T1]{fontenc}
\usepackage{ae,aecompl}

%\usepackage{txfonts}

\usepackage{microtype}

\usepackage{calc}
\usepackage{enumitem}
\setenumerate{leftmargin=!,labelindent=0pt,itemindent=0em,labelwidth=\widthof{\ref{last-item}}}

\makeatother

\usepackage{babel}
\begin{document}
\noindent \begin{center}
\textbf{\large{}MATH 146 - Assignment 1}\\
\textbf{\large{}Chris Ji 20725415}
\par\end{center}{\large \par}
\medskip{}

\begin{enumerate}
\item \begin{enumerate}
    \item Let $L:\mathbb{R}^2\rightarrow P_1(\mathbb{R})$ be defined by $L\left(\begin{bmatrix}a\\b\end{bmatrix}\right)=a+bx$. \\ 
    $L\left(s\left(\begin{bmatrix}a\\b\end{bmatrix}\right)+t\left(\begin{bmatrix}c\\d\end{bmatrix}\right)\right)=L\left(\begin{bmatrix}sa+tc\\sb+td\end{bmatrix}\right)=(sa+tc)+(sb+td)x$ and $sL\left(\begin{bmatrix}a\\b\end{bmatrix}\right)+tL\left(\begin{bmatrix}c\\d\end{bmatrix}\right)=(sa+sbx)+(tc+tdx)=sa+tc+(sb+td)x$, so therefore $L$ is linear. \\ 
    By lemma 8.4.1, $(L)$ is injective if Ker$(L)=\{\vec{0}\}$. By the rank nullity theorem, $n=2=\text{Rank}(L)+\text{nullity}(L)$. Note that a basis for the range of $L$ is $\{1,x\}$, so rank$(L)=2$. Then $n=\text{Rank}(L)+\text{nullity}(L)\Rightarrow\text{nullity}(L)=0$, and hence ker$(L)=\{\vec{0}\}$. Therefore, $(L)$ is injective. 
    Note that dim$(\mathbb{R}^2)=2=\text{dim}(P_1(\mathbb{R}))$, so therefore by theorem 8.4.3, $L$ is also surjective. 
    Since $L$ is bijective and linear, it is an isomorphism. 
    
    \item Let $A=\begin{bmatrix}a&b\\c&d\end{bmatrix}$. Then $A\begin{bmatrix}1&1\\0&1\end{bmatrix}=\begin{bmatrix}a&a+b\\c&c+d\end{bmatrix}$, and $\begin{bmatrix}1&1\\0&1\end{bmatrix}A=\begin{bmatrix}a+c&b+d\\c&d\end{bmatrix}$. Then if $\begin{bmatrix}a&a+b\\c&c+d\end{bmatrix}=\begin{bmatrix}a+c&b+d\\c&d\end{bmatrix}$, then $c=0,a=d$. Hence every matrix in $\mathbb{S}$ has the form of $\begin{bmatrix}a&b\\0&a\end{bmatrix}$. The set $\left\{\begin{bmatrix}0&1\\0&0\end{bmatrix},\begin{bmatrix}1&0\\0&1\end{bmatrix}\right\}$ is linearly dependent, and spans $\mathbb{S}$, so therefore it is a basis for $\mathbb{S}$. \\ 
    Note that $p(1)=0\Rightarrow a1+bx+cx^2=0\Rightarrow a+b+c=0$. Then every $p\in\mathbb{U}$ has the form $p(x)=(-b-c)+bx+cx^2=a(x^2-1)+b(x-1)$. Then $\{x-1,x^2-1\}$ is obviously linearly independent, and spans $\mathbb{U}$, so it is a basis for $\mathbb{U}$. \\ 
    Define $L:\mathbb{S}\rightarrow\mathbb{U}$ be defined by $L(a(x-1)+b(x^2-1))=a\begin{bmatrix}1&0\\0&1\end{bmatrix}+b\begin{bmatrix}0&1\\0&0\end{bmatrix}=\begin{bmatrix}a&b\\0&a\end{bmatrix}$. \\ 
    $$L(s(a(x^2-1)+b(x-1))+t(c(x^2-1)+d(x-1)))=L((as+ct)(x^2-1)+(bs+dt)(x-1))=\begin{bmatrix}sa+tc&sb+td\\0&sa+tc\end{bmatrix}$$ $$sL(a(x-1)+b(x^2-1))+tL(c(x-1)+d(x^2-1))=\begin{bmatrix}sa&sb\\0&sa\end{bmatrix}+\begin{bmatrix}tc&td\\0&tc\end{bmatrix}=\begin{bmatrix}sa+tc&sb+td\\0&sa+tc\end{bmatrix}$$ Therefore, $L$ is linear. \\ 
    By Lemma 8.4.1 and Theorem 8.4.3, $L$ is bijective if Ker$(L)=\{\vec{0}\}$. Let $a(x-1)+b(x^2-1)\in\text{Ker}(L)$. Then $\begin{bmatrix}0&0\\0&0\end{bmatrix}=L(a(x^2-1)+b(x-1))=\begin{bmatrix}a&b\\0&a\end{bmatrix}\Rightarrow a=b=0$. Thus $a(x-1)+b(x^2-1)=0$, and Ker$(L)=\{\vec{0}\}$. Therefore, $L$ is bijective, and since it is bijective and linear, it is an isomorphism.
\end{enumerate}

\pagebreak
\item \begin{enumerate}
    \item If $L$ is onto, then for every vector $\vec{u}\in\mathbb{U}$, there exists a $\vec{v}\in\mathbb{V}$ such that $L(\vec{v})=\vec{u}$. If $M$ is onto, then for every vector $\vec{w}\in\mathbb{W}$, there exists a $\vec{u}\in\mathbb{U}$ such that $L(\vec{u})=\vec{w}$. $(M\circ L)$ is onto if, for every vector $\vec{w}\in\mathbb{W}$, there exists a $\vec{v}\in\mathbb{V}$ such that $(M\circ L)(\vec{v})=\vec{w}$. Then $(M\circ L)(\vec{v})=M(L(\vec{v}))$. Note that $L$ is surjective, so $M(L(\vec{v}))=M(\vec{u})$ for every $\vec{u}\in\mathbb{U}$. Note that $M$ is surjective, so $M(\vec{u})=\vec{w}$, for every $\vec{w}\in\mathbb{W}$. Hence, if $L$ and $M$ are surjective, $M\circ L$ is also surjective. 
    
    \item Let $L:\mathbb{R}^2\rightarrow\mathbb{R}^2$ be defined as $L\left(\begin{bmatrix}a\\b\end{bmatrix}\right)=\begin{bmatrix}a\\b\end{bmatrix}$ and $M:\mathbb{R}^2\rightarrow\mathbb{R}^3$ be defined as $M\left(\begin{bmatrix}a\\b\end{bmatrix}\right)=\begin{bmatrix}a\\b\\b\end{bmatrix}$. 
    Note that $M$ is linear because $M\left(s\begin{bmatrix}a\\b\end{bmatrix}+t\begin{bmatrix}c\\d\end{bmatrix}\right)=\begin{bmatrix}sa\\sb\\sb\end{bmatrix}+\begin{bmatrix}tc\\td\\td\end{bmatrix}=sL\left(\begin{bmatrix}a\\b\end{bmatrix}\right)+tL\left(\begin{bmatrix}c\\d\end{bmatrix}\right)$
    Then $M\circ L$ is not onto, as any vector of the form $\begin{bmatrix}a\\b\\c\end{bmatrix}$ such that $b\neq c$ is not in Range$(M\circ L)$. 
\end{enumerate}



\pagebreak
\item By definition, $[L(\vec{v})]_\mathcal{B}=[L]_\mathcal{B}[\vec{v}]_\mathcal{B}=[L]_\mathcal{B}\vec{x}$. But then if $\vec{x}$ is an eigenvector of $[L]_\mathcal{B}$, $[L]_\mathcal{B}\vec{x}=\lambda\vec{x}=\lambda[\vec{v}]_\mathcal{B}$. Then from this, we can see $[L(\vec{v})]_\mathcal{B}=\lambda[\vec{v}]_\mathcal{B}$. Now note that converting a vector to its coordinates representation is an isomorphic function (proof below, as question 5(a) for your convenience i copy-pasted it), so we can apply its inverse to both sides to get $L(\vec{v})=\lambda\vec{v}$, as required. \\ 
Let $\Phi_\mathcal{B}:\mathbb{V}\rightarrow \mathbb{R}^n$ be defined as $\Phi_{\mathcal{B}}(\vec{v})=\Phi_{\mathcal{B}}(b_1v_1+b_2v_2+\ldots+b_nv_n)=\begin{bmatrix}b_1\\\vdots\\b_n\end{bmatrix}$, for all $\vec{v}\in \mathbb{V}$. Let $\vec{a}=a_1v_1+\ldots+a_nv_n,\vec{b}=b_1v_1+\ldots+b_nv_n$ be vectors in $\mathbb{V}$. Then $\Phi_{\mathcal{B}}(s\vec{v}+t\vec{w})=\Phi_{\mathcal{B}}(s(a_1v_1+\ldots+a_nv_n)+t(b_1v_1+\ldots+b_nv_n))=\begin{bmatrix}sa_1\\\vdots\\sa_n\end{bmatrix}+\begin{bmatrix}tb_1\\\vdots\\tb_n\end{bmatrix}=s\begin{bmatrix}a_1\\\vdots\\a_n\end{bmatrix}+t\begin{bmatrix}b_1\\\vdots\\b_n\end{bmatrix}$ \\ 
$s\Phi_\mathcal{B}(a_1v_1+\ldots+a_nv_n)+t\Phi_\mathcal{B}(b_1v_1+\ldots+b_nv_n)=s\begin{bmatrix}b_1\\\vdots\\b_n\end{bmatrix}+t\begin{bmatrix}b_1\\\vdots\\b_n\end{bmatrix}$, as required. \\ 
Note that dim$(V)=\text{Rank}(\Phi_\mathcal{B})+\text{nullity}(\Phi_\mathcal{B})$. Note that $\text{dim}(V)=n=\text{Rank}(\Phi_\mathcal{B})$ ($\mathbb{V}$ has $n$ vectors, by definition), so therefore, nullity$(\Phi_\mathcal{B})=0$, and so $\text{Ker}(\Phi_\mathcal{B})=\{\vec{0}\}$, and so $\Phi_\mathcal{B}$ is bijective, and an isomorphism. 




\pagebreak
\item $L(a_1,a_2)=(a_1+a_2)+a_1x^2\Rightarrow L\left(\begin{bmatrix}a_1\\a_2\end{bmatrix}\right)=\begin{bmatrix}a_1+a_2\\0\\a_1\end{bmatrix}$. $\mathcal{B}=\left\{\begin{bmatrix}1\\-1\end{bmatrix},\begin{bmatrix}1\\2\end{bmatrix}\right\}$, and $\mathcal{C}=\{1+x^2,1+x,-1-x+x^2\}=\left\{\begin{bmatrix}1\\0\\1\end{bmatrix},\begin{bmatrix}1\\1\\0\end{bmatrix}\begin{bmatrix}-1\\-1\\1\end{bmatrix}\right\}$. By definition, $_\mathcal{C}[L]_\mathcal{B}=\left[[L(\vec{v}_1)]_\mathcal{C}\,\,\,[L(\vec{v}_2)]_\mathcal{C}\right]=\left[[L\left(\begin{bmatrix}1\\-1\end{bmatrix}\right)]_\mathcal{C}[L\left(\begin{bmatrix}1\\2\end{bmatrix}\right)]_\mathcal{C}\right]=\left[\begin{bmatrix}0\\0\\1\end{bmatrix}_\mathcal{C}\begin{bmatrix}3\\0\\1\end{bmatrix}_\mathcal{C}\right]=\begin{bmatrix}0&3\\0&-2\\1&-2\end{bmatrix}$

\pagebreak
\item \begin{enumerate}
    \item Note that $\text{Range}(L)=\{L(\vec{v})|\vec{v}\in\mathbb{V}\}=\{L(\vec{v}_1),\ldots,L(\vec{v}_n)\}$, and $\text{Col}(A)=\{[L(\vec{v}_1)]_\mathcal{C},\ldots,[L(\vec{v}_n)]_\mathcal{C}\}$. Then this question is just asking us to prove taking a vector to its $\mathcal{C}$-coordinates is isomorphic. Let $\vec{v}\in\mathbb{V}$ be represented as a linear combination of all the $v_i$'s in $\mathbb{V}$, aka let every $\vec{v}\in\mathbb{V}$ be represented by $b_1v_1+b_2v_2$, for all $b_i\in\mathbb{N}$. Then clearly we can define $\Phi_\mathcal{B}:\mathbb{V}\rightarrow \mathbb{W}$ as $\Phi_{\mathcal{B}}(\vec{v})=\Phi_{\mathcal{B}}(b_1v_1+b_2v_2+\ldots+b_nv_n)=\begin{bmatrix}b_1\\\vdots\\b_n\end{bmatrix}$, for all $\vec{v}\in \mathbb{V}$. Let $\vec{a}=a_1v_1+\ldots+a_nv_n,\vec{b}=b_1v_1+\ldots+b_nv_n$ be vectors in $\mathbb{V}$. Then $\Phi_{\mathcal{B}}(s\vec{v}+t\vec{w})=\Phi_{\mathcal{B}}(s(a_1v_1+\ldots+a_nv_n)+t(b_1v_1+\ldots+b_nv_n))=\begin{bmatrix}sa_1\\\vdots\\sa_n\end{bmatrix}+\begin{bmatrix}tb_1\\\vdots\\tb_n\end{bmatrix}=s\begin{bmatrix}a_1\\\vdots\\a_n\end{bmatrix}+t\begin{bmatrix}b_1\\\vdots\\b_n\end{bmatrix}$ \\ 
$s\Phi_\mathcal{B}(a_1v_1+\ldots+a_nv_n)+t\Phi_\mathcal{B}(b_1v_1+\ldots+b_nv_n)=s\begin{bmatrix}b_1\\\vdots\\b_n\end{bmatrix}+t\begin{bmatrix}b_1\\\vdots\\b_n\end{bmatrix}$, as required. \\ 
Note that dim$(V)=\text{Rank}(\Phi_\mathcal{B})+\text{nullity}(\Phi_\mathcal{B})$. Note that $\text{dim}(V)=n=\text{Rank}(\Phi_\mathcal{B})$, so therefore, nullity$(\Phi_\mathcal{B})=0$, and so $\text{Ker}(\Phi_\mathcal{B})=\{\vec{0}\}$, and so $\Phi_\mathcal{B}$ is bijective, and an isomorphism. 
    
    \item As $\text{Range}(L)=\{L(\vec{v})|\vec{v}\in\mathbb{V}\}=\{L(\vec{v}_1),\ldots,L(\vec{v}_n)\}$, and $\text{Col}(A)=\{[L(\vec{v}_1)]_\mathcal{C},\ldots,[L(\vec{v}_n)]_\mathcal{C}\}$, then $\text{Rank}(L)=\text{dim}(\text{Range}(L))=n=\text{dim}(\text{Col}(A))=\text{Rank}(A)$. 
    
\end{enumerate}


\end{enumerate}

\end{document}
