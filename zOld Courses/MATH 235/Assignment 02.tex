\documentclass[10pt,english]{article}
\usepackage[T1]{fontenc}
\usepackage[latin9]{inputenc}
\usepackage{geometry}
\geometry{verbose,tmargin=1.5in,bmargin=1.5in,lmargin=1.5in,rmargin=1.5in}
\usepackage{amsthm}
\usepackage{amsmath}
\usepackage{amssymb}

\makeatletter
\usepackage{enumitem}
\newlength{\lyxlabelwidth}

\usepackage[T1]{fontenc}
\usepackage{ae,aecompl}

%\usepackage{txfonts}

\usepackage{microtype}

\usepackage{calc}
\usepackage{enumitem}
\setenumerate{leftmargin=!,labelindent=0pt,itemindent=0em,labelwidth=\widthof{\ref{last-item}}}

\makeatletter
\renewcommand*\env@matrix[1][*\c@MaxMatrixCols c]{%
   \hskip -\arraycolsep
   \let\@ifnextchar\new@ifnextchar
   \array{#1}}
\makeatother

\usepackage{babel}
\begin{document}
\noindent \begin{center}
\textbf{\large{}MATH 235 - Assignment 2}\\
\textbf{\large{}Chris Ji 20725415}
\par\end{center}{\large \par}
\medskip{}

\begin{enumerate}
\item \begin{enumerate}
    \item  By definition, $\text{Range}(L)=\{L(\vec{v})|\vec{v}\in\mathbb{V})\}$. Every $L(\vec{v})$ in $\mathbb{W}$ is just a scalar number, so the range is $\{\begin{bmatrix}1\end{bmatrix}\}$. This set is obviously linearly independent, so it is a basis for Range$(L)$. 
    
    By definition, $\text{Ker}(L)=\{\vec{v}\in\mathbb{V}|L(\vec{v})=\vec{0}_\mathbb{W}\}$. Then $0+0=\text{tr}\left(\begin{bmatrix}a&b\\c&d\end{bmatrix}\right)=a+d$ implies that $a=-d$. Then Ker$(L)=\left\{\begin{bmatrix}0&1\\0&0\end{bmatrix},\begin{bmatrix}0&0\\1&0\end{bmatrix},\begin{bmatrix}1&0\\0&-1\end{bmatrix}\right\}$. This set is obviously linearly independent, so it is a basis for Ker$(L)$. 
     Notice that $\text{dim}M_{2\times2}(\mathbb{R})=4=\text{rank}(L)+\text{nullity}(L)=1+3$
   
    
    
    \item  Every vector in the range of $L$ has the form $(a+c)+bx+(a+b+c)x^2=a+ax^2+bx+bx^2+c+cx^2=a(1+x^2)+b(x+x^2)+c(1+x^2)=(a+c)(1+x^2)+b(x+x^2)$. Hence the set $\{1+x^2,x+x^2\}$ spans the range of $(L)$ and is linearly independent, so it is a basis for Range$(L)$. \\ 
    Let $\begin{bmatrix}a\\b\\c\end{bmatrix}\in\text{Ker}(L)$. Then $0+0x+0x^2=L(a+bx+cx^2)=\begin{bmatrix}a+c\\b\\a+b+c\end{bmatrix}$. This implies that $a=-c$, and $b=0$. Thus, every vector in Ker$(L)$ has the form $\begin{bmatrix}a\\b\\c\end{bmatrix}=\begin{bmatrix}-c\\0\\c\end{bmatrix}=c\begin{bmatrix}-1\\0\\1\end{bmatrix}$, so a basis for Ker$(L)$ is $\left\{\begin{bmatrix}-1\\0\\1\end{bmatrix}\right\}$ \\ 
    Notice that $\text{dim}P_2(\mathbb{R})=3=\text{rank}(L)+\text{nullity}(L)=2+1$
    
    
\end{enumerate}

\pagebreak
\item Changing $P_2(\mathbb{R})$ to $\mathbb{R}^3$, $L\left(\begin{bmatrix}a\\b\\c\end{bmatrix}\right)=\begin{bmatrix}a\\0\\b+c\end{bmatrix}$, and $\mathcal{B}=\left\{\begin{bmatrix}1\\0\\1\end{bmatrix},\begin{bmatrix}-1\\1\\0\end{bmatrix},\begin{bmatrix}1\\-1\\1\end{bmatrix}\right\}$. By definition, $[L]_\mathcal{B}=\left[[L(\vec{v}_1)]_\mathcal{B}\quad[L(\vec{v}_2)]_\mathcal{B}\quad[L(\vec{v}_3)]_\mathcal{B}\right]=\left[L\left(\begin{bmatrix}1\\0\\1\end{bmatrix}\right)L\left(\begin{bmatrix}-1\\1\\0\end{bmatrix}\right)L\left(\begin{bmatrix}1\\-1\\1\end{bmatrix}\right)\right]$. Applying $L$ to each of these, 

$$L\left(\begin{bmatrix}1\\0\\1\end{bmatrix}\right)=\begin{bmatrix}1\\0\\0+1\end{bmatrix}$$ $$L\left(\begin{bmatrix}-1\\1\\0\end{bmatrix}\right)=\begin{bmatrix}-1\\0\\1+0\end{bmatrix}$$ $$L\left(\begin{bmatrix}1\\-1\\1\end{bmatrix}\right)=\begin{bmatrix}1\\0\\(-1)+1\end{bmatrix}$$ So, 

$$[L]_\mathcal{B}=\begin{bmatrix}1&-1&1\\0&0&0\\1&1&0\end{bmatrix}$$




\pagebreak
\item Changing $P_2(\mathbb{R})$ to $\mathbb{R}^3$, $\mathcal{B}=\left\{\begin{bmatrix}1\\0\\0\end{bmatrix},\begin{bmatrix}-1\\1\\0\end{bmatrix},\begin{bmatrix}1\\-2\\1\end{bmatrix}\right\}$, and $\mathcal{C}=\left\{\begin{bmatrix}1\\0\\0\end{bmatrix},\begin{bmatrix}1\\1\\0\end{bmatrix},\begin{bmatrix}1\\2\\1\end{bmatrix}\right\}$. Then, we need to solve $_\mathcal{C}[L]_\mathcal{B}=\left[[L(\vec{v}_1)]_\mathcal{C}\ldots[L(\vec{v}_n)]_\mathcal{C}\right]$\\ 


To find the $\mathcal{C}$-coordinates of the images of the vectors in $\mathcal{B}$ under $L$, we will solve the multiple augmented matrix, with the left side coming from comparing entries from $\mathcal{C}$, and each vector on the right side being the vectors in $\mathcal{B}$. $$\left[\begin{array}{ccc|ccc}1 & 1 & 1 & 1 & -1 & 1 \\0 & 1 & 2 & 0 & 1 & -2 \\0 & 0 & 1 & 0 & 0 & 1 \end{array}\right]\sim \left[\begin{array}{ccc|ccc}1 & 0 & 0 & 1 & -2 & 4\\0 & 1 & 0 & 0 & 1 & -4\\0 & 0 & 1 & 0 & 0 & 1\end{array}\right]$$ Hence $[L(\vec{v}_1)]_\mathcal{C}=\begin{bmatrix}1\\0\\0\end{bmatrix}, [L(\vec{v}_2)]_\mathcal{C}=\begin{bmatrix}-2\\1\\0\end{bmatrix}, [L(\vec{v}_3)]_\mathcal{C}=\begin{bmatrix}4\\-4\\1\end{bmatrix}$, and so $$_\mathcal{C}[L]_\mathcal{B}=[[L(\vec{v}_1)]_\mathcal{C}\quad[L(\vec{v}_2)]_\mathcal{C}\quad L(\vec{v}_3)]_\mathcal{C}]=\begin{bmatrix}1&-2&4\\0&1&-4\\0&0&1\end{bmatrix}$$


\pagebreak
\item By the rank-nullity theorem, $n=\text{rank}(L)+\text{nullity}(L)=\text{dim}(\text{Range}(L))+\text{dim}(\text{Ker}(L))$. $n$ is $n$, and dim$(\text{Range}(L))$ is also $n$, since $\{L(\vec{v}_1),\ldots,L(\vec{v}_n)\}$ is a basis for $\mathbb{V}$. Therefore, nullity$(L)=0=\text{dim}(\text{Ker}(L))\Rightarrow\text{Ker}(L)=\{\vec{0}\}$. 

\pagebreak
\item \begin{enumerate}
    \item This is not true. If $\{L(\vec{v}_1),\ldots,L(\vec{v}_n)\}$ is a basis for $\mathbb{W}$ (and hence the range of $L$), it does not necessarily mean that $\{\vec{v}_1,\ldots,\vec{v}_n\}$ forms a basis for the domain. Take $L:\mathbb{R}^3\rightarrow\mathbb{R}^2$ defined by $L\left(\begin{bmatrix}a\\b\\c\end{bmatrix}\right)=\begin{bmatrix}a\\b\end{bmatrix}$. Then $\left\{\begin{bmatrix}1\\0\end{bmatrix},\begin{bmatrix}0\\1\end{bmatrix}\right\}$ forms a basis for $\mathbb{W}$, but $\mathcal{B}=\left\{\begin{bmatrix}1\\0\\1\end{bmatrix},\begin{bmatrix}0\\1\\1\end{bmatrix}\right\}$ can not be a basis for $\mathbb{V}$, as $\text{dim}\mathcal{B}=2\neq3=\text{dim}\mathbb{R}^3$
    
    \item This is not true. Take $\mathbb{V}=\mathbb{R}^3=\mathbb{W}$. Then define $L:\mathbb{V}\rightarrow\mathbb{W}$ as $L\left(\begin{bmatrix}a\\b\\c\end{bmatrix}\right)=\begin{bmatrix}0\\c\\b\end{bmatrix}$. Note that a basis for Ker$(L)$ is $\left\{\begin{bmatrix}1\\0\\0\end{bmatrix}\right\}$, which is not the zero vector, hence $\text{dim}(\text{Ker}(L))=\text{nullity}(L)\neq0$
    
    \item If ker$(L)\neq\{\vec{0}\}$, then nullity$(L)\neq0$. By rank nullity theorem, $n=\text{Rank}(L)+\text{nullity}(L)$. Since nullity$(L)\neq0$, Rank$(L)$ must be less than $n$. Therefore, $\{L(\vec{v}_1),\ldots,L(\vec{v}_n)\}$, must be linearly independent, as the range of $L$ only contains $(n-\text{nullity}(L))$ vectors. Therefore, this statement is true.
    
    
    \item Assume that $\{\vec{v}_1,\ldots,\vec{v}_n\}$ is linearly dependent. Then there exists $\{c_1,\ldots,c_k\}$, not all zero, such that $\sum_{i=0}^nc_iv_i=0$. Applying $L$ to both sides, $L(\sum_{i=0}^nc_iv_i)=L(0)$. Applying linearity, $\sum_{i=0}^nc_iL(v_i)=0$. But then, this is the definition of $L(v_i)$ being linearly dependent, which by the question is false. Therefore, $\{\vec{v}_1,\ldots,\vec{v}_n\}$ is linearly independent, and so this statement is true. 
    
    
\end{enumerate}
\end{enumerate}

\end{document}
