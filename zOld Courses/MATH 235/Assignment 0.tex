\documentclass[10pt,english]{article}
\usepackage[T1]{fontenc}
\usepackage[latin9]{inputenc}
\usepackage{geometry}
\geometry{verbose,tmargin=1.5in,bmargin=1.5in,lmargin=1.5in,rmargin=1.5in}
\usepackage{amsthm}
\usepackage{amsmath}
\usepackage{amssymb}

\makeatletter
\usepackage{enumitem}
\newlength{\lyxlabelwidth}

\usepackage[T1]{fontenc}
\usepackage{ae,aecompl}

%\usepackage{txfonts}

\usepackage{microtype}

\usepackage{calc}
\usepackage{enumitem}
\setenumerate{leftmargin=!,labelindent=0pt,itemindent=0em,labelwidth=\widthof{\ref{last-item}}}

\makeatother

\usepackage{babel}
\begin{document}
\noindent \begin{center}
\textbf{\large{}MATH 235 - Assignment 0}\\
\textbf{\large{}Chris Ji 20725415}
\par\end{center}{\large \par}
\medskip{}

\begin{enumerate}
\item We define the \textbf{span} $S$ of set $B$ by $S=\text{Span }B=\text{Span}\{\vec{v_1},\ldots,\vec{v_k}\}=\{t_1\vec{v_1}+\ldots+t_k\vec{v_k} | t_1,\ldots,t_k\in\mathbb{R}\}$. We also say that $S$ is \textbf{spanned} by $B$ and that $B$ is a \textbf{spanning set} for $S$. \pagebreak
\item A set of vectors $\{\vec{v}_1,\ldots,\vec{v}_k\}$ in $\mathbb{R}^n$ is said to be \textbf{linearly dependent} if there exist coefficients $c_1,\ldots,c_k$ not all zero such that $\vec{0}=c_1\vec{v}_1+\ldots+c_k\vec{v}_k$. A set of vectors $\{\vec{v}_1,\ldots,\vec{v}_k\}$ is said to be \textbf{linearly independent} if the only solution to $\vec{0}=c_1\vec{v}_1+\ldots+c_k\vec{v}_k$ is $c_1=c_2=\ldots=c_k=0$, called the \textbf{trivial solution}.\pagebreak
\item A subset $\mathbb{S}$ of $\mathbb{R}^n$ is called a \textbf{subspace} of $\mathbb{R}^n$ if for every $\vec{x},\vec{y},\vec{w}\in\mathbb{S}$ and $c,d\in\mathbb{R}$, we have \begin{enumerate}
    \item $\vec{x}+\vec{y}\in\mathbb{S}$ 
    \item $(\vec{x}+\vec{y})+\vec{w}=\vec{x}+(\vec{y}+\vec{w})$ 
    \item $\vec{x}+\vec{y}=\vec{y}+\vec{x}$ 
    \item There exists a vector $\vec{0}\in\mathbb{S}$ called the \textbf{zero vector}, such that $\vec{x}+\vec{0}=\vec{x}$ for all $\vec{x}\in\mathbb{S}$. 
    \item There exists a vector $(-\vec{x})\in\mathbb{S}$ such that $\vec{x}+(-\vec{x})=\vec{0}$ 
    \item $c\vec{x}\in\mathbb{S}$ 
    \item $c(d\vec{x})=(cd)\vec{x}$ 
    \item $(c+d)\vec{x}=c\vec{x}+d\vec{x}$ 
    \item $c(\vec{x}+\vec{y})=c\vec{x}+c\vec{y}$
    \item $1\vec{x}=\vec{x}$
\end{enumerate} \pagebreak
\item If a subset of $S$ of $\mathbb{R}^n$ is spanned by a set of vectors $\{\vec{v}_1,\ldots,\vec{v}_k\}$ where the set is linearly independent, then $\{\vec{v}_1,\ldots,\vec{v}_k\}$ is called a \textbf{basis} for $S$. We define a basis for the set $\{\vec{0}\}$ to be the empty set.\pagebreak
\item The \textbf{rank} of a matrix $A$ is the number of leading ones in the RREF of the matrix and is denoted rank $A$.\pagebreak
\item Let $L:\mathbb{R}^n\rightarrow\mathbb{R}^n$ and $M:\mathbb{R}^n\rightarrow\mathbb{R}^n$ be linear mappings. If $(L\circ M)(\vec{x})=\vec{x}$ and $(M\circ L)(\vec{x})=\vec{x}$ for all $\vec{x}$, then $L$ and $M$ are said to be \textbf{invertible}. We write $M=L^{-1}$ and $L=M^{-1}$.
\end{enumerate}

\end{document}
