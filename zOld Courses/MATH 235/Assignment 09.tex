\documentclass[10pt,english]{article}
\usepackage[T1]{fontenc}
\usepackage[latin9]{inputenc}
\usepackage{geometry}
\geometry{verbose,tmargin=1.5in,bmargin=1.5in,lmargin=1.5in,rmargin=1.5in}
\usepackage{amsthm}
\usepackage{amsmath}
\usepackage{amssymb}

\makeatletter
\usepackage{enumitem}
\newlength{\lyxlabelwidth}

\usepackage[T1]{fontenc}
\usepackage{ae,aecompl}

%\usepackage{txfonts}

\usepackage{microtype}

\usepackage{calc}
\usepackage{enumitem}
\setenumerate{leftmargin=!,labelindent=0pt,itemindent=0em,labelwidth=\widthof{\ref{last-item}}}

\newenvironment{amatrix}[3]{%
  \left[\begin{array}{@{}*{#1}{c}|c@{}}
}{%
  \end{array}\right]
}

\makeatother

\usepackage{babel}
\begin{document}
\noindent \begin{center}
\textbf{\large{}MATH 235 - Assignment 9}\\
\textbf{\large{}Chris Ji 20725415}
\par\end{center}{\large \par}
\medskip{}

\begin{enumerate}
    \item Note that $Q(\vec{x})$ has corresponding symmetric matrix $\begin{bmatrix}7&6\\6&12\end{bmatrix}$. Then by theorem 10.5.1, the maximum and minimum of $Q(\vec{x})$ subject to constraint $||\vec{x}||=1$ are the largest and smallest eigenvalues of $A$, respectively. $\left|\begin{matrix}7-\lambda&6\\6&12-\lambda\end{matrix}\right|=\lambda^2-19\lambda+48=(\lambda-3)(\lambda-16)$. Then the maximum and minimum of $Q(\vec{x})$ are $16$ and $3$, respectively.
    
    \pagebreak
    \item \begin{enumerate}
        \item $A^TA=\begin{bmatrix}6&6\\6&11\end{bmatrix}$, which has eigenvalues $15,2$ which correspond to the normal eigenvectors $\begin{bmatrix}\frac{2}{\sqrt{13}}\\\frac{3}{\sqrt{13}}\end{bmatrix},\begin{bmatrix}-\frac{3}{\sqrt{13}}\\\frac{2}{\sqrt{13}}\end{bmatrix}$, which will be our columns for $V$. \\ 
        $\Sigma=\begin{bmatrix}\sqrt{15}&0\\0&\sqrt{2}\\0&0\end{bmatrix}$, from above.\\
        
        $\vec{u}_1=\frac{1}{\sigma_1}A\vec{v}_1=\begin{bmatrix}\frac{11\sqrt{195}}{195}\\\frac{7\sqrt{195}}{195}\\\frac{\sqrt{195}}{39}\end{bmatrix}$, $\vec{u}_2=\frac{1}{\sigma_2}A\vec{v}_2=\begin{bmatrix}\frac{3\sqrt{26}}{26}\\-\frac{2\sqrt{26}}{13}\\-\frac{\sqrt{26}}{26}\end{bmatrix}$ A basis for $\text{Null }A^T$ is $\left\{\begin{bmatrix}-\frac{1}{5}\\-\frac{2}{5}\\1\end{bmatrix}\right\}$ and applying the gram-schmidt procedure to these we get the vectors $\begin{bmatrix}\frac{11\sqrt{195}}{195}\\\frac{7\sqrt{195}}{195}\\\frac{\sqrt{195}}{39}\end{bmatrix},\begin{bmatrix}\frac{3\sqrt{26}}{26}\\-\frac{2\sqrt{26}}{13}\\-\frac{\sqrt{26}}{26}\end{bmatrix},\begin{bmatrix}-\frac{\sqrt{30}}{30}\\-\frac{\sqrt{30}}{15}\\\frac{\sqrt{30}}{6}\end{bmatrix}$: the columns for $U$. \\
        Then $U=\begin{bmatrix}\frac{11\sqrt{195}}{195}&\frac{3\sqrt{26}}{26}&-\frac{\sqrt{30}}{30}\\\frac{7\sqrt{195}}{195}&-\frac{2\sqrt{26}}{13}&-\frac{\sqrt{30}}{15}\\\frac{\sqrt{195}}{39}&-\frac{\sqrt{26}}{26}&\frac{\sqrt{30}}{6}\end{bmatrix}, \Sigma=\begin{bmatrix}\sqrt{15}&0\\0&\sqrt{2}\\0&0\end{bmatrix}, V=\begin{bmatrix}\frac{2}{\sqrt{13}}&-\frac{3}{\sqrt{13}}\\\frac{3}{\sqrt{13}}&\frac{2}{\sqrt{13}}\end{bmatrix}$, and $A=U\Sigma V^T$. 
        
        \item $A^TA=\begin{bmatrix}12&-6\\-6&3\end{bmatrix}$, which has eigenvalues $15,0$ which correspond to the normal eigenvectors $\begin{bmatrix}-\frac{2}{\sqrt{5}}\\\frac{1}{\sqrt{5}}\end{bmatrix},\begin{bmatrix}\frac{1}{\sqrt{5}}\\\frac{2}{\sqrt{5}}\end{bmatrix}$, which will be our colums for $V$. Then $\Sigma=\begin{bmatrix}\sqrt{15}&0\\0&0\\0&0\end{bmatrix}$\\ 
        $\vec{u}_1=\frac{1}{\sigma_1}A\vec{v}_1=\begin{bmatrix}-\frac{\sqrt{3}}{3}\\-\frac{\sqrt{3}}{3}\\-\frac{\sqrt{3}}{3}\end{bmatrix}$. A basis for $\text{Null }A^T$ is $\left\{\begin{bmatrix}-1\\1\\0\end{bmatrix},\begin{bmatrix}-1\\0\\1\end{bmatrix}\right\}$, and applying the Gram-Schmidt procedure to these we get the vectors $\begin{bmatrix}-\frac{\sqrt{3}}{3}\\-\frac{\sqrt{3}}{3}\\-\frac{\sqrt{3}}{3}\end{bmatrix},\begin{bmatrix}-\frac{\sqrt{2}}{2}\\\frac{\sqrt{2}}{2}\\0\end{bmatrix},\begin{bmatrix}-\frac{\sqrt{6}}{6}\\-\frac{\sqrt{6}}{6}\\\frac{\sqrt{6}}{3}\end{bmatrix}$, the columns for $U$. \\ 
        Then $U=\begin{bmatrix}-\frac{\sqrt{3}}{3}&-\frac{\sqrt{2}}{2}&-\frac{\sqrt{6}}{6}\\-\frac{\sqrt{3}}{3}&\frac{\sqrt{2}}{2}&-\frac{\sqrt{6}}{6}\\-\frac{\sqrt{3}}{3}&0&\frac{\sqrt{6}}{3}\end{bmatrix},\Sigma=\begin{bmatrix}\sqrt{15}&0\\0&0\\0&0\end{bmatrix},V=\begin{bmatrix}-\frac{2}{\sqrt{5}}&\frac{1}{\sqrt{5}}\\\frac{1}{\sqrt{5}}&\frac{2}{\sqrt{5}}\end{bmatrix}$, and $A=U\Sigma V^T$. 
        \pagebreak
        \item $A^TA=\begin{bmatrix}3&0&0\\0&3&0\\0&0&3\end{bmatrix}$, which has eigenvalues $3,3,3$, which correspond to the eigenvectors $e_1,e_2,e_3$, and so $V$ is the identity matrix. $\Sigma=\begin{bmatrix}\sqrt{3}&0&0\\0&\sqrt{3}&0\\0&0&\sqrt{3}\end{bmatrix}$. $\vec{u}_1=\begin{bmatrix}\frac{1}{\sqrt{3}}\\\frac{1}{\sqrt{3}}\\-\frac{1}{\sqrt{3}}\\0\end{bmatrix},\vec{u}_2=\begin{bmatrix}0\\\frac{1}{\sqrt{3}}\\\frac{1}{\sqrt{3}}\\\frac{1}{\sqrt{3}}\end{bmatrix},\vec{u}_3=\begin{bmatrix}\frac{1}{\sqrt{3}}\\-\frac{1}{\sqrt{3}}\\0\\\frac{1}{\sqrt{3}}\end{bmatrix}$, and the basis for the nullspace of $A^T$, $\left\{\begin{bmatrix}-1\\0\\-1\\1\end{bmatrix}\right\}$gives us $\vec{u}_4=\begin{bmatrix}-\frac{1}{\sqrt{3}}\\0\\-\frac{1}{\sqrt{3}}\\\frac{1}{\sqrt{3}}\end{bmatrix}$.\\
        Then $U=\begin{bmatrix}\frac{1}{\sqrt{3}}&0&\frac{1}{\sqrt{3}}&-\frac{1}{\sqrt{3}}\\\frac{1}{\sqrt{3}}&\frac{1}{\sqrt{3}}&-\frac{1}{\sqrt{3}}&0\\-\frac{1}{\sqrt{3}}&\frac{1}{\sqrt{3}}&0&-\frac{1}{\sqrt{3}}\\0&\frac{1}{\sqrt{3}}&\frac{1}{\sqrt{3}}&\frac{1}{\sqrt{3}}\end{bmatrix}, \Sigma=\begin{bmatrix}\sqrt{3}&0&0\\0&\sqrt{3}&0\\0&0&\sqrt{3}\end{bmatrix}, V=\begin{bmatrix}1&0&0\\0&1&0\\0&0&1\end{bmatrix}$, and $A=U\Sigma V^T$. 
    \end{enumerate}
    
    \pagebreak
    \item \begin{enumerate}
        \item We have $A^TA=\begin{bmatrix}4&6&0\\6&13&0\\0&0&1\end{bmatrix}$, which has eigenvalues $16,1,1$ which correspond to normal eigenvectors $\begin{bmatrix}\frac{1}{\sqrt{5}}\\\frac{2}{\sqrt{5}}\\0\end{bmatrix},\begin{bmatrix}0\\0\\1\end{bmatrix},\begin{bmatrix}-\frac{2}{\sqrt{5}}\\\frac{1}{\sqrt{5}}\\0\end{bmatrix}$, respectively. Then our basis $\mathcal{B}=\left\{\begin{bmatrix}\frac{1}{\sqrt{5}}\\\frac{2}{\sqrt{5}}\\0\end{bmatrix},\begin{bmatrix}0\\0\\1\end{bmatrix},\begin{bmatrix}-\frac{2}{\sqrt{5}}\\\frac{1}{\sqrt{5}}\\0\end{bmatrix}\right\}$. We know these vectors are orthogonal, so they are linearly independent, and hence form a basis for $\mathbb{R}^3$. 
        \item With our work from above, we can find the left singular vectors by calculating $\vec{u}_1=\frac{1}{\sigma_1}A\vec{v}_1=\frac{1}{\sqrt{16}}A\begin{bmatrix}\frac{1}{\sqrt{5}}\\\frac{2}{\sqrt{5}}\\0\end{bmatrix}=\begin{bmatrix}\frac{2}{\sqrt{5}}\\\frac{1}{\sqrt{5}}\\0\end{bmatrix},\vec{u}_2=\begin{bmatrix}0\\0\\1\end{bmatrix}\vec{u}_3=\begin{bmatrix}-\frac{1}{\sqrt{5}}\\\frac{2}{\sqrt{5}}\\0\end{bmatrix}$, and so $\mathcal{C}=\left\{\begin{bmatrix}\frac{1}{\sqrt{5}}\\\frac{2}{\sqrt{5}}\\0\end{bmatrix},\begin{bmatrix}0\\0\\1\end{bmatrix},\begin{bmatrix}-\frac{1}{\sqrt{5}}\\\frac{2}{\sqrt{5}}\\0\end{bmatrix}\right\}$. We know these vectors are orthogonal, so they are linearly independent, and hence form a basis for $\mathbb{R}^3$. 
        \item $_C[L]_B=\left[\begin{bmatrix}\frac{1}{\sqrt{5}}\\\frac{2}{\sqrt{5}}\\0\end{bmatrix}_\mathcal{C}\quad\begin{bmatrix}0\\0\\1\end{bmatrix}_\mathcal{C}\quad\begin{bmatrix}-\frac{2}{\sqrt{5}}\\\frac{1}{5}\\0\end{bmatrix}_\mathcal{C}\right]$. Then we need to row reduce the multiple augmented matrix $\left[\begin{array}{ccc|ccc}\frac{1}{\sqrt{5}}&0&-\frac{2}{\sqrt{5}}&\frac{1}{\sqrt{5}}&0&-\frac{1}{\sqrt{5}}\\\frac{2}{\sqrt{5}}&0&\frac{1}{\sqrt{5}}&\frac{2}{\sqrt{5}}&0&\frac{2}{\sqrt{5}}\\0&1&0&0&1&0\end{array}\right]\sim\left[\begin{array}{ccc|ccc}1&0&0&1&0&\frac{3}{5}\\0&1&0&0&1&0\\0&0&1&0&0&\frac{4}{5}\end{array}\right]$, and so $_C[L]_B=\begin{bmatrix}1&0&\frac{3}{5}\\0&1&0\\0&0&\frac{4}{5}\end{bmatrix}$
    \end{enumerate}
    
    \pagebreak
    \item If $\vec{v}$ is a right singular vector of $A$ such that $A\vec{v}=\vec{0}$, then clearly $\vec{v}$ is an eigenvector with eigenvalue 0. Otherwise, by definition of right-singular vector, $A^T\vec{u}=\sigma\vec{v}$. Multiplying both sides by $\sigma$ we get $A^T\vec{u}\sigma=\sigma^2\vec{v}$. Since $\vec{u}\sigma=\sigma\vec{u}=A\vec{v}$ by our definition of right-singular vector. Then $A^TA\vec{v}=\sigma^2\vec{v}$. Recall by definition of singular value, $\sigma=\sqrt{\lambda}$, for all eigenvalues $\lambda$ of $A^TA$. Then $A^TA\vec{v}=\lambda\vec{v}$, as required. 
    
    \pagebreak
    \item Note that the singular values of $PA$ are the square roots of the eigenvalues of $(PA)^TPA=A^TP^TPA=A^TA$, since $P$ is an orthogonal matrix ($P^TP=I$). This is exactly the singular values of $A$. 
    
    
\end{enumerate}

\end{document}
