\documentclass[10pt,english]{article}
\usepackage[T1]{fontenc}
\usepackage[latin9]{inputenc}
\usepackage{geometry}
\geometry{verbose,tmargin=1.5in,bmargin=1.5in,lmargin=1.5in,rmargin=1.5in}
\usepackage{amsthm}
\usepackage{amsmath}
\usepackage{amssymb}

\makeatletter
\usepackage{enumitem}
\newlength{\lyxlabelwidth}

\usepackage[T1]{fontenc}
\usepackage{ae,aecompl}

%\usepackage{txfonts}

\usepackage{microtype}

\usepackage{calc}
\usepackage{enumitem}
\setenumerate{leftmargin=!,labelindent=0pt,itemindent=0em,labelwidth=\widthof{\ref{last-item}}}

\makeatother

\usepackage{babel}
\begin{document}
\noindent \begin{center}
\textbf{\large{}MATH 235 - Assignment 5}\\
\textbf{\large{}Chris Ji 20725415}
\par\end{center}{\large \par}
\medskip{}

\begin{enumerate}
\item We need to find all trinomials $p=a+bx+cx^2$ such that $\langle 1+x^2,a+bx+cx^2\rangle=0$. $\langle1+x^2,a+bx+cx^2\rangle=(2)(a-b+c)+(a)+2(a+b+c)=5a+4c=0\Rightarrow a=-\frac{4}{5}c$. Then $p(x)=(-\frac{4}{5}c)+bx+cx^2=bx+c(-\frac{4}{5}+x^2)$. Then a basis for $\mathbb{S}^\perp$ is Span$\left\{x,\frac{-4}{5}+x^2\right\}$.

\pagebreak
\item \begin{enumerate}
    \item By the Gram-Schmidt procedure, we can find an orthogonal basis for $\mathbb{S}$, $\{\vec{v}_1,\vec{v}_2\}$ by setting $\vec{v}_1=\vec{s}_1=\begin{bmatrix}1\\-1\\1\end{bmatrix}$, and then calculating $\vec{v}_2=\vec{s}_2-\frac{\langle\vec{s}_2,\vec{v}_1\rangle}{||\vec{v}_1||^2}\vec{v}_1=\begin{bmatrix}1\\0\\1\end{bmatrix}-\frac{2}{3}\begin{bmatrix}1\\-1\\1\end{bmatrix}=\begin{bmatrix}\frac{1}{3}\\\frac{2}{3}\\\frac{1}{3}\end{bmatrix}$. Note that neither of these are unit vectors, and so we can find $$\hat{v}_1=\frac{1}{||\vec{v}_1||}\vec{v}_1=\frac{1}{\sqrt{3}}\begin{bmatrix}1\\-1\\1\end{bmatrix}=\begin{bmatrix}\frac{1}{\sqrt{3}}\\\frac{-1}{\sqrt{3}}\\\frac{1}{\sqrt{3}}\end{bmatrix}$$ $$\hat{v}_2=\frac{1}{||\vec{v}_2||}\vec{v}_2=\frac{1}{\sqrt{\frac{2}{3}}}\begin{bmatrix}\frac{1}{3}\\\frac{2}{3}\\\frac{1}{3}\end{bmatrix}=\begin{bmatrix}\frac{1}{\sqrt{6}}\\\sqrt{\frac{2}{3}}\\\frac{1}{\sqrt{6}}\end{bmatrix}$$ Then an orthonormal basis for $\mathbb{S}$ is Span$\left\{\begin{bmatrix}\frac{1}{\sqrt{3}}\\\frac{-1}{\sqrt{3}}\\\frac{1}{\sqrt{3}}\end{bmatrix},\begin{bmatrix}\frac{1}{\sqrt{6}}\\\sqrt{\frac{2}{3}}\\\frac{1}{\sqrt{6}}\end{bmatrix}\right\}$
    \item Letting $\vec{v}_1=\begin{bmatrix}\frac{1}{\sqrt{3}}\\\frac{-1}{\sqrt{3}}\\\frac{1}{\sqrt{3}}\end{bmatrix}$ and $\vec{v}_2=\begin{bmatrix}\frac{1}{\sqrt{6}}\\\sqrt{\frac{2}{3}}\\\frac{1}{\sqrt{6}}\end{bmatrix}$, $\text{proj}_\mathbb{S}\vec{w}=\frac{\langle\vec{w},\vec{v}_1\rangle}{||\vec{v}_1||^2}\vec{v}_1+\frac{\langle\vec{w},\vec{v}_2\rangle}{||\vec{v}_2||^2}\vec{v}_2=\frac{\frac{2}{\sqrt{3}}}{1}\begin{bmatrix}\frac{1}{\sqrt{3}}\\\frac{-1}{\sqrt{3}}\\\frac{1}{\sqrt{3}}\end{bmatrix}+\frac{\sqrt{\frac{3}{2}}+\sqrt{\frac{2}{3}}}{1}\begin{bmatrix}\frac{1}{\sqrt{6}}\\\sqrt{\frac{2}{3}}\\\frac{1}{\sqrt{6}}\end{bmatrix}=\begin{bmatrix}\frac{2}{3}\\\frac{-2}{3}\\\frac{2}{3}\end{bmatrix}+\begin{bmatrix}\frac{5}{6}\\\frac{5}{3}\\\frac{5}{6}\end{bmatrix}=\begin{bmatrix}\frac{3}{2}\\1\\\frac{3}{2}\end{bmatrix}$
    \item We need to find all vectors $A=\begin{bmatrix}a_1\\a_2\\a_3\end{bmatrix}$ such that $$0=\langle A,\begin{bmatrix}1\\-1\\1\end{bmatrix}\rangle=a_1-a_2+a_3$$ $$0=\langle A,\begin{bmatrix}1\\0\\1\end{bmatrix}\rangle=a_1+a_3$$ Then clearly $a_1=-s,a_2=0,a_3=s$, and so every vector $A$ can be expressed by $\begin{bmatrix}-s\\0\\s\end{bmatrix}=s\begin{bmatrix}-1\\0\\1\end{bmatrix}$, and hence $\mathbb{S}^\perp=\text{Span}\left\{\begin{bmatrix}-1\\0\\1\end{bmatrix}\right\}$
\end{enumerate}
\pagebreak
\item Let $\mathcal{C}=\{\vec{v}_1,\vec{v}_2,\vec{v}_3\}$, and $\vec{w}_1=2,\vec{w}_2=x,\vec{w}_3=x^2$ Then by the Gram-Schmidt procedure, $\vec{v}_1=\vec{w}_1=2$, $$\vec{v}_2=\vec{w}_2-\frac{\langle\vec{w}_2,\vec{v}_1\rangle}{||\vec{v}_1||^2}\vec{v}_1=x-\frac{\langle x,2\rangle}{||2||^2}2=x-2\frac{2\langle 1,x\rangle}{2*2*\langle 1,1\rangle}=x-1$$ \begin{multline*}\vec{v}_3=\vec{w}_3-\frac{\langle\vec{w}_3,\vec{v}_1\rangle}{||\vec{v}_1||^2}\vec{v}_1-\frac{\langle\vec{w}_3,\vec{v}_2\rangle}{||\vec{v}_2||^2}\vec{v}_2=x^2-\frac{\langle x^2,2\rangle}{||2||^2}2-\frac{\langle x^2,x-1\rangle}{||x-1||^2}(x-1)=\\x^2-\frac{2\langle x^2,1\rangle}{2*2*\langle 1,1\rangle}2-\frac{\langle x^2,x\rangle-\langle x^2,1\rangle}{\langle x,x\rangle-\langle x,1\rangle-\langle1,x\rangle+\langle1,1\rangle}(x-1)=x^2-\frac{2*(-2)*2}{2*2*2}+\frac{(-2-(-2))(x-1)}{(4-2-2+2)}=x^2+1\end{multline*} Then an orthogonal basis $\text{Span}\{2,x-1,x^2+1\}$, and so the orthonormal basis we want is $$\hat{v}_1=\frac{1}{||\vec{v}_1||}\vec{v}_1=\frac{1}{\sqrt{2*2*2}}2=\frac{1}{\sqrt{2}}$$ $$\hat{v}_2=\frac{1}{||\vec{v}_2||}\vec{v}_2=\frac{1}{\sqrt{4-2-2+2}}(x-1)=\frac{x-1}{\sqrt{2}}$$ $$\hat{v}_3=\frac{1}{||\vec{v}_3||}\vec{v}_3=\frac{1}{\sqrt{\langle x^2,x^2\rangle+\langle 1,x^2\rangle+\langle x^2,1\rangle+\langle 1,1\rangle}}(x^2+1)=\frac{x^2+1}{1}=x^2+1$$ Then the orthonormal basis $\mathcal{C}$ is $\left\{\frac{1}{\sqrt{2}},\frac{x-1}{\sqrt{2}},x^2+1\right\}$

\pagebreak
\item Note that $\left\{\begin{bmatrix}1\\1\\1\end{bmatrix},\begin{bmatrix}0\\1\\0\end{bmatrix},\begin{bmatrix}0\\0\\1\end{bmatrix}\right\}$ is a basis for $\mathbb{R}^3$, so we can apply the Gram-Schmidt procedure on this basis to find an orthogonal basis for $\mathbb{R}^3$. Let $\vec{v}_1=\begin{bmatrix}1\\1\\1\end{bmatrix}$ $$\vec{v}_2=\begin{bmatrix}0\\1\\0\end{bmatrix}-\frac{\left\langle\begin{bmatrix}0\\1\\0\end{bmatrix},\vec{v}_1\right\rangle}{||\vec{v}_1||^2}\vec{v}_1=\begin{bmatrix}0\\1\\0\end{bmatrix}-\frac{2}{5}\begin{bmatrix}1\\1\\1\end{bmatrix}=\begin{bmatrix}-\frac{2}{5}\\\frac{3}{5}\\-\frac{2}{5}\end{bmatrix}$$ $$\vec{v}_3=\begin{bmatrix}0\\0\\1\end{bmatrix}-\frac{\left\langle\begin{bmatrix}0\\0\\1\end{bmatrix},\vec{v}_1\right\rangle}{||\vec{v}_1||^2}\vec{v}_1-\frac{\left\langle\begin{bmatrix}0\\0\\1\end{bmatrix},\vec{v}_2\right\rangle}{||\vec{v}_2||^2}\vec{v}_2=\begin{bmatrix}0\\0\\1\end{bmatrix}-\frac{1}{5}\begin{bmatrix}1\\1\\1\end{bmatrix}-\frac{\frac{-2}{5}}{\frac{6}{5}}\begin{bmatrix}-\frac{2}{5}\\\frac{3}{5}\\-\frac{2}{5}\end{bmatrix}=\begin{bmatrix}-\frac{1}{3}\\0\\\frac{2}{3}\end{bmatrix}$$ Then an orthogonal basis for $\mathbb{R}^3$ under this inner product, extended from $\left\{\begin{bmatrix}1\\1\\1\end{bmatrix}\right\}$ is $\text{Span}\left\{\begin{bmatrix}1\\1\\1\end{bmatrix},\begin{bmatrix}-\frac{2}{5}\\\frac{3}{5}\\-\frac{2}{5}\end{bmatrix},\begin{bmatrix}-\frac{1}{3}\\0\\\frac{2}{3}\end{bmatrix}\right\}$

\pagebreak
\item Note that $T$ looks like this: $\begin{bmatrix}\langle\vec{w}_1,\vec{v}_1\rangle&\cdots&\langle\vec{w}_m\vec{v}_1\rangle\\\vdots&\ddots&\vdots\\\langle\vec{w}_1,\vec{v}_m\rangle&\cdots&\langle\vec{w}_m,\vec{v}_m\rangle\end{bmatrix}$, but since $\langle\vec{w}_i,\vec{v}_j\rangle=0$ for $j>i$ by the definition of the Gram-Schmidt procedure, all elements below the diagonal are equal to 0. Then, $T$ is upper triangular, and we can calculate $\text{det }T=\langle\vec{w}_1,\vec{v}_1\rangle+\ldots+\langle\vec{w}_m,\vec{v}_m\rangle$. By the Gram-Schmidt procedure, since each $\vec{v}_i=\vec{w}_i-\sum_{j=1}^{i-1}\frac{\langle\vec{w}_i,\vec{v}_j\rangle}{||\vec{v}_j||^2}\vec{v}_j\Rightarrow \vec{w}_i=\vec{v}_i+\sum_{j=1}^{i-1}\frac{\langle\vec{w}_i,\vec{v}_j\rangle}{||\vec{v}_j||^2}\vec{v}_j$. Then $\langle\vec{w}_i,\vec{v}_i\rangle=\langle\vec{v}_i+\sum_{j=1}^{i-1}\frac{\langle\vec{w}_i,\vec{v}_j\rangle}{||\vec{v}_j||^2}\vec{v}_j,\vec{v}_i\rangle=\langle\vec{v}_i,\vec{v}_i\rangle+\sum_{j=1}^{i-1}\frac{\langle\vec{w}_i,\vec{v}_j\rangle}{||\vec{v}_j||^2}\langle\vec{v}_i,\vec{v}_j\rangle$, but since $\langle\vec{v}_i,\vec{v}_j\rangle=0$ by the Gram-Schmidt procedure, $\langle\vec{w}_i,\vec{w}_i\rangle=\langle\vec{v}_i,\vec{v}_i\rangle$ for all $1\leq i\leq m$. Then $\text{det }T=\sum_{i=1}^{m}\langle\vec{v}_i,\vec{v}_i\rangle=\sum_{i=1}^m||\vec{v}_i||^2$, which is greater than 0, and so by Theorem 5.3.8, $T$ is invertible. 



\end{enumerate}

\end{document}
