\documentclass[10pt,english]{article}
\usepackage[T1]{fontenc}
\usepackage[latin9]{inputenc}
\usepackage{geometry}
\geometry{verbose,tmargin=1.5in,bmargin=1.5in,lmargin=1.5in,rmargin=1.5in}
\usepackage{amsthm}
\usepackage{amsmath}
\usepackage{amssymb}

\makeatletter
\usepackage{enumitem}
\newlength{\lyxlabelwidth}

\usepackage[T1]{fontenc}
\usepackage{ae,aecompl}

%\usepackage{txfonts}

\usepackage{microtype}

\usepackage{calc}
\usepackage{enumitem}
\setenumerate{leftmargin=!,labelindent=0pt,itemindent=0em,labelwidth=\widthof{\ref{last-item}}}

\makeatother

\usepackage{babel}
\begin{document}
\noindent \begin{center}
\textbf{\large{}MATH 235 - Assignment 7}\\
\textbf{\large{}Chris Ji 20725415}
\par\end{center}{\large \par}
\medskip{}

\begin{enumerate}
\item \begin{enumerate}
    \item $C(\lambda)=\begin{vmatrix}0-\lambda&1&-1\\1&0-\lambda&1\\-1&1&0-\lambda\end{vmatrix}=-(\lambda-1)^2(\lambda+2)$. Thus the eigenvalues are $\lambda_1=1$, and $\lambda_2=-2$, with $a_{\lambda_1}=2,a_{\lambda_2}=1$. \\ 
    For $\lambda_1$, $A-\lambda_1I=\begin{bmatrix}-1&1&-1\\1&-1&1\\-1&1&-1\end{bmatrix}\sim\begin{bmatrix}1&-1&1\\0&0&0\\0&0&0\end{bmatrix}$. Thus a basis for the eigenspace of $\lambda_1$ is $\left\{\begin{bmatrix}1\\1\\0\end{bmatrix},\begin{bmatrix}-1\\0\\1\end{bmatrix}\right\}$. Since these are not orthogonal to each other, by the Gram-Schmidt procedure we can find $\vec{v}_2=\vec{w}_2-\frac{\vec{w}_2\cdot\vec{v}_1}{||\vec{v}_1||^2}\vec{v}_1=\begin{bmatrix}-\frac{1}{2}\\\frac{1}{2}\\1\end{bmatrix}$. \\ 
    For $\lambda_2$, $A-\lambda_2I=\begin{bmatrix}2&1&-1\\1&2&1\\-1&1&2\end{bmatrix}\sim\begin{bmatrix}1&0&-1\\0&1&1\\0&0&0\end{bmatrix}$. Thus a basis for the eigenspace of $\lambda_2$ is $\left\{\begin{bmatrix}1\\-1\\1\end{bmatrix}\right\}$. \\ 
    Then, normalizing the vectors we get $P=\begin{bmatrix}\frac{1}{\sqrt{2}}&-\frac{1}{\sqrt{6}}&\frac{1}{\sqrt{3}}\\\frac{1}{\sqrt{2}}&\frac{1}{\sqrt{6}}&-\frac{1}{\sqrt{3}}\\0&\sqrt{\frac{2}{3}}&\frac{1}{\sqrt{3}}\end{bmatrix}$, and $P^TAP=D=\begin{bmatrix}1&0&0\\0&1&0\\0&0&-2\end{bmatrix}$
    
    \pagebreak
    \item $C(\lambda)=\begin{vmatrix}2-\lambda&-2&-4\\-2&5-\lambda&-2\\-4&-2&2-\lambda\end{vmatrix}=-(\lambda-6)^2(\lambda+3)$. Thus the eigenvalues are $\lambda_1=6$, and $\lambda_2=-3$, with $a_{\lambda_1}=2$, and $a_{\lambda_2}=1$. \\
    For $\lambda_1$, $A-\lambda_1I=\begin{bmatrix}-4&-2&-4\\-2&-1&-2\\-4&-2&-4\end{bmatrix}\sim\begin{bmatrix}1&\frac{1}{2}&1\\0&0&0\\0&0&0\end{bmatrix}$, and thus a basis for the eigenspace of $\lambda_1$ is $\left\{\begin{bmatrix}-\frac{1}{2}\\1\\0\end{bmatrix},\begin{bmatrix}-1\\0\\1\end{bmatrix}\right\}$. By the Gram-Schmidt procedure, we can find $\vec{v}_2=\vec{w}_2-\frac{\vec{w}_2\cdot\vec{v}_1}{||\vec{v}_1||^2}\vec{v}_1=\begin{bmatrix}-\frac{4}{5}\\-\frac{2}{5}\\1\end{bmatrix}$. \\ 
    For $\lambda_2$, $A-\lambda_2I=\begin{bmatrix}5&-2&-4\\-2&8&-2\\-4&-2&5\end{bmatrix}\sim\begin{bmatrix}1&0&-1\\0&1&-\frac{1}{2}\\0&0&0\end{bmatrix}$. Thus a basis for the eigenspace of $\lambda_2$ is $\left\{\begin{bmatrix}1\\\frac{1}{2}\\1\end{bmatrix}\right\}$. \\ 
    Then, normalizing the vectors we get $P=\begin{bmatrix}-\frac{1}{\sqrt{5}}&-\frac{4}{3\sqrt{5}}&\frac{2}{3}\\\frac{2}{\sqrt{5}}&-\frac{2}{3\sqrt{5}}&\frac{1}{3}\\0&\frac{\sqrt{5}}{3}&\frac{2}{3}\end{bmatrix}$, and $P^TAP=D=\begin{bmatrix}6&0&0\\0&6&0\\0&0&-3\end{bmatrix}$
    
    \pagebreak
    \item $C(\lambda)=\begin{vmatrix}3-\lambda&0&7\\0&5-\lambda&0\\7&0&3-\lambda\end{vmatrix}=-(\lambda+4)(\lambda-5)(\lambda-10)$. Thus the eigenvalues are $\lambda_1=-4,\lambda_2=5,\lambda_3=10$, each with algebraic multiplicity 1. \\ 
    For $\lambda_1$, $A-\lambda_1I=\begin{bmatrix}7&0&7\\0&9&0\\7&0&7\end{bmatrix}\sim\begin{bmatrix}1&0&1\\0&1&0\\0&0&0\end{bmatrix}$, so the basis for the eigenspace of $\lambda_2$ is $\left\{\begin{bmatrix}-1\\0\\1\end{bmatrix}\right\}$. \\ 
    For $\lambda_2$, $A-\lambda_2I=\begin{bmatrix}-2&0&7\\0&0&0\\7&0&-2\end{bmatrix}\sim\begin{bmatrix}1&0&0\\0&0&1\\0&0&0\end{bmatrix}$, so the basis for the eigenspace of $\lambda_2$ is $\left\{\begin{bmatrix}0\\1\\0\end{bmatrix}\right\}$. \\ 
    For $\lambda_3$, $A-\lambda_3I=\begin{bmatrix}-7&0&7\\0&-5&0\\7&0&-7\end{bmatrix}\sim\begin{bmatrix}1&0&-1\\0&1&0\\0&0&0\end{bmatrix}$, so the basis for the eigenspace of $\lambda_3$ is $\left\{\begin{bmatrix}1\\0\\1\end{bmatrix}\right\}$. \\ 
    Then, normalizing the vectors we get $P=\begin{bmatrix}-\frac{1}{\sqrt{2}}&0&\frac{1}{\sqrt{2}}\\0&1&0\\\frac{1}{\sqrt{2}}&0&\frac{1}{\sqrt{2}}\end{bmatrix}$, and $P^TAP=D=\begin{bmatrix}-4&0&0\\0&5&0\\0&0&10\end{bmatrix}$
    
\end{enumerate}

\pagebreak
\item $C(\lambda)=\begin{vmatrix}2-\lambda&1&0\\-1&4-\lambda&0\\1&1&3-\lambda\end{vmatrix}=-(\lambda-3)^3$. So then the only eigenvalue is $\lambda=3$, and the corresponding eigenvector is $\vec{v}_1=\begin{bmatrix}0\\0\\1\end{bmatrix}$. Clearly, we can extend this to $\left\{\begin{bmatrix}0\\0\\1\end{bmatrix},\begin{bmatrix}0\\1\\0\end{bmatrix},\begin{bmatrix}1\\0\\0\end{bmatrix}\right\}$, which are the columns for $P_1$. Hence $P_1^TAP_1=\begin{bmatrix}3&\vec{b}^T\\\vec{0}&A_1\end{bmatrix}$, where $\vec{b}=\begin{bmatrix}1\\1\end{bmatrix}$ and $A_1=\begin{bmatrix}4&-1\\1&2\end{bmatrix}$. Then for $A_1$, we see that $C(\lambda)=(\lambda-3)^2$. For $\lambda=3$ we get $A_1-3I=\begin{bmatrix}0&-1\\1&-2\end{bmatrix}\Rightarrow \vec{v}_1=\begin{bmatrix}\frac{1}{\sqrt{2}}\\\frac{1}{\sqrt{2}}\end{bmatrix}$. We can extend this to an orthonormal basis for $\mathbb{R}^2$ by taking $\vec{v}_2=\begin{bmatrix}\frac{1}{\sqrt{2}}\\-\frac{1}{\sqrt{2}}\end{bmatrix}$, hence $Q=\begin{bmatrix}\frac{1}{\sqrt{2}}&\frac{1}{\sqrt{2}}\\\frac{1}{\sqrt{2}}&-\frac{1}{\sqrt{2}}\end{bmatrix}$, then we take $P_2=\begin{bmatrix}1&0&0\\0&\frac{1}{\sqrt{2}}&\frac{1}{\sqrt{2}}\\0&\frac{1}{\sqrt{2}}&-\frac{1}{\sqrt{2}}\end{bmatrix}$, then $P=P_1P_2=\begin{bmatrix}0&\frac{\sqrt{2}}{2}&-\frac{\sqrt{2}}{2}\\0&\frac{\sqrt{2}}{2}&\frac{\sqrt{2}}{2}\\1&0&0\end{bmatrix}$, so we get that $P^TAP=T=\begin{bmatrix}3&\sqrt{2}&0\\0&3&2\\0&0&3\end{bmatrix}$.


\pagebreak
\item If $A$ is orthogonally diagonalizable, then it is orthogonally similar to a diagonal matrix $B$, such that there exists an orthogonal matrix $P$ such that $P^TAP=B$. Since $A$ is invertible, and $A$ and $B$ are similar, then $\text{rank }A=\text{rank }B$, and so $B$ is invertible by the invertible matrix theorem.. Furthermore, $P^{-1}=P^T$, so \begin{alignat*}{2}&\quad\,\,\,P^TAP&&=B\\&\Rightarrow (P^TAP)^{-1}&&=B^{-1}\\&\Rightarrow PA^{-1}P^T&&=B{-1}\end{alignat*} Hence $A^{-1}$ is orthogonally diagonalizable by orthogonal matrix $P^T$, and it is orthogonally similar to a diagonal matrix $B^{-1}$. 


\pagebreak
\item By triangularization theorem, $A$ is orthogonally similar to a upper triangle matrix $T$. Then there exists an orthogonal matrix $P$ such that $P^TAP=T$. Note that taking the transpose of both sides gives us $P^TA^TP=T^T$, so $A^T$ is orthogonally similar to a lower triangular matrix, $T^T$. $A$ clearly has the same characteristic polynomial as $A^T$, so $A$ and $A^T$ are orthogonally similar, say by some orthogonal matrix $Q$. Then $P^TQ^TAQP=T^T$. By Theorem 9.2.7, since $P$ and $Q$ are both orthogonal matrices, so are $P^TQ^T$ and $QP$, so and so $A$ is orthogonally similar to a lower triangular matrix, $T^T$, by orthogonal matrix $QP$. 


\pagebreak
\item \begin{enumerate}
    \item Let $A=\begin{bmatrix}1&3\\3&2\end{bmatrix},B=\begin{bmatrix}1&0\\0&0\end{bmatrix}$, then $AB=\begin{bmatrix}1&0\\3&0\end{bmatrix}$, which is not symmetric, even though $A$ and $B$ are, then by the principal axis theorem and theorem 10.2.1, $AB$ is not orthogonally diagonalizable, but $A$ and $B$ are. 
    \item Let $A=\begin{bmatrix}1&3\\3&2\end{bmatrix},B=\begin{bmatrix}1&0\\0&0\end{bmatrix}$, then $AB=\begin{bmatrix}1&0\\3&0\end{bmatrix}$, which is not symmetric, even though $A$ and $B$ are. 
    \item If $A$ is orthogonally similar to a symmetric matrix $B$, then there exists an orthogonal matrix $P$ such that $P^TAP=B\Rightarrow (P^TAP)^T=B^T\Rightarrow P^TA^TP=B^T=B$, then $P^TAP=B=P^TA^TP\Rightarrow A=A^T$, and hence $A$ is symmetric and therefore orthogonally diagonalizable.
\end{enumerate}
\end{enumerate}

\end{document}
