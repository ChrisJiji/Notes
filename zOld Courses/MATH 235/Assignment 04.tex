\documentclass[10pt,english]{article}
\usepackage[T1]{fontenc}
\usepackage[latin9]{inputenc}
\usepackage{geometry}
\geometry{verbose,tmargin=1.5in,bmargin=1.5in,lmargin=1.5in,rmargin=1.5in}
\usepackage{amsthm}
\usepackage{amsmath}
\usepackage{amssymb}

\makeatletter
\usepackage{enumitem}
\newlength{\lyxlabelwidth}

\usepackage[T1]{fontenc}
\usepackage{ae,aecompl}

%\usepackage{txfonts}

\usepackage{microtype}

\usepackage{calc}
\usepackage{enumitem}
\setenumerate{leftmargin=!,labelindent=0pt,itemindent=0em,labelwidth=\widthof{\ref{last-item}}}

\makeatother

\usepackage{babel}
\begin{document}
\noindent \begin{center}
\textbf{\large{}MATH 235- Assignment 4}\\
\textbf{\large{}Chris Ji 20725415}
\par\end{center}{\large \par}
\medskip{}

\begin{enumerate}
\item \begin{enumerate}
    \item  Let $\vec{a}=\begin{bmatrix}a_1\\a_2\end{bmatrix},\vec{b}=\begin{bmatrix}b_1\\b_2\end{bmatrix},\vec{c}=\begin{bmatrix}c_1\\c_2\end{bmatrix}$. \\
    $\langle\vec{a},\vec{a}\rangle=2a_1a_1-2a_1a_2-2a_2a_1+4a_2a_2=2a_1^2-4a_1a_2+4a_2^2=2((a_1-a_2)^2+a_2^2)$, which is obviously greater than or equal to $0$ for all $a_1,a_2$. Moreover, the only way this equation equals zero is if $a_1=a_2=0$, and so $\vec{a}=\vec{0}$. Therefore, $\langle\vec{x},\vec{y}\rangle$ is positive definite. \\ 
    $\langle\vec{a},\vec{b}\rangle=2a_1b_1-2a_1b_2-2a_2b_1+4a_2b_2=2b_1a_1-2b_2a_1-2b_1a_2+4b_2a_2=\langle\vec{b},\vec{a}\rangle$, so $\langle\vec{x},\vec{y}\rangle$ is symmetric. \\ 
    $\langle s\vec{a}+t\vec{b},\vec{c}\rangle=\langle\begin{bmatrix}sa_1+tb_1\\sa_2+tb_2\end{bmatrix},\begin{bmatrix}c_1\\c_2\end{bmatrix}\rangle=2(sa_1+tb_1)c_1-2(sa_1+tb_1)c_2-2(sa_2+tb_2)c_1+4(sa_2+tb_2)c_2=s(2a_1c_1-2a_1c_2-2a_2c_1+4a_2c_2)+t(2b_1c_1-2b_1c_2-2b_2c_1+4b_2c_2)=s\langle\vec{a},\vec{c}\rangle+t\langle\vec{b},\vec{c}\rangle$. Therefore $\langle\vec{x},\vec{y}\rangle$ is left linear (and bilinear), and so it is an inner product on $\mathbb{R}^2$. 
    
    \item On $M_{2\times 2}(\mathbb{R})$, $\langle A,B\rangle=\text{det}(AB)$ Let $A=\begin{bmatrix}a_1&a_2\\a_3&a_4\end{bmatrix},B=\begin{bmatrix}b_1&
    b_2\\b_3&b_4\end{bmatrix},C=\begin{bmatrix}c_1&c_2\\c_3&c_4\end{bmatrix}$. \\ 
    $\langle A,A\rangle=\text{det}\left(\begin{bmatrix}a_1a_1+a_2a_3&a_1a_2+a_2a_4\\a_3a_1+a_4a_3&a_3a_2+a_4a_4\end{bmatrix}\right)=(a_1^2+a_2a_3)(a_3a_2+a_4^2)-(a_1a_2+a_2a_4)(a_3a_1+a_4a_3)=a_2^2a_3^2-2a_1a_2a_3a_4+a_1^2a_4^2=(a_2a_3-a_1a_4)^2$. The square of any number is always non-negative, and the only way this equation equals $0$ is if $a_1=a_2=a_3=a_4=0$ (easily shown on the second last step by the term $2a_1a_2a_3a_4$), so $\langle A,B\rangle$ is positive definite. \\ 
    Note that $AB=\begin{bmatrix}a_1b_1+a_2b_3&a_1b_2+a_2b_4\\a_3b_1+a_4b_3&a_3b_2+a_4b_4\end{bmatrix}$, so $\langle A,B\rangle=\text{det}(AB)=(a_1b_1+a_2b_3)(a_3b_2+a_4b_4)-(a_1b_2+a_2b_4)(a_3b_1+a_4b_3)=a_1a_4b_1b_4+a_2a_3b_2b_3-a_1a_4b_2b_3-a_2a_3b_1b_4=b_1b_4a_1a_4+b_2b_3a_2a_3-b_1b_4a_2a_3-b_2b_3a_1a_4=\text{det}(AB)=\langle A, B\rangle$. So $\langle A,B\rangle$ is symmetric. \\ 
    $\langle sA+tB,C\rangle=\left\langle\begin{bmatrix}sa_1+tb_1&sa_2+tb_2\\sa_3+tb_3&sa_4+tb_4\end{bmatrix},\begin{bmatrix}c_1&c_2\\c_3&c_4\end{bmatrix}\right\rangle=(c_1(sa_1+tb_1)+c_3(sa_2+tb_2))(c_2(sa_3+tb_3)+c_4(sa_4+tb_4))-(c_2(sa_1+tb_1)+c_4(sa_2+tb_2))(c_1(sa_3+tb_3)+c_3(sa_4+tb_4))=a_1a_4c_1c_4s^2+a_2a_3c_2c_3s^2-a_1a_4c_2c_3s^2-a_2a_3c_1c_4s^2+a_1c_1c_4stb_4+a_4c_1c_4stb_1+a_2c_2c_3stb_3+a_3c_2c_3stb_2-a_1c_2c_3stb_4-a_4c_2c_3stb_1-a_2c_1c_4stb_3-a_3c_1c_4stb_2+c_1c_4t^2b_1b_4-c_1c_4t^2b_2b_3+c_2c_3t^2b_2b_3-c_2c_3t^2b_1b_4$. Note that this sum has squares of $s$ and $t$, which cannot exist in $s\langle A,C\rangle+t\langle B,C\rangle$, so therefore $\langle A,B\rangle$ is not left linear, and not an inner product on $M_{2\times2}(\mathbb{R})$. 
    
    \item On $P_2(\mathbb{R}),\langle p(x),q(x)\rangle=p(-1)q(1)+p(0)q(0)+p(1)q(-1)$. Let $p(x)=p_0+p_1x+p_2x^2,q(x)=q_0+q_1x+q_2x^2,r(x)=r_0+r_1x+r_2x^2$. \\ 
    $\langle p(x),p(x)\rangle=p(-1)p(1)^2+p(0)^2=(p_0-p_1+p_2)^2+p_0^2$. Note that this is always greater than zero, with it only being equal to zero when $p_0=p_1=p_2=0$. Thus, $\langle p(x),q(x)\rangle$ is positive definite. \\ 
    $\langle p(x),q(x)\rangle=p(−1)q(1)+p(0)q(0)+p(1)q(−1)=q(-1)p(1)+q(0)p(0)+q(1)p(-1)=\langle q(x),p(x)\rangle$, so it is symmetric. \\ 
    Note that $ap(x)+bq(x)=a(p_0+p_1x+p_2x^2)+b(q_0+q_1x+q_2x^2)=ap_0+bq_0+(ap_1+bq_1)x+(ap_2+bq_2)x^2$. Then $\langle ap(x)+bq(x),r(x)\rangle=(ap_0+bq_0-(ap_1+bq_1)+(ap_2+bq_2))(c_0+c_1+c_2)+(ap_0+bq_0)c_0+(ap_0+bq_0+(ap_1+bq_1)+(ap_2+bq_2))(c_0-c_1+c_2)=a(3c_0p_0+2c_2p_0-2c_1p_1+2c_0ap_2+2c_2p_2)+b(3c_0q_0+2c_0q_2-c_1q_1+2c_2q_0+2c_2q_2)=a\langle p(x),c(x)\rangle+b\langle q(x),c(x)\rangle$. Then $\langle p(x),q(x)\rangle$ is left linear, and so it is an inner product space. 
\end{enumerate}

\pagebreak
\item \begin{enumerate}
    \item $\langle1+x-2x^2,x+x^2\rangle=(1-1-2(-1)^2)(-1+(-1)^2)+1(0)+(1+1-2(1)^2)(1+(1)^2)=0+0+0=0$. 
    \item Note that $\text{dim}P_2(\mathbb{R})=3$, and $\{1+x-2x^2,x+x^2\}$ is linearly independent. Then by theorem 9.2.3, there's only one more vector in $P_2(\mathbb{R})$ such that they will all be orthogonal. Let the third vector be $f(x)=f_0+f_1x+f_2x^2$. Then $p(-1)f(-1)+p(0)f(0)+p(1)f(1)=q(-1)f(-1)+q(0)f(0)+q(1)f(1)=0$. Setting $p(x)=1+x-2x^2$, and $q(x)=x+x^2$ \\ 
    $p(-1)f(-1)+p(0)f(0)+p(1)f(1)=(1-1-2)(f_0-f_1+f_2)+(1)(f_0)+(1+1-2)(f_0+f_1+f_2)=-2(f_0-f_1+f_2)+f_0$ \\ 
    $q(-1)f(-1)+q(0)f(0)+q(1)f(1)=(-1+1)(f_0-f_1+f_2)+(0)(f_0)+(1+1)(f_0+f_1+f_2)=2(f_0+f_1+f_2)=-2(f_0-f_1+f_2)+f_0\Rightarrow f_0=f_2=0$. Then the third vector can be represented by the set of all $f_1x$, so the set of all vectors in $P_2(\mathbb{R})$ which are orthogonal to $x+x^2$ is spanned by $\{1+x-2x^2,x\}$
    \item $||1+x+x^2||=\sqrt{\langle 1+x+x^2,1+x+x^2\rangle}=\sqrt{(1-1+1)(1-1+1)+(1)(1)+(1+1+1)(1+1+1)}=\sqrt{1+1+9}=\sqrt{11}$
    \item $1+x^2=\frac{\langle1+x^2,1-x-2x^2\rangle}{||1-x-2x^2||^2}(1-x-2x^2)+\frac{\langle1+x^2,2-x^2\rangle}{||2-x^2||^2}(2-x^2)+\frac{\langle1+x^2,2+3x-4x^2\rangle}{||2+3x-4x^2||^2}(2+3x-4x^2)=\frac{-3}{5}(1-x-2x^2)+\frac{6}{6}(2-x^2)+\frac{-6}{30}(2+3x-4x^2)=\frac{-3}{5}(1-x-2x^2)+(2-x^2)+\frac{-1}{5}(2+3x-4x^2)$
\end{enumerate}

\pagebreak
\item Let $\mathcal{B}=\{\vec{v}_1,\ldots,\vec{v}_n\}$, and $\vec{x}=x_1\vec{v}_1+\ldots+x_n\vec{v}_n$. Assume that $\mathcal{B}$ is not an orthonormal basis. Then $\vec{x}\neq\langle\vec{x},\vec{v}_1\rangle\vec{v}_1+\ldots+\langle\vec{x},\vec{v}_n\rangle\vec{v}_n$. Note that $[\vec{x}]_\mathcal{B}=\begin{bmatrix}x_1\\\vdots\\x_n\end{bmatrix}$, and $[\vec{v}_i]_\mathcal{B}$ has a $1$ in the $i^{th}$ dimension, and $0$ elsewhere. Then $\sum_{i=0}^n\langle\vec{x},\vec{v}_i\rangle\vec{v}_i=\sum_{i=0}^nx_i\vec{v}_i$, as $[\vec{x}]_\mathcal{B}\cdot[\vec{v}_i]_\mathcal{B}$ will have $x_jv_j=0$ for all $j\neq i$. However, note that $\langle\vec{x},\vec{v}_1\rangle\vec{v}_1+\ldots+\langle\vec{x},\vec{v}_n\rangle\vec{v}_n=\sum_{i=0}^nx_i\vec{v}_i=x_1\vec{v}_1+\ldots+x_n\vec{v}_n=\vec{x}$, so therefore $\mathcal{B}$ is orthonormal. 

\pagebreak
\item Let $a=\begin{bmatrix}a_1\\a_2\end{bmatrix},b=\begin{bmatrix}b_1\\b_2\end{bmatrix}$. Define $L:\mathbb{R}^2\rightarrow\mathbb{R}^2$ such that $L\left(\begin{bmatrix}v_1\\v_2\end{bmatrix}\right)=\begin{bmatrix}v_1+v_2\\v_2\end{bmatrix}$. $L(s\vec{a}+t\vec{b})=\begin{bmatrix}sa_1+tb_1+sa_2+tb_2\\sa_2+tb_2\end{bmatrix}=s\begin{bmatrix}a_1\\a_2\end{bmatrix}+t\begin{bmatrix}b_1\\b_2\end{bmatrix}=sL(\vec{a})+tL(\vec{b})$, so $L$ is linear. Note that a basis for $\text{Range}(L)$ is $\left\{\begin{bmatrix}1\\0\end{bmatrix},\begin{bmatrix}1\\1\end{bmatrix}\right\}$, so by the rank-nullity theorem, $\text{Ker}(L)=\{\vec{0}\}$, and so by lemma 8.4.1 and 8.4.3, $L$ is an isomorphism. \\ 
Note that $\left\{\begin{bmatrix}1\\0\end{bmatrix},\begin{bmatrix}0\\1\end{bmatrix}\right\}$ is an orthonormal basis of $\mathbb{R}^2$, but $\left\{L\left(\begin{bmatrix}1\\0\end{bmatrix}\right),L\left(\begin{bmatrix}0\\1\end{bmatrix}\right)\right\}=\left\{\begin{bmatrix}1\\0\end{bmatrix},\begin{bmatrix}1\\1\end{bmatrix}\right\}$ is clearly not an orthonormal basis of $\mathbb{R}^2$. 

\pagebreak
\item Let $\vec{v}_1=\begin{bmatrix}1/2\\1/2\\1/2\end{bmatrix},\vec{v}_2=\begin{bmatrix}-1/2\\1/2\\-1/2\end{bmatrix},\vec{v}_3=\begin{bmatrix}1/\sqrt{2}\\0\\1/\sqrt{2}\end{bmatrix}$. Then for these three vectors to form an orthonormal basis for $\mathbb{R}^3$ under $\langle,\rangle$, $\langle\vec{v}_i,\vec{v}_j\rangle$ must equal $0$ for all $i\neq j$. Furthermore, $||\vec{v}||$ must equal $1$ for all $\vec{v}$. \\ 
$$\langle\vec{v}_1,\vec{v}_2\rangle=\frac{1}{2}\frac{-1}{2}+2\frac{1}{2}\frac{1}{2}+\frac{1}{2}\frac{-1}{2}=\frac{-1}{4}+\frac{1}{2}+\frac{-1}{4}=0$$
$$\langle\vec{v}_1,\vec{v}_3\rangle=\frac{1}{2}\frac{1}{\sqrt{2}}+2\frac{1}{2}0+\frac{1}{2}\frac{-1}{\sqrt{2}}=\frac{1}{2\sqrt{2}}+0+\frac{-1}{2\sqrt{2}}=0$$
$$\langle\vec{v}_2,\vec{v}_3\rangle=\frac{-1}{2}\frac{1}{\sqrt{2}}+2\frac{1}{2}0+\frac{-1}{2}\frac{-1}{\sqrt{2}}=\frac{-1}{2\sqrt{2}}+0+\frac{1}{2\sqrt{2}}=0$$
$$||\vec{v}_1||=\sqrt{\frac{1}{2}\frac{1}{2}+2\frac{1}{2}\frac{1}{2}+\frac{1}{2}\frac{1}{2}}=\sqrt{1}=1$$
$$||\vec{v}_2||=\sqrt{\frac{-1}{2}\frac{-1}{2}+2\frac{1}{2}\frac{1}{2}+\frac{-1}{2}\frac{-1}{2}}=\sqrt{1}=1$$
$$||\vec{v}_3||=\sqrt{\frac{1}{\sqrt{2}}\frac{1}{\sqrt{2}}+2\cdot0\cdot0+\frac{-1}{\sqrt{2}}\frac{-1}{\sqrt{2}}}=\sqrt{1}=1$$
Therefore, $\left\{\begin{bmatrix}1/2\\1/2\\1/2\end{bmatrix},\begin{bmatrix}-1/2\\1/2\\-1/2\end{bmatrix},\begin{bmatrix}1/\sqrt{2}\\0\\1/\sqrt{2}\end{bmatrix}\right\}$ is an orthonormal basis for $\mathbb{R}^3$ under $\langle,\rangle$. 


\end{enumerate}

\end{document}
