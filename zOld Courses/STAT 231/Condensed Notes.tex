\documentclass[10pt,letter]{article}
\usepackage{amsmath}
\usepackage{amssymb}
\usepackage{amsthm}
\usepackage{graphicx}
\usepackage{setspace}
\onehalfspacing
\usepackage{fullpage}
\newtheorem*{remark}{Remark}
\theoremstyle{plain}
\newtheorem*{theorem*}{Theorem}
\newtheorem{theorem}{Theorem}[section]
\newtheorem{corollary}{Corollary}[theorem]
\newtheorem*{lemma*}{Lemma}
\newtheorem{lemma}[theorem]{Lemma}
\theoremstyle{definition}
\newtheorem{definition}{Definition}[section]
\newtheorem*{definition*}{Definition}
\newcommand{\Mod}[1]{\ (\mathrm{mod}\ #1)}

\begin{document}

\paragraph{Types of Variates}
\begin{itemize}
    \item Continuous
    \item Discrete
    \item Categorical
    \item Ordinal
    \item Complex
\end{itemize}

\paragraph{Types of Studies}
\begin{itemize}
    \item Sample Survey - select a representative sample of units
    \item Observational Study - dont change any variables 
    \item Experimental Study - change variables
\end{itemize}

\paragraph{PPDAC terms}
\begin{itemize}
    \item Types of Problems     
    \begin{itemize}
        \item Descriptive - Determine an attribute 
        \item Causative - Determine the relationship between two variates 
        \item Predictive - Predict the response of a variate
    \end{itemize}
    \item Errors in Plans 
    \begin{itemize}
        \item Study Error - study population and target population 
        \item Sample Error - sample and study population 
        \item Measurement Error - measured value and true value of a variate 
    \end{itemize}
\end{itemize}

\paragraph{Chi-Squared}
Let $W\sim\chi^2(n)$
\begin{itemize}
    \item $\chi^2(1)\sim Z^2$ 
    \item $\chi^2(2)\sim\text{Exp}(2)$
    \item if $n>>50$, then $W\sim N(n,2n)$ 
\end{itemize}

\paragraph{T-distribution}
\begin{itemize}
    \item Symmetric around 0
    \item Kurtosis$>3$
\end{itemize}

\paragraph{Likelihood Functions}
\begin{itemize}
    \item Find $l(\theta)$, derive it, set it equal to $0$ and solve
\end{itemize}

\paragraph{Confidence Interval}
\begin{itemize}
    \item For Binomial, it is $\hat{\theta}\pm C\sqrt{\frac{\hat{\theta}(1-\hat{\theta})}{n}}$ 
    \item For Poisson, it is $\hat{\theta}\pm C\sqrt{\frac{\hat{\theta}}{n}}$ 
    \item For Exponential, it is $\hat{\theta}\pm C\frac{\hat{\theta}}{\sqrt{n}}$ 
    \item For Normal $\mu$, $\sigma$ known it is $\bar{y}\pm c\sigma$ 
    \item For Normal $\mu$, $\sigma$ unknown it is $\bar{y}\pm c\frac{s}{\sqrt{n}}$
    \item For Normal $\sigma$, is it $\left[\sqrt{\frac{(n-1)s^2}{b}},\sqrt{\frac{(n-1)s^2}{a}}\right]$, with $a,b$ such that 
    \item To convert from p\% likelihood interval to confidence coefficient, use $2P(Z\leq\sqrt{-2\log p})-1$
    \item To convert from confidence interval to likelihood interval, use $\lbrace\theta:R(\theta)\geq e^{-\frac{a^2}{2}}\rbrace$. ie. given a q\% confidence interval, it is a $e^\frac{-q^*}{2}$\% likelihood interval. 
    \item If we want to choose a sample size to guarantee our margin of error will be less than $\pm l$, then we choose $n\geq\left(\frac{Z^*}{l}\right)^2\times\frac{1}{4}$
\end{itemize}

\paragraph{Prediction Interval}
For a new value $Y\sim G(\mu,\sigma)$, a 100p\% prediction interval is given by $\bar{y}\pm c\cdot s\sqrt{1+\frac{1}{n}}$, where $P(T\leq c)=\frac{1+p}{2}$ where $T\sim t(n-1)$ 







\end{document}