\documentclass[10pt,english]{article}
\usepackage[T1]{fontenc}
\usepackage[latin9]{inputenc}
\usepackage{geometry}
\geometry{verbose,tmargin=1.5in,bmargin=1.5in,lmargin=1.5in,rmargin=1.5in}
\usepackage{amsthm}
\usepackage{amsmath}
\usepackage{amssymb}

\makeatletter
\usepackage{enumitem}
\newlength{\lyxlabelwidth}

\usepackage[T1]{fontenc}
\usepackage{ae,aecompl}

%\usepackage{txfonts}

\usepackage{microtype}

\usepackage{calc}
\usepackage{enumitem}
\setenumerate{leftmargin=!,labelindent=0pt,itemindent=0em,labelwidth=\widthof{\ref{last-item}}}

\makeatother

\usepackage{babel}
\begin{document}
\noindent \begin{center}
\textbf{\large{}MATH 239 - Assignment 2}\\
\textbf{\large{}Chris Ji 20725415}
\par\end{center}{\large \par}
\medskip{}

\begin{enumerate}
\item \begin{enumerate}
    \item Let $S_1$ be the set of all even numbers in $\mathbb{N}_0$, and $S_2$ be the set of all odd numbers in $\mathbb{N}_0$. Then $w_1(a)=2a$, and $w_2(a)=a+2$. Then $\Phi_{S_1}^{w_1}(x)=\sum_{n\in S_1}x^{2n}=x^0+x^4+x^8+x^{12}+x^{16}\ldots$, and $\Phi_{S_2}^{w_2}(x)=\sum_{n\in S_2}x^{n+2}=x^3+x^5+x^7+x^9+x^{11}+x^{13}\ldots$. By inspection, we can see $\Phi_S^w(x)=\sum_{n=0}(x^n-x^{4n+2})-x^1=\frac{1}{1-x}-\frac{x^2}{1-x^4}-x=-\frac{x^5+x^3+1}{(x-1)(x+1)(x^2+1)}$, as $\sum_{n=0}x^{4n+2}=\sum_{n=0}x^2x^{4n}=x^2\sum_{n=0}x^{4n}=\frac{x^2}{1-x^4}$
    
    
    
    
    \item By inspection (refer to the matrices below, the numbers are the exponents of $x$), we can see that the series that result from $S_1=1$ and $S_1=3$ are the same past the first 2 terms of $S_1$, since $w(a)=2a$, $S_3-S_1=3-1=2$, and $2(2)=4$, which is the rate of change in the infinite series due to $w(c)$. The exponents of $x$ that come from $S_1=2$ are unique, so the coefficients of $x^{4n}$ and $x^{4n-1}$ for all $n>1$ is $2$, and the coefficients of $x^{4n-2}$ and $x^{4n-3}$ for all $n>1$ is $1$. Writing this in sigma notation, $\Phi_{A\times B\times C}^w(x)=\sum_{n=3}x^n+\sum_{n=2}(x^{4n}+x^{4n-1})=x\left(\sum_{n=2}x^n\right)+\sum_{n=2}(x^{4n}+x^{4n-1})=x\left(\sum_{n=2}x^n+x^{4n-1}+x^{4n-2}\right)=\frac{-x(x^4+x^2+1)}{x^2(x^2+1)(x-1)}$ \\ \\ \\ \\ \\ \\ 
% Let $A=\{1,2,3\},B=\{1,2,\ldots,2018\}$, and $C=\mathbb{N}_0$. Then $w_A(\alpha)=2\alpha,w_B(\beta)=\beta$, and $w_C(\gamma)=4\gamma$. Obviously, $w((A,B,C))=w_A(\alpha)+w_B(\beta)+W_c(\gamma)$. Then, by the product lemma, $\Phi_{A\times B\times C}^{w}(x)=\Phi_A^{w_\alpha}(x)\cdot\Phi_B^{w_\beta}(x)\cdot\Phi_C^{w_\gamma}(x)$. \\ $\Phi_A^{w_\alpha}(x)=\sum_{\alpha\in A}x^{w(\alpha)}=x^2+x^4+x^6$, $\Phi_B^{w_\beta}(x)=\sum_{\beta\in B}x^{w(\beta)}=x^1+x^2+\ldots+x^{2018}$, and $\Phi_C^{w_\gamma}(x)=\sum_{\gamma\in C}x^{w(\gamma)}=x^0+x^4+x^8+\ldots$. Then $\Phi_{A\times B\times C}^{w}(x)=\sum_{n=1}^3x^{2n}\cdot\sum_{n=1}^{2018}x^n\cdot\sum_{n=0}x^{4n}$. 

The following is an expansion of the exponents of $x$ of $(x^2+x^4+x^6)\cdot(x^1+x^2+\ldots+x^{2018})\cdot(x^0+x^4+\ldots)$ \\ 
$$x^2\cdot(x^1+x^2\ldots+x^{2018})\cdot(x^0+x^4+\ldots)=\begin{matrix}2+1+0=3&2+2+0=4&\cdots&2+2018+0=2020\\ 2+1+4=7&2+2+4=8&\cdots&2+2018+4=2024\\\vdots&\vdots&\cdots&\vdots\end{matrix}$$
$$x^4\cdot(x^1+x^2\ldots+x^{2018})\cdot(x^0+x^4+\ldots)=\begin{matrix}4+1+0=5&4+2+0=6&\cdots&4+2018+0=2022\\ 4+1+4=9&4+2+4=10&\cdots&4+2018+4=2026\\\vdots&\vdots&\cdots&\vdots\end{matrix}$$
$$x^6\cdot(x^1+x^2\ldots+x^{2018})\cdot(x^0+x^4+\ldots)=\begin{matrix}6+1+0=7&6+2+0=8&\cdots&6+2018+0=2024\\ 6+1+4=11&6+2+4=12&\cdots&6+2018+4=2028\\\vdots&\vdots&\cdots&\vdots\end{matrix}$$
    
\end{enumerate}

\pagebreak
\item Taking the hint, there are obviously $3^n$ ternary strings of length $n$, as for each index in the string, there are $|\{0,1,2\}|=3$ choices, so for every added element, there are $\times3$ more choices, hence $3^n$ strings. \\ 
Now choose $n-k$ places in the string to be, say, $0$. There are ${n\choose n-k}={n\choose k}$ ways you can choose this. Then, for the remaining $n-(n-k)=k$ objects, they will be either $1$ or $2$, so there are $\sum_{k=0}^n{n\choose k}=2^k$ ways to choose them. Hence, there are $\sum_{k=0}^n2^k{n\choose k}$ ternary strings of length $n$. \\ 
Since $3^n$ and $\sum_{k=0}^n2^k{n\choose k}$ are both counting the number of ternary strings of length $n$, $3^n=\sum_{k=0}^n2^k{n\choose k}$. 

\pagebreak
\item \begin{enumerate}
    \item This series can be represented by $-A(-x)$. This makes every odd index positive, and every even index negative, as an even parity exponent usually results in a positive, but in this series we need them to be negative, so we turn the whole thing negative. 
    
    \item Every $a_n$ is still in the sequence, just that the corresponding $x$'s exponent is $2n+1$, as opposed to just $n$ (the first few terms are $a_0x^1,a_1x^3,a_2x^5,a_3x^7$). Then by inspection, we can see this series can be represented by $xA(x^2)$. 
    
    \item $a_0x^0,(a_1+2a_0)x^1,(a_2+2a_1-a_0)x^2,(a_3+2a_2-a_1)x^3$. From this we can see that every $a_n$ will be the coefficients of the following: $a_nx^n+ 2a_nx^{n+1}-a_nx^{n+2}$. From this we can see that a generating series for this sequence is $A(x)+2xA(x)-x^2A(x)=(1+2x-x^2)A(x)$.  
\end{enumerate}

\pagebreak
\item Note that $\sum_{n=0}^\infty x^n=\frac{1}{1-x}$. Also note that $\Phi_S^w(x)=\sum_{\sigma\in S}x^{w(\sigma)}=\sum_{k\geq0}a_kx^k$. Then $\frac{\Phi_S^w(x)}{1-x}=
\sum_{k\geq0}a_kx^k\cdot\sum_{n=0}x^n=\sum_{n=0}a_nx^nx^n=\sum_{n=0}a_nx^{2n}$. This is just the definition of a generating series, with $x^{2n}$ instead of $x^n$. Then by definition, the coefficients of $x^{2n}$ in $\frac{\Phi_S^w(x)}{1-x}$ counts the number of elements of weight $2n$ in $S$. 

\pagebreak
\item $$A(2x)=\sum_{n\geq0}\frac{1}{n!}(2x)^n=\sum_{n\geq0}\frac{1}{n!}2^nx^n=\sum_{n\geq0}\frac{1}{n!}\left(\sum_{k\geq0}\frac{n!}{k!(n-k)!}\right)x^n=\sum_{n\geq0}\sum_{k\geq0}\frac{1}{k!(n-k)!}x^n$$  \\ 
$$(A(x))^2=\left(\sum_{n\geq0}\frac{1}{n!}x^n\right)^2=\sum_{n\geq0}\frac{1}{n!}x^{n}\sum_{n\geq0}\frac{1}{n!}x^{n}=\sum_{n\geq0}\left(\sum_{k\geq0}\frac{1}{k!}\cdot\frac{1}{(n-k)!}\right)x^n$$ By the definition of formal power series multiplication.\\ Then $\sum_{n\geq0}\sum_{k\geq0}\frac{1}{k!(n-k)!}x^n=\sum_{n\geq0}\left(\sum_{k\geq0}\frac{1}{k!}\cdot\frac{1}{(n-k)!}\right)x^n$, clearly. 


\end{enumerate}

\end{document}
