\documentclass[10pt,english]{article}
\usepackage[T1]{fontenc}
\usepackage[latin9]{inputenc}
\usepackage{geometry}
\geometry{verbose,tmargin=1.5in,bmargin=1.5in,lmargin=1.5in,rmargin=1.5in}
\usepackage{amsthm}
\usepackage{amsmath}
\usepackage{amssymb}
\usepackage[usenames, dvipsnames]{color}

\makeatletter
\usepackage{enumitem}
\newlength{\lyxlabelwidth}

\usepackage[T1]{fontenc}
\usepackage{ae,aecompl}

%\usepackage{txfonts}

\usepackage{microtype}

\usepackage{calc}
\usepackage{enumitem}
\setenumerate{leftmargin=!,labelindent=0pt,itemindent=0em,labelwidth=\widthof{\ref{last-item}}}

\makeatother

\usepackage{babel}
\begin{document}
\noindent \begin{center}
\textbf{\large{}MATH 239 - Assignment 1}\\
\textbf{\large{}Chris Ji 20725415}
\par\end{center}{\large \par}
\medskip{}

\begin{enumerate}
\item Note that $|A\cup B\cup C|=|A|+|B|+|C|-|A\cap B|-|B\cap C|-|A\cap C|+|A\cap B\cap C|$ (a proof for this is below). Subbing that out on the right side, we are left with $\left(|A|+|B|+|C|-|A\cap B|-|B\cap C|-|A\cap C|+|A\cap B\cap C|\right)+|A\cap B|+|A\cap C|+|B\cap C|$ on the right, which simplifies to $|A|+|B|+|C|+|A\cap B\cap C|$, which we can see is exactly equal to the left side. 
\\ From the picture below, everything encased in the red areas (red, orange, and yellow) are all added once ($|A|+|B|+|C|$), but then we have 2 of everything encased in the orange areas (orange and yellow), so we must subtract each ($|A\cap B|$, $|A\cap C|$, and $|B\cap C$). But then we have subtracted the yellow area three times, so we must add it back ($|A\cap B\cap C$). 

\pagebreak

\item Binary strings of length 1: 0\quad1. Changes: 2 with zero changes.\\ Binary strings of length 2: 00\quad01\quad10\quad11. Changes: 2 with zero, 2 with one.\\Binary strings of length 3: 000\quad001\quad010\quad011\quad100\quad101\quad110\quad111. Changes: 2 with zero, 4 with one, and 2 with two. By inspection, we can see a pattern similar to Pascal's triangle emerging, and so we can deduce that $2{n-1\choose k}$ binary strings of length $n$ have $k$ changes. \\
Proof: As shown above, this was true for all $k$ for $n=0$. By induction, we can assume that this will hold true for all $n=a$, $a\in\mathbb{N}$. For $n=a+1$, we have double the amount of strings present in $n=a$, but we can partition the set of all $k$ with $n=a+1$ into two equal sets, one with each element in the set with $n=a$, with a zero in front, and one with each element in the set with $n=a$ with a one in front. Adding a 1 in front of every element starting with a 0 would add a change, and adding a 0 in front of every element starting with a 1 would also add a change. Adding a 1 in front of every element starting with a 1 does not add any changes, and adding a 0 to every element starting with a 0 also does not add any changes. So half of the strings will have +1 change, and half will have +0 change. If we start with the row $n=0$, and write each resulting $n=n+1$ row below it, we can see this results in a triangle, with the total number of elements with $k$ changes being sum of amount of elements in the row above it with the same amount of $k$ changes, and the number of elements in the row above it with $k-1$ amount of changes. This is exactly the relationship in Pascal's triangle, which is represented by ${n\choose k}$. The reason it is $2{n-1\choose k}$ is because there are only a maximum of $n-1$ changes for each $n$ amount of elements, and because there are 2 possible elements (0 and 1), each added element results in double the amount of choices, hence the factor of 2. 

\pagebreak

%Consider partitioning the binary strings (written in increasing order of binary numbers) in the middle, and reversing one of the sides. We are left with two similar binary strings, with one side being the complementary binary string of the other. Note that each string and its complement have the same number of changes. Using this method, we can just count one of the halves. \pagebreak

 
\item \begin{enumerate}
    \item All binary strings of length 3, 4, and 5 are listed below. The binary strings that do not have consecutive zeros are in red, with the total amount below each list. 
\\Binary strings of length 3: 000\quad001\quad\textcolor{Red}{010}\quad\textcolor{Red}{011}\quad100\quad\textcolor{Red}{101}\quad\textcolor{Red}{110}\quad\textcolor{Red}{111}
\\There are 5 strings.
\\Binary strings of length 4: 0000\quad0001\quad0010\quad0011\quad0100\quad\textcolor{Red}{0101}\quad\textcolor{Red}{0110}\quad\textcolor{Red}{0111}\quad1000\quad1001\quad\textcolor{Red}{1010}\quad\textcolor{Red}{1011}\quad1100\quad\textcolor{Red}{1101}\quad\textcolor{Red}{1110}\quad\textcolor{Red}{1111}
\\There are 8 strings.
\\Binary strings of length 5: 00000\quad00001\quad00010\quad00011\quad00100\quad00101\quad00110\quad00111\quad01000\quad01001\quad\textcolor{Red}{01010}\quad\textcolor{Red}{01011}\quad01100\quad\textcolor{Red}{01101}\quad\textcolor{Red}{01110}\quad\textcolor{Red}{01111}\quad10000\quad10001\quad10010\quad10011\quad10100\quad\textcolor{Red}{10101}\quad\textcolor{Red}{10110}\quad\textcolor{Red}{10111}\quad11000\quad11001\quad\textcolor{Red}{11010}\quad\textcolor{Red}{11011}\quad11100\quad\textcolor{Red}{11101}\quad\textcolor{Red}{11110}\quad\textcolor{Red}{11111}
\\There are 13 strings. 

\item I guess that there will be 21 strings of 6 numbers that do not have leading zeros. This pattern is obviously the Fibonacci sequence, and the relationship is that, with $s_n$ being the number of binary strings of size $n$ without leading zeros, $s_{n-1}+s_{n-2}=s_n$. \pagebreak
\end{enumerate}

\item \begin{enumerate}
    \item Since $S$ has an odd cardinality, if $|A|$ is odd, then $|S\backslash A|$ must be even, as the difference between two odd numbers is always even ($S$ has an odd number of elements, $A$ has an odd number of elements). Similarly, if $|A|$ is even, then $|S\backslash A|$ must be odd, as the difference between an even and an odd number is always odd. Therefore, for every odd subset $A$ of $S$, there is a unique even subset of $S$, $S\backslash A$, so there are an equal amount of each.  
    
    \item We can represent the power set of $S$ as the set of all binary strings with the length of the binary strings equal to $n$. Each digit in the binary string represents whether or not the element in $S$ is in that particular subset (say 1 means that that element is in). We can split the set of binary strings into 2 subsets, those with even amounts of 1's (and so the related subset will have even cardinality), and those with odd amounts of 1's (and so the related subset will have odd cardinality). If the binary strings are sorted in increasing order, then we can see there is a direct bijection. For every string on the even side of the partition ending with a one, there is a string on the odd side that is the same, but it ends with a zero. For every string on the even side of the partition ending with a zero, there is a string on the odd side that is the same, but it ends with a one. 
\pagebreak
\end{enumerate}

\item \begin{enumerate}
    \item A set $S$ with $|S|=n$ will have $2^n$ subsets. ${n\choose0}$ subsets of size 0, ${n\choose1}$ subsets of size 1, ${n\choose 2}$ subsets of size 2, up to ${n\choose n}$ subsets of size $n$. Since each subset (say the size of the subset is $k$) will have $k$ elements, there are $k$ possible ordered pairs to be generated from each subset. So there are $\sum_{k=0}^nk{n\choose k}$ possible subsets. 
    
    \item Since the subsets have to have $k$ elements, and there are $n$ total elements, there are ${n\choose k}$ subsets. For each subset, there are obviously $k$ elements, and so there are $k*{n\choose k}$ ordered pairs. 
    
    \item (a) and (b) are related through the fact that (a) is counting up \textbf{all} of the subsets of size $k=\{1,2,\ldots,n\}$, whereas (b) is only counting up \textbf{one} subset of size $k\in\{1,2,\ldots,n\}$. Therefore, if you add up all of the $n$ answers of (b), you get the answer for (a). In an equation, $$\left(1*{n\choose 1}\right)+\left(2*{n\choose 2}\right)+\ldots+\left(n*{n\choose n}\right)=\sum_{k=0}^nk*{n\choose k}$$
    
\end{enumerate}

\end{enumerate}

\end{document}
