\documentclass[10pt,english]{article}
\usepackage[T1]{fontenc}
\usepackage[latin9]{inputenc}
\usepackage{geometry}
\geometry{verbose,tmargin=1.5in,bmargin=1.5in,lmargin=1.5in,rmargin=1.5in}
\usepackage{amsthm}
\usepackage{amsmath}
\usepackage{amssymb}

\makeatletter
\usepackage{enumitem}
\newlength{\lyxlabelwidth}

\usepackage[T1]{fontenc}
\usepackage{ae,aecompl}

%\usepackage{txfonts}

\usepackage{microtype}

\usepackage{calc}
\usepackage{enumitem}
\setenumerate{leftmargin=!,labelindent=0pt,itemindent=0em,labelwidth=\widthof{\ref{last-item}}}

\makeatother

\usepackage{babel}
\begin{document}
\noindent \begin{center}
\textbf{\large{}MATH 239 - Assignment 3}\\
\textbf{\large{}Chris Ji 20725415}
\par\end{center}{\large \par}
\medskip{}

\begin{enumerate}
\item \begin{enumerate}
    \item $\frac{(1+x)^2}{1-x^3}=(1+x)^2\cdot\frac{1}{1-x^3}=(1+x)^2\cdot\sum_{n=0}^\infty x^{3n}=\sum_{n=0}^\infty x^{3n}+2x\sum_{n=0}^\infty x^{3n}+x^2\sum_{n=0}^\infty x^{3n}$. Note that the first term makes the coefficients of $x$ with powers congruent to $0$ mod $3$ 1, the second term makes the coefficients of $x$ with powers congruent to $1$ mod $3$ 2, and the last term makes the coefficients of $x$ with powers congruent to $2$ mod $3$ 1. As $2018\equiv2\quad\text{mod}\,3$, so $[x^{2018}]=1$ 
    
    \item $\left(\frac{1}{1-5x^2}\right)^{10}=\sum_{n=0}^\infty{n+9\choose9}(\sqrt{5}x)^{2n}$, and $\left(\frac{1}{1-3x}\right)^{100}=\sum_{k=0}^\infty{k+99\choose99}(3x)^k$. So then  $[x^{1009}]\left(\frac{1}{1-5x^2}\right)^{10}\left(\frac{1}{1-3x}\right)^{100}=[x^{1009}]\left(\sum_{n=0}^\infty{n+9\choose9}(\sqrt{5}x)^{2n}\right)\left(\sum_{k=0}^\infty{k+99\choose99}(3x)^k\right) = \sum_{2k+n=1009,k\geq0,n\geq0}{k+9\choose9}{n+99\choose99}5^k3^n$
    \end{enumerate}

\pagebreak
\item Note first that $n$ must obviously be greater than 3, as all the compositions have at least 4 parts. Observe that for a composition of $n$ with 4 parts where each part is congruent 1 mod 3, $n$ must be the sum of $4$ numbers that are all 1 mod 3, and so $n$ must be 1 mod 3. For a composition of $n$ with 5 parts where each part is congruent 1 mod 3, $n$ must be the sum of $5$ numbers that are all 1 mod 3, and so $n$ must be 2 mod 3. Therefore, for $n\equiv0\quad\text{mod}\,3$, there are no compositions of 4 or 5. \\ 
For $n\equiv1\,\,\text{mod}\,3$, let $S=\mathbb{N}^4_{\text{1 mod 3}}$, where $\mathbb{N}_{\text{1 mod 3}}=\{1,4,7,10,\ldots\}$ Then the required compositions are the elements of $S$ of weight $n$, so the required generating series is $\Phi_S(x)=\Phi_{\mathbb{N}^4_{\text{1 mod 3}}}(x)=\left(\Phi_{\mathbb{N}_{\text{1 mod 3}}}(x)\right)^4$ by product lemma. This is $\left(\sum_{k\geq0}x^{3k+1}\right)^4=\left(\frac{x}{1-x^3}\right)^4$ by geometric series. So then the number of compositions of the required form is $[x^n]x^4(1-x^3)^{-4}=[x^{n-4}](1-x^3)^{-4}=[x^{n-4}]\sum_{i\geq0}{i+3\choose3}x^{3i}$. The coefficient is zero if $n-4$ is not 1 mod 3. If $n-4$ is 1 mod 3, then the required coefficient occurs for $i=\frac{n-4}{3}$. So the required coefficient is ${i+3\choose3}={\frac{n-4}{3}+3\choose 3}={n+5\choose 3}$. \\ 
For $n\equiv2\,\,\text{mod}\,3$, Let $S=\mathbb{N}^5_{\text{2 mod 3}}$, where $\mathbb{N}_{\text{1 mod 3}}=\{1,4,7,10,\ldots\}$ Then the required compositions are the elements of $S$ of weight $n$, so the required generating series is $\Phi_S(x)=\Phi_{\mathbb{N}^5_{\text{1 mod 3}}}(x)=\left(\Phi_{\mathbb{N}_{\text{1 mod 3}}}(x)\right)^5$ by product lemma. This is $\left(\sum_{k\geq0}x^{3k+1}\right)^5=\left(\frac{x}{1-x^3}\right)^5$ by geometric series. So then the number of compositions of the required form is $[x^n]x^5(1-x^3)^{-5}=[x^{n-5}](1-x^3)^{-5}=[x^{n-5}]\sum_{i\geq0}{i+4\choose4}x^{3i}$. The coefficient is zero if $n-5$ is not 1 mod 3, so if $n-5$ is 1 mod 3, then the required coefficient occurs for $i=\frac{n-5}{3}$. So the required coefficient is ${i-3\choose3}={\frac{n-5}{3}+4\choose 4}={n+7\choose 4}$.

\pagebreak
\item Let $S_k$ represent a subset of $S$ with $k$ components. Then $\Phi_S^w=\sum_{k=0}(\Phi_{S_k}^w(x))$. Note that a generating series for all even elements is $x^2+x^4+\ldots=x^2\sum_{n=0}x^{2n}=\frac{x^2}{1-x^2}$, and a generating series for all odd elements is $x+x^3+\ldots=\sum_{n=0}x^{2n+1}=\frac{x}{1-x^2}$. If $k$ is even, then the amount of odd elements will be equal to the amount of even elements.  So a generating series for all $S_k$ such that $k$ is even is $\sum_{i\geq0}\left(\frac{x}{1-x^2}\cdot\frac{x^2}{1-x^2}\right)^i=\sum_{i\geq0}\left(\frac{x^3}{(1-x^2)^2}\right)^i$. If $k$ is odd, then there will be one more odd element than even element, so a generating series for all $S_k$ such that $k$ is odd is $\sum_{i\geq0}\left(\frac{x}{1-x^2}\right)^{i+1}\left(\frac{x^2}{1-x^2}\right)^i=\left(\frac{x}{1-x^2}\right)\sum_{i\geq0}\left(\frac{x^3}{(1-x^2)^2}\right)$. Then by sum lemma, $\Phi_{S_k}^w(x)=\Phi_{S_{k\text{ that are even}}}+\Phi_{S_{k\text{ that are odd}}}=\sum_{i\geq0}\left(\frac{x^3}{(1-x^2)^2}\right)^i+\left(\frac{x}{1-x^2}\right)\sum_{i\geq0}\left(\frac{x^3}{(1-x^2)^2}\right)=\left(1+\frac{x}{1-x^2}\right)\sum_{i\geq0}\left(\frac{x^3}{(1-x^2)^2}\right)^i=\left(1+\frac{x}{1-x^2}\right)\left(\frac{1}{1-\left(\frac{x^3}{(1-x^2)^2}\right)}\right)=\left(1+\frac{x}{1-x^2}\right)\left(\frac{(1-x^2)^2}{(-x^2+1)^2-x^3}\right)=\frac{(-x^2+x+1)(-x^2+1)}{(-x^2+1)^2-x^3}$

\pagebreak
\item \begin{enumerate}
    \item Split $S_n$ into two sets $A$, and $B$. Let $A$ be all of the compositions that start with 2, and $B$ be all of the compositions that don't start with 2 (and therefore start with something greater than 2). \\ 
Define the function $f:A\rightarrow S_{n-2}$ to be $f(\{2,n_1,\ldots,n_k\})\rightarrow(\{n_1,\ldots,n_k\})$. Then clearly it is a function to $S_{n-2}$, as a two has been removed from $A$, changing its domain from $S_n$ to $S_{n-2}$. Furthermore, we can define $f^{-1}:S_{n-2}\rightarrow A$ as $f^{-1}(\{n_1,\ldots,n_k\})=\{2,n_1,\ldots,n_k\}$. Note that $f(f^{-1}(A))=A$ and $f^{-1}(f(A))=A$. So there is a bijection from $A$ to $S_{n-2}$. \\ 
Now define the function $g:B\rightarrow S_{n-1}$ to be $g(\{n_1,\ldots,n_k\})=\{n_1-1,\ldots,n_k\}$. Then clearly it is a function to $S_{n-1}$, as one of the elements has been decremented by 1, changing the domain of the composition from $S_n$ to $S_{n-1}$. Furthermore, since the first element was greater than 2, the element is still within our definition of $S_n$. Define $g^{-1}:S_{n-1}\rightarrow B$ as $g(\{n_1,\ldots,n_k\})=\{n_1+1,\ldots,n_k\}$. Note that $g(g^{-1}(B))=B$, and $g^{-1}(g(B))=B$. So there is a bijection from $B$ to $S_{n-1}$. 
Since $|S_n|=|A|+|B|$, and $A\mapsto S_{n-2}, B\mapsto S_{n-1}$, $|S_n|=|S_{n-2}|+|S_{n-1}|$.

\item Note that a generating series for compositions of size $n>2$ is $x^2+x^3+x^4+\ldots=x^2\sum_{n\geq0}x^n$. By product lemma, the generating series for all compositions of size $k$ is equal to $\left(x^2\cdot\sum_{n\geq0}x^n\right)^k=\left(\frac{x^2}{1-x}\right)^k$. Then to find all the compositions of all sizes $k$ we get $\Phi_S(x)=\sum_{k\geq0}\left(\frac{x^2}{1-x}\right)^k=\frac{1}{1-\frac{x^2}{1-x}}=\frac{1-x}{1-x-x^2}$. Then $[x^n](1-x-x^2)\Phi_S(x)=[x^n](1-x-x^2)\frac{1-x}{1-x-x^2}=[x^n]1-x$. So obviously the coefficient of $x^n$ is 1 when $n=0$, $-1$ when $n=1$, and 0 otherwise. 

\end{enumerate}


\pagebreak
\item \begin{enumerate}
    \item Assume that there exists a $C(x)$ such that $A(x)C(x)=1$. Let $C(x)=\sum_{k=0}c_kx^k$. Then $A(x)C(x)=\frac{1-x}{1+3x+6x^2}\sum_{k=0}c_kx^k=\frac{\sum_{k=0}(c_kx^k(1-x))}{1+3x+6x^2}=\frac{\sum_{k=0}c_kx^k}{1+3x+6x^2}-x\frac{\sum_{k=0}c_kx^k}{1+3x+6x^2}=\frac{c_0x^0}{1+3x+6x^2}-\frac{c_0x}{1+3x+6x^2}+\frac{c_1x}{1+3x+6x^2}-\frac{c_1x^2}{1+3x+6x^2}+\ldots=\frac{c_0x^0}{1+3x+6x^2}+\frac{x}{1+3x+6x^2}(c_1-c_0)+\frac{x^2}{1+3x+6x^2}(c_2-c_1)+\ldots$. So then $\frac{c_0}{1+3x+6x^2}=1,c_1-c_0=0,c_2-c_1=0,\ldots$ and for all $n\geq0,c_n=1+3x+6x^2$. So $\left(\frac{1-x}{1+3x+6x^2}\right)^{-1}=1+3x+6x$
    
    Assume that there exists a $Q(x)$ such that $B(x)Q(x)=1$. Note that $B(x)=\frac{x}{1-x^3}=x\sum_{n=0}x^{3n}$ by geometric series. Let $Q(x)=\sum_{k=0}q_kx^k$. Then $B(x)Q(x)=x\sum_{n=0}\left(\sum_{k=0}^nq_k\right)x^{3n}=1$. Note that the right side has a constant term of 1, but the left side does not have a constant term, as the entire series is multiplied by $x$. Therefore $B(x)$ does not have an inverse. 
    
    \item Since $B(x)=\frac{x}{1-x^3}=x\sum_{n=0}x^{3n}$, there is no constant term in $B(x)$. Then, by theorem 1.7.10, $A(B(x))$ is a formal power series, as the constant term of $B(x)$ is 0.
    \\

Again note that $B(x)=\frac{x}{1-x^3}=x\sum_{n=0}x^{3n}$. Then $B(A(x))=B\left(\frac{1-x}{1+3x+6x^2}\right)=\left(\frac{1-x}{1+3x+6x^2}\right)\sum_{n=0}\left(\frac{1-x}{1+3x+6x^2}\right)^{3n}=\left(\frac{1-x}{1+3x+6x^2}\right)+\left(\frac{1-x}{1+3x+6x^2}\right)^4+\ldots$. Note that this series does not have a constant term, so $B(A(x))$ is not a formal power series. 
    
    
\end{enumerate}
\end{enumerate}

\end{document}
