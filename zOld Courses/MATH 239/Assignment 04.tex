\documentclass[10pt,english]{article}
\usepackage[T1]{fontenc}
\usepackage[latin9]{inputenc}
\usepackage{geometry}
\geometry{verbose,tmargin=1.5in,bmargin=1.5in,lmargin=1.5in,rmargin=1.5in}
\usepackage{amsthm}
\usepackage{amsmath}
\usepackage{amssymb}

\makeatletter
\usepackage{enumitem}
\newlength{\lyxlabelwidth}

\usepackage[T1]{fontenc}
\usepackage{ae,aecompl}

%\usepackage{txfonts}

\usepackage{microtype}

\usepackage{calc}
\usepackage{enumitem}
\setenumerate{leftmargin=!,labelindent=0pt,itemindent=0em,labelwidth=\widthof{\ref{last-item}}}

\makeatother

\usepackage{babel}
\begin{document}
\noindent \begin{center}
\textbf{\large{}MATH 239 - Assignment 1}\\
\textbf{\large{}Chris Ji 20725415}
\par\end{center}{\large \par}
\medskip{}

\begin{enumerate}
\item \begin{enumerate}
    \item Note that \begin{gather*}A\times B=\{(01,0111),(01,11),(01,101),(00,0111),(00,11),(00,101),(110,0111),\\(110,11),(110,101),(0101,0111),(0101,11),(0101,101)\}\end{gather*}
    \begin{gather*}AB=\{010111,0111,01101,000111,0011,00101,1100111,11011,110101,01010111,0101101\}\end{gather*} So $\Phi_{AB}(x)=2x^4+3x^5+3x^6+2x^7+x^8$, but $\Phi_{A\times B}(x)=2x^4+3x^5+4x^6+2x^7+x^8$, as the string $010111$ was created by concatenating both $(01,0111)$ and $(0101,11)$. Therefore, $AB$ is ambiguous. \\ 
    Note that \begin{gather*}B\times A=\{(0111,01),(0111,00),(0111,110),(0111,0101),(11,01),(11,00),\\(11,110),(11,0101)(101,01),(101,00),(101,110),(101,0101)\}\end{gather*} \begin{gather*}BA=\{011101,011100,0111110,01110101,1101,1100,11110,110101,10101,10100,101110,1010101\}\end{gather*} So $\Phi_{BA}(x)=2x^4+3x^5+4x^6+2x^7+x^8=\Phi_{B\times A}(x)$, therefore $BA$ is unambiguous. 
    \item As above, $\Phi_{BA}(x)=\Phi_{B\times A}(x)=\Phi_{B}(x)\Phi_{A}(x)=\Phi_{A}(x)\Phi_{B}(x)$
\end{enumerate}
\pagebreak
\item
\begin{enumerate}
    \item Note that $\{0,11\}^*$ represents all binary strings in which every block of $1$'s has even length. Then $\Phi_{\{0,11\}}(x)=x+x^2$, and so therefore by theorem 2.6.1(b), $\Phi_{\{0,11\}^*}(x)=\frac{1}{1-(x+x^2)}$
    
    \item Note that $(\{0,1\}^*\{0\})^*\{1\}\{11\}^*(\{0\}\{0,1\}^*)^*$ represents all binary strings with at least one odd length block of 1's. $\Phi_{(\{0\}\{0,1\}^*)^*}=\Phi_{(\{0,1\}^*\{0\})^*}=\frac{1}{1-\Phi_{\{0,1\}^*\{0\}}}=\frac{1}{1-\Phi_{{\{0,1\}^*}}\Phi_{\{0\}}}=\frac{1}{1-\frac{1}{1-2x}(x)}=\frac{1-2x}{1-3x}$, and clearly $\Phi_{\{1\}\{11\}^*}=x\frac{1}{1-x^2}$, so $\Phi_{(\{0,1\}^*\{0\})^*\{1\}\{11\}^*(\{0\}\{0,1\}^*)^*}=\frac{1-2x}{1-3x}\cdot x\cdot\frac{1}{1-x^2}\cdot\frac{1-2x}{1-3x}=\frac{x(-2x+1)^2}{(-x^2+1)(-3x+1)^2}$
\end{enumerate}
\pagebreak
\item \begin{enumerate}
    \item As with the last assignment, we can prove it is the Fibonacci sequence by partitioning $a_n$ into two disjoint sets, $A,B$, and then proving $A\mapsto a_{n-1}$, and $B\mapsto a_{n-2}$. Let $A$ be the set of $a_n$ that starts with $1$. Define $f:A\rightarrow a_{n-1}$ to remove the 1 in front of every number in $a_{n-1}$. Then define $f^{-1}:a_{n-1}\rightarrow A$ to add a one in front of every number. Clearly this is bijective, as, for every string $x\in a_{n-1}$, $f^{-1}(f(1x))=f^{-1}(x)=1x$, and $f(f^{-1}(x))=f(1x)=x$. Also clearly, $f(a)\in S$ for all $a\in a_n$. \\ 
Let $B$ be the set of $a_n$ that doesn't start with a 1. Define $g:B\rightarrow a_{n-2}$ to remove 01 from every string that starts with one zero, and remove a 0 from the front and a 1 from the back of every string that starts with more than one zero. Then $g^{-1}:a_{n-2}\rightarrow B$ adds 01 to the front of every string that starts with 1, and adds 0 to the front and 1 to the back of every string that starts with a 0. Clearly, for every string $y_1\in a_{n-2}$ that starts with a 1, and $y_2\in a_{n-2}$ that starts with a 0, $g^{-1}(g(01y_1))=g^{-1}(y_1)=01y_1$, $g^{-1}(g(0y_21))=g^{-1}(y_2)=0y_21$, and $g(g^{-1}(y_1))=g(01y_1)=y_1$, $g(g^{-1}(y_2))=g(0y_21)=y_2$. Also clearly, $f(a)\in S$ for all $a\in a_n$.\\
Note that since $f$ takes $a\in a_n$ that start with 1, and $g$ takes $a\in a_n$ that start with 0, they are disjoint, and since they are bijective between $a_{n-1}$ and $a_{n-2}$ respectively, $\{a_n\}=|A|+|B|=\{a_{n-1}\}+\{a_{n-2}\}$, and hence, $\{a_n\}$ is the Fibonacci sequence.

\item 
\end{enumerate}

\pagebreak
\item By product lemma, \begin{align*}\Phi_{\{00\}\{0\}^*(\{111\}\{1\}^*\{0\}\{0\}^*)^*(\{\epsilon\}\cup\{11\}\{1\}^*) }(x)\\=\Phi_{\{00\}\{0\}^*}(x)\Phi_{(\{111\}\{1\}^*\{0\}\{0\}^*)^* }(x)\Phi_{\{\epsilon\}\cup\{11\}\{1\}^* }(x)\end{align*} We can do this because all of these strings are unambiguous. 
$$\Phi_{\{00\}\{0\}^* }(x)=\Phi_{\{00\}}(x)\Phi_{\{0\}^* }(x)=x^2\cdot\frac{1}{1-x}$$  
$$\Phi_{(\{111\}\{1\}^*\{0\}\{0\}^!*)^*}(x)=\frac{1}{1-\Phi_{\{111\}\{1\}^*\{0\}\{0\}^*}}=\frac{1}{1-(x^3)(\frac{1}{1-x})(x)(\frac{1}{1-x})}=\frac{(1-x)^2}{-x^4+x^2-2x+1}$$
$$\Phi_{\{\epsilon\}\cup\{11\}\{1\}^* }(x)=1+x^2\left(\frac{1}{1-x}\right)=\frac{x^2-x+1}{-x+1}$$ 
Then the generating function we want is $x^2\cdot\frac{1}{1-x}\cdot\frac{(1-x)^2}{-x^4+x^2-2x+1}\cdot\frac{x^2-x+1}{-x+1}=-\frac{x^2}{x^2+x-1}$

\pagebreak
\item Decomposing $\{0\}^*(\{1\}\{0\}^*)^*$, clearly the decomposition we want is given by $\{00000\}^*(\{1\}\{00000\}^*)^*$, as $\{00000\}^*$ is the subset of $\{0\}^*$ where all blocks of $0$ are divisible by $5$. This is still ambiguous as the string we are decomposing is ambiguous, and we are just removing strings we do not want. So then the generating function for this would be $\Phi_{\{00000\}^*(\{1\}\{00000\}^*)^*}(x)=\frac{1}{1-\Phi_{\{00000\}}}\frac{1}{1-\Phi_{\{1\} }(x)\Phi_{\{00000\}^*}(x)}=\frac{1}{1-x^5}\frac{1}{1-x\frac{1}{1-\Phi_{\{00000\} }}}=\frac{1}{1-x^5}\frac{1}{1-x\frac{1}{1-x^5}}=\frac{1}{1-x^5-x}$

\end{enumerate}

\end{document}
