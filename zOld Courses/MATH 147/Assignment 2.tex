\documentclass[10pt,letter]{article}
\usepackage{amsmath}
\usepackage{amssymb}
\usepackage{graphicx}
\usepackage{setspace}
\onehalfspacing
\usepackage{fullpage}
\begin{document}

\title{MATH147 Assignment 2}
\author{Chris Ji 20725415}

\maketitle 

\section*{Question 1}
As shown in class, \\
Let P be the statement S $\neq$ $\mathbb{N}$, and let T = N\textbackslash S. Then T $\neq$ $\Phi$ as S $\neq$ $\mathbb{N}$ and T $\subseteq$ $\mathbb{N}$. By the well-ordering principle for $\math{N}$, T has a minimum element $m_o$  so that m_o $\leq$ t for all t $\in$ T. Since 1 $\in$ S by the first rule of weak induction, and m_o $\in$ T = $\mathbb{N}$ \textbackslash S, we see that m_o $\neq$ 1. Thus m_o - 1 $\in$ $\mathbb{N}$. Also $m_o - 1 < m_o$ and $m_o \leq$ t for all t$\in$T. This implies that m_o - 1 $\notin$ T. Hence m_o - 1 $\in$ S. Thus m_o - 1 $\in$ $\math{N}$. Also $m_o - 1 < m_o$ and $m_o \leq$ t for all t$\in$T. This implies that m_o - 1 $\notin$ T. Hence m_o - 1 $\in$ S. But then by the second rule of weak induction, (m_o - 1) $\in$ S implies that (m_o = (m_o - 1) + 1) $\in$ S. But then m_o $\in$ S $\cup$ T = $\phi$, a contradiction. This shows that P was false, therefore S = $\math{N}$, and (a) implies (b).\\

Let's assume for statement Q(n) that the first two rules of strong induction have already been proven to be true. Let P(k) now be the statement that "Q(n) is true for all n $\leq$ k". Since Q(1) is true, then P(1) must also be true. Q(k+1) is also true by our first assumption, so if P(k) is true, then Q(k+1) is also true. If Q(k+1) and P(k) are both true, then that implies P(k+1) is also true. Therefore, (b) implies (c).\\

Let $\phi \neq S \subseteq \mathbb{N}$ such that a minimum element, $m_o$, doesn't exist. $n = 1 \notin S$, as then it would be $m_o$. Similarly, $n = 2 \notin S$, as 1 doesn't exist, so 2 would be $m_o$. 1, 2, ... n $\notin S$. Then by induction, $(n + 1) \notin S$, as (n + 1) would be $m_o$. But then S would be empty, a contradiction.  Therefore, $m_o$ does exist, and (c) implies (a). 

\section*{Question 2} 
\paragraph{(a)} 
Since $y>x$, $y-x > 0$. By the second Archimedean Property, we can find $n \in \mathbb{N}$ such that $y-x > \frac{1}{n}$, such that $\frac{1}{n}$ is as small as possible. Let us also choose $a \in \mathbb{N}$ so that $\frac{a}{n} \leq x$, such that $\frac{a}{n}$ is as large as possible. Therefore, $x < \frac{a+1}{n}$. Also, $\frac{a+1}{n} < y$ because of how we chose a and n. Therefore, $x < \frac{a+1}{n} < y$ and $\frac{a+1}{n} \in \mathbb{Q}$

\paragraph{(b)} 
Let's assume $\sqrt{2}$ is rational. Therefore, $\sqrt{2} = \frac{p}{q}$, where p, q $\in \mathbb{Z}$ and p and q have no common factors. Squaring both sides... 
$$2 = \frac{p^2}{q^2}$$ 
$$2q^2 = p^2$$
$2q^2$ is even, so p must be even. Let p = 2a $$2q^2 = (2a)^2$$ $$q^2 = 2a^2$$
Therefore q must also be even. If q and p are both even, then they have a common factor of 2, therefore they are reducible. But by definition, p and q have no common factor, a contradiction, so $\sqrt{2} \neq \frac{p}{q}$ where p, q $\in \mathbb{Z}$.

\paragraph{(c)}
$$ x < y$$
$$ x + \sqrt{2} < y + \sqrt{2}$$ 
Using the proof from (a), there must exist a rational number q between $x + \sqrt{2}$ and $y + \sqrt{2}$.
$$ x + \sqrt{2} < q < y + \sqrt{2}$$
$$ x < q - \sqrt{2} < y$$
$q - \sqrt{2}$ is irrational, therefore there exists an irrational number between any two real numbers.

\section*{Question 3}
$$\sum_{j=1}^{n}((j+1)^3 - j^3) = (n+1)^3 - 1^3 = n^3 + 3n^2 + 3n$$
But $(j+1)^3 - j^3 = 3j^2 + 3j + 1$, and so we have just shown that 
$$\sum_{j=1}^{n}(3j^2 + 3j + 1) = 3\sum_{j=1}^{n}j^2 + 3\sum_{j=1}^{n}j + \sum_{j=1}^{n}1 = n^3 + 3n^2 + 3n$$
Substituting $\sum_{j=1}^{n}(j) = \frac{n(n+1)}{2}$ and $\sum_{j=1}^{n}(1) = n$ and doing a bit of algebra...
$$\sum_{j=1}^{n}j^2 =  \frac{2n^3 + 3n^2 + n}{6}$$

For $n = 1$...
$$1^2 = \frac{2(1)^3 + 3(1)^2 + 1}{6}$$
$$1 = \frac{6}{6}$$

For $n = n+1$...
$$\sum_{j=1}^{n}(j^2) + (n + 1)^2 = \frac{2(n+1)^3 + 3(n+1)^2 + (n+1)}{6}$$
Assuming that $\sum_{j=1}^{n}(j^2) = \frac{2n^3 + 3n^2 + n}{6}$...
$$\frac{2n^3 + 3n^2 + n}{6} + (n + 1)^2 = \frac{2(n+1)^3 + 3(n+1)^2 + (n+1)}{6}$$
$$2n^3 + 9n^2 + 13n + 6 = 2(n^3 + 3n^2 + 3n + 1) + 3(n^2 + 2n + 1) + n + 1$$
$$2n^3 + 9n^2 + 13n + 6 = 2n^3 + 9n^2 + 13n + 6$$\\ 
$$\sum_{j=1}^{n}j^2 =  \frac{2n^3 + 3n^2 + n}{6}$$
\hfill\blacksquare

\section*{Question 4}

\paragraph{(a)}
Let there be another multiplicative identity, y, such that $y * x = x = x * y$. Multiplying the equality by the inverse of x ($x^{-1}$), we are left with $y = 1 = y$. 1 was the given unique multiplicative identity, so therefore the multiplicative identity is unique. 

\paragraph{(b)}
Similar to above, let there be another additive identity, y, such that $y + x = x = x + y$. Adding -x to the equality, we are left with $y = 1 = y$. 1 was the given unique additive identity, so therefore the additive identity is unique. 

\paragraph{(c)} 
Adding $-x$ to $x + y = 0$, we get $y = -x$. Let there be another additive inverse, z, so that $x + z = 0$. Adding $-x$ to the equality we get $z = -x$, however since $-x = y$, $z = y$, so the additive inverse is unique. 

\paragraph{(d)} 
Adding b to $a*b = 0$ we get $a*b + b = b$. Multiplying both sides by the inverse of b, we are left with $a + 1 = 1$. Because the additive identity is unique, $a = 0$. If the inverse of b doesn't exist, then $b = 0$, as the multiplicative inverse exists for all $0 \neq b \in \mathbb{F}$

\section*{Question 5}
\paragraph{(a)} 
Let $a, b \in \mathbb{Z}_n$ such that $a * b = n$ (When $\sqrt{n} \in \mathbb{Q}$, a = b = $\sqrt{n}$). $a \otimes b = 0$, however we showed in Question 4 d) that if a, b $\in \mathbb{F}$, and $a * b = 0$, then either a or b is 0. This is not the case by our definition of a and b, therefore $(\mathbb{Z}_n,\oplus,\otimes)$ is not a field when n is not prime.

\paragraph{(b)}
Let us look at the definition of a field. (i) is true because modulos are closed. (ii) is true because we may assume addition in $\mathBB{Z}$ is associative. (iii) is true because 0 exists in $\mathbb{Z}_p$. (iv) is true because the inverse of k is always (p - k). (v) is true because we may assume addition is commutative. (vi) is true because modulos are closed. (vii) is true because we may assume multiplication is associative. (viii) is true, as the multiplicative identity, 1, exists in $\math{Z}_p$-* (ix) is true by Bezout's lemma.\\
\textbf{Bezout's lemma} \textit{Let a and b be non-zero integers, and let d be their greatest common divisor. Then there exists such integers such that ax + by = d}
$$k * k^-1 \equiv 1 (mod p)$$
let u be some integer such that
$$k * k^-1 = 1 + up$$
$$k * k^-1 - up = 1$$
A form of Bezout's lemma. (x) is true because we can assume multiplication is commutative. (xi) is also true by the following. $x_p \otimes (y_p \oplus z_p) = (x(y+z))_p = (xy + xz)_p = (xy)_p \oplus (xz)_p = (x_p \otimes y_p) \oplus (x_p \otimes z_p)$ Therefore, $(\mathbb{Z}_p,\oplus,\otimes)$ is a field when p is prime.

\section*{Question 6}

\paragraph{(a)} Multiplying $b_N$ by (1-p), $$(1-p)b_N = (1-p) + (p - p^2) + ... + (p^N - p^{n+1})$$the expression simplifies to $$b_N = \frac{1-p^{N+1}}{1-p}$$ Taking the limit as N approaches infinity, we have $b_N = \frac {1}{1-p}$ This is obviously bounded above, as it is a constant. 

\paragraph{(b)} Letting p = 1/2 (from part a), we have 
$$ \frac{1}{1-1/2} = 1 + \frac{1}{2} + \frac{1}{4} + ...$$ 
adding one to both sides... 
$$3 = 1 + 1 + \frac{1}{2} + \frac{1}{4} + ...$$
Comparing this to $a_N$...
$$a_N = \sum_{n=0}^{N} \frac{1}{n!} = 1 + \frac{1}{1!} + \frac{1}{2*3} + \frac{1}{2*3*4} + ... + \frac{1}{n!}$$ 
When p = $\frac{1}{2}$, $(b_N + 1)$'s first three terms are equal to $a_N$'s first three terms, however $b_N$'s subsequent terms ($\frac{1}{4}, \frac{1}{8}, \frac{1}{16}$, etc.), all the way to infinity, are greater than $a_N$'s subsequent terms ($\frac{1}{6} \frac{1}{24} \frac{1}{120}$). Since $b_N$ = 3, and $b_N > a_N$, $3 > a_N \forall \mathbb{N}$


\end{document}