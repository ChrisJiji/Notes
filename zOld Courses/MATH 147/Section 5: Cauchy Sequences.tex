\documentclass[10pt,letter]{article}
\usepackage{amsmath}
\usepackage{amssymb}
\usepackage{amsthm}
\usepackage{graphicx}
\usepackage{setspace}
\onehalfspacing
\usepackage{fullpage}
\newtheorem*{remark}{Remark}
\begin{document}
\title{Cauchy Sequences}
\paragraph{5.2: Definition}
Let $x_n$. We say that $x_n$ is a Cauchy sequence if for all $\epsilon > 0$ there exists $N\in\mathbb{N}$ such that $m > n \geq N$ implies $|x_n - x_m| < \epsilon$. 

\paragraph{5.5: Theorem}
Let $x_n$. The following are equivalent: \\ 
(a) $x_n$ is convergent, that is, there exists $\beta$ such that $lim x_n = \beta$.\\ 
(b) $x_n$ is a cauchy sequence. 

\paragraph{5.9: Definition}
A subset $E \subseteq \mathbb{R}$ is said to be sequentially compact if, given a sequence $x_n \in E^\mathbb{N}$, there exists a subsequence $x_n_k$ of $x_n$ which converges to some element $\beta \in \mathbb{R}$, and furthermore we must have $\beta \in E$


\end{document}