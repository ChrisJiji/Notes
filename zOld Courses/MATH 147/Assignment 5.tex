\documentclass[10pt,letter]{article}
\usepackage{amsmath}
\usepackage{amssymb}
\usepackage{amsthm}
\usepackage{graphicx}
\usepackage{setspace}
\onehalfspacing
\usepackage{fullpage}
\newtheorem*{remark}{Remark}
%\date{\vspace{-5ex}}
\begin{document}

\title{MATH147 Assignment 5}
\author{Chris Ji 20725415}

\maketitle 

\section*{Question 1}

\paragraph{(a)} Answer to Problem 1(a) here.

\paragraph{(b)} Answer to Problem 1(a) here.
\paragraph{(c)} Answer to Problem 1(a) here.
\paragraph{(d)} Answer to Problem 1(a) here.


\section*{Question 2}
Let there be a sequence $x_n \in A$ such that for some $n$, $x_n = x_{n+1}$. But then by the definition of a set (each element is distinct), this can't be true. Then $x_n$ is either greater than or less than $x_{n+1}$, thus making the sequence either strictly increasing or strictly decreasing. 

\section*{Question 3}
around 2.8

\section*{Question 4}
\paragraph{(a)} When $a = 1$, the proof is trivial. When $a \in (0, 1)$, for all $\epsilon > 0$, $1 - \sqrt[n]{a} < \epsilon$. By algebra, $(1 - \epsilon)^n < a$. As $a \in (0, 1)$, this is the same as $(1 - \epsilon)^n \leq 0$. As $n \rightarrow\infty$, this is clearly true. 
\\When $a \in (1, \infty)$, for all $\epsilon > 0$, $\sqrt[n]{a} - 1 < \epsilon$. By algebra, $a < (\epsilon + 1)^n$. As $n\rightarrow\infty$, this is a simple application of Archimedean principle, as $a \in \mathbb{N}$ and $(\epsilon + 1)^n \in \mathbb{N}$. 

\paragraph{(b)} $x_n$ converges to 7. 


\section*{Question 5}
\paragraph{(a)} 
\paragraph{(b)} Answer to Problem 1(b) here.


\section*{Question 6}
\paragraph{(a)} Answer to Problem 1(a) here.
\paragraph{(b)} Answer to Problem 1(b) here.
\paragraph{(c)} Answer to Problem 1(c) here.








\end{document}