\documentclass{article}     % Specifies the document class 

\title{MATH147 Section 0: Induction}  % Declares the document's title.
\author{Chris Ji}      % Declares the author's name.
\date{September 8, 2017}      % Deleting this command produces today's date.

\usepackage{hyperref}
\usepackage{amsfonts}
\hypersetup{
    colorlinks=true,
    linkcolor=blue,
    filecolor=magenta,      
    urlcolor=blue,
}

\newcommand{\ip}[2]{(#1, #2)}
                             % Defines \ip{arg1}{arg2} to mean
                             % (arg1, arg2).

\begin{document}             % End of preamble and beginning of text.

\maketitle     % Produces the title.

\section*{0.1 Purpose}      % Produces section heading.  Lower-level
                             % sections are begun with similar
                             % \subsection and \subsubsection commands.
The purpose of this course is two-fold. First, we shall review some of the mechanics of the calculus you have seen before- including differentiation, graphing curves, etc. The second component (and the most important) is that we will do everything in as much rigor and detail as possible; we will prove as much as we can, and to induce precise definitions of limits, continuity, derivatives, etc. 

\section*{0.2 Notation}
\href{https://learn.uwaterloo.ca/d2l/le/content/335487/viewContent/1934236/View?ou=335487}{Notation contained on the learn website}

\section*{0.3 The well ordering principle for $\mathBB{N}$}
If $\Phi \neq S \subset\mathBB{N}$, then S admits a minimum element (has a minimum element)
i.e. there exists some m_o $\in$ S so that m_o $\leq$ t for all t $\in$ S

\section*{0.4 The First Principle Of Mathematical Induction}
Let S $\subset$ $\mathBB{N}$ If:

\\
(a) 1 $\in$ S

\\
(b) k $\in$ S implies that (k+1) $\in$ S 
\\
Then S $\in$ $\mathBB{N}$
\\

\underline{Proof}
\\
We shall use \underline{proof by contradiction}
\\

\textbf
{\underline{NOTE} the idea behind a proof by contradiction is the following.
Suppose P is a true statement and that after applying correct logical reasoning, you end up with a statement Q. Then Q must also be true.
e.g. 5>4 if you add 3 to both sides, it should still be true
\\
\underline{We use this as follows}
Suppose P is a statement, but you don’t know if P is true of false. Suppose that after applying correct reasoning, and suppose that after applying correct reasoning, you end up with a statement Q that you can recognize is false. Therefore, P is false must have been false as well. 
\\
\underline{NOTE:} If Q is true- then this tells you nothing about whether P is true or false
e.g.
\\
P: 5=4
\\
Q: multiply both sides by 0 to get 0=0
\\
Q is true but P is false}

\breakline
\breakline
Let P be the statement S $\neq$ $\mathbb{N}$
\\
Let T = N\textbackslash S
\\
Then T $\neq$ $\Phi$ as S $\neq$ $\mathbb{N}$ and T $\subset$ $\mathbb{N}$
\\
By the well-ordering principle for $\math{N}$, T has a minimum element $m_o$  so that m_o $\leq$ t for all t $\in$ T. Since 1 $\in$ S by (a) and m_o $\in$ T = $\mathbb{N}$ \textbackslash S, we see that m_o $\neq$ 1. Thus m_o - 1 $\in$ $\math{N}$. Also m_o - 1 $<$ m_o and m_o $\leq$ t for all t$\in$T. This implies that m_o - 1 $\notin$ T. Hence m_o - 1 $\in$ S

Thus m_o - 1 $\in$ $\math{N}$. Also $m_o - 1 < m_o$ and $m_o \leq$ t for all t$\in$T. This implies that m_o - 1 $\notin$ T. Hence m_o - 1 $\in$ S
But then by (b) (m_o - 1) $\in$ S implies that (m_o = (m_o - 1) + 1) $\in$ S
But then m_o $\in$ S $\cup$ T = $\phi$, a contradiction
This shows that P was false, therefore S = $\math{N}$

\end{document}               % End of document.










