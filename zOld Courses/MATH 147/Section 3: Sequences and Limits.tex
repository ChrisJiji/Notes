\documentclass[10pt,letter]{article}
\usepackage{amsmath}
\usepackage{amssymb}
\usepackage{graphicx}
\usepackage{setspace}
\onehalfspacing
\usepackage{fullpage}
\begin{document}
\title{Sequences and limits}

\paragraph{3.2: Definition}
A sequence $(x_n)_{n=1}^\infty \in \mathbb{R}^\mathbb{N}$ is said to converge to a limit $\beta \in \mathbb{R}$ if for all $\epsilon > 0$ there exists $N \in \mathbb{N}$ such that $n > N$ implies $|x_n - \beta| < \epsilon$. 

\paragraph{3.5: Proposition}
If $lim_{n\rightarrow\infty}x_n = \alpha$, and $lim_{n\rightarrow\infty}y_n = \beta$, then $lim_{n\rightarrow\infty}(kx_n + y_n)$.\\ 
Proof: Let $\epsilon > 0$, and set $\epsilon_o = \frac{\epsilon}{|k|+1}$.\\ 
Since $lim_{n\rightarrow\infty}x_n = \alpha$, there exists $N_1 \in \mathbb{N}$ so that $n\geq N_1$ implies $|x_n - \alpha| < \epsilon_o$. \\ 
Since $lim_{n\rightarrow\infty}y_n = \beta$, there exists $N_2 \in \mathbb{N}$ so that $n\geq N_2$ implies $|y_n - \beta| < \epsilon_o$. \\ 
With N:= max($N_1, N_2$), we see that $n \geq N$ implies that both $|x_n - \alpha| < \epsilon_o$, and $|y_n - \beta| < \epsilon_o$. Thus $n \geq N$ implies that \\ 
$|(kx_n+y_n) - (k\alpha+\beta)| \leq |k||x_n - \alpha|+|y_n - \beta|$\\
$|(kx_n+y_n) - (k\alpha+\beta)| \leq |k|\epsilon_o +\epsilon_o$\\
$|(kx_n+y_n) - (k\alpha+\beta)| \leq \epsilon_o(|k|+1)$\\ 
We wanted $|(kx_n+y_n)-(k\alpha+\beta)| < \epsilon$. If we know that $(|k|+1)\epsilon_o \leq \epsilon$, we would be done. Setting $e_o = \frac{\epsilon}{|k|+1}$ works. 


\paragraph{3.7: Proposition}
If $(x_n)_{n=1}^\infty \in \mathbb{R}^\mathbb{N}$ is a convergent sequence, then $(x_n)_{n=1}^\infty$ is bounded.\\ 
Proof: Suppose that $lim_{n\rightarrow\infty}x_n = \beta \in \mathbb{R}$. Let $\epsilon = 1$, and choose $N \in \mathbb{N}$ so that $n \geq N$ implies that $|x_n - \beta| < \epsilon = 1$. Then $n \geq N$ implies that $|x_n| \leq |x_n - \beta| + |beta| < 1 + |beta|$. Let M:= max$\{|x_1|, |x_2|, ... ,|x_{N-1}|, 1 + |\beta|\}$. For any $n\geq 1$, $|x_n| \leq M$, and so $(x_n)_{n=1}^\infty$ is bounded. 


\paragraph{3.10: Multiplying and Dividing Limits}
If $lim_{n\rightarrow\infty}x_n = \alpha$, and $lim_{n\rightarrow\infty}y_n = \beta$, then $lim_{n\rightarrow\infty}(x_ny_n) = \alpha\beta$.\\
Proof: Let $\epsilon > 0$. and set $\epsilon_o = \frac{\epsilon}{M_2 +|\alpha| + 1}$. Because $x_n$ and $y_n$ are convergent, they must be bounded. Say $|x_n| \leq M_1$ for all $n \geq 1$, and $|y_n| \leq M_2$ for all $n \geq 1$. We want to prove that there exists $N \in \mathbb{N}$ so that $n \geq N$ implies $|x_ny_n - \alpha\beta| < \epsilon$.\\ 
We know: there exists $N_1 \in \mathbb{N}$ such that $n \geq N_1$ implies $|x_n - \alpha| < \epsilon_o$ and there exists $N_2 \in \mathbb{N}$ such that $n \geq N_2$ implies $|y_n - \beta| < \epsilon_o$. So if N:= max($N_1,N_2$) and $n \geq N$, then $|x_n - \alpha| < \epsilon_o$ and $|y_n - \beta| < \epsilon_o$. That is, $n \geq N$ implies that \\ 
$|x_ny_n - \alpha\beta \leq |x_ny_n - \alpha y_n| + |\alpha y_n - \alpha\beta|$ \\ 
$|x_ny_n - \alpha\beta \leq |x_n - \alpha||y_n| + |\alpha||y_n - \beta|$ \\ 
$|x_ny_n - \alpha\beta \leq |x_n - \alpha|M_2 + |\alpha||y_n - \beta|$\\ 
$|x_ny_n - \alpha\beta \leq M_2\epsilon_o + |\alpha|\epsilon_o$ \\ 
$|x_ny_n - \alpha\beta \leq \epsilon_o(M_2 + |\alpha|)$ \\ 

If $lim_{n\rightarrow\infty}x_n = \alpha$, and $lim_{n\rightarrow\infty}y_n = \beta$, then $lim_{n\rightarrow\infty}(x_n/y_n) = \alpha/\beta$.

\paragraph{3.13: examples of subsequences}
Let ($x_n$)_{n=1}^{\infty} $\in \mathbb{R}^\mathbb{N}$\\

(a) ERASING ALL THE ODD TERMS\\
let n_k = 2k k $\in \mathbb{N}$, Then (x_{n_k}) = (x_{2_k})_{k=1}^{\infty} = (x_2, x_4, x_6)

(b) let n_k = k^2, k $\in \mathbb{N}$ Then (x_1, x_4, x_9, etc.)

(c) consider the sequence $x_2$, $x_3$, $x_4, x_6, x_5, x_7, x_8,$ etc. This is not a sequence of ($x_n$) since n_4 = 6, n_5=5 and so n_4 $\not<$ n_5

(d) consider ($x_1, x_2,..., x_10$)\\
This is not a sequence, so it is not a subsequence of ($x_n$)

\paragraph{3.14}
Proposition: Suppose that (x_n) $\in \mathbb{R}$ and the limit of $x_N = \beta$\\
If $x_{n_k}$ is a subsequence of ($x_n$) then lim as k approaches $\infty = \beta$\\

PROOF: $n_1$ is at least 1, and $n_2$ has to be bigger than $n_1$, so $n_2$ has to be at least 1. \\ 
Let $\epsilon > 0$\\
We want: there exists N $\in \mathbb{N}$ such that k = N implies $|x_{n_k} - \beta| < \epsilon$\\
we know: there exists $N \in \mathbb{N}$ such that N=$\mathbb{N}$ implies $|x_{n_k} - \beta| < \epsilon$\\
But $k<n_1<n_2<...$ implies that $n_k \geq k$ for all k $\in \mathbb{N}$. So if $k \geq N_o$, then $n_k \geq k \geq N_o$ and hence $|x_{n_k} - \beta| < \epsilon$. By definition, the $lim_{k\rightarrow\infty} (x_{n_k}) = \beta$

\paragraph{3.15 - example}
Let x_n = (1, 2, 1, 2, 1, 2, ...)\\
Then (x_2k) = (2, 2, 2, 2, ...)\\
and so $lim_{k\rightarrow\infty}(x_2k) = 2$\\
and (x_{2k-1}) = (1, 1, 1, 1, ...)\\
and so $lim_{k\rightarrow\infty}(x_{2k-1} = 1)$\\
Since 1 $\neq$ 2, by Proposition 3.14, $(x_n)_{n=1}^\infty$ diverges.\\

\paragraph{3.16 - squeeze theorem}
(a)
let $x_n$, $y_n$, and $z_n \in \mathbb{R}^\mathbb{N}$\\
Suppose that $lim_{n\rightarrow\infty} (x_n) = \beta = lim_{n\rightarrow\infty} (z_n)$ \\
Suppose that there exists $N_o$ such that $n \geq N$ implies $x_n \leq y_n \leq z_n$.\\
Then $lim_{n\rightarrow\infty} (y_n) = \beta$\\
Proof. \\
Let $\epsilon > 0$\\
We want: there exists some N $\in \mathbb{N}$ such that n $\geq N$ implies $|y_n - \beta| < \epsilon$\\
We know: there exists $N_1 \in \mathbb{N}$ such that n $\geq N_1$, implies $|x_n - \beta| < \epsil on$. AKA $\beta - \epsilzn < x_n < \beta + \epsilon$\\ 
there exists there exists $N_2 \in \mathbb{N}$ such that n $\geq N_2$, implies $|z_n - \beta| < \epsilon$. AKA $\beta - \epsilon < z_n < \beta + \epsilon$\\ 
And there exists $N_o \in \mathbb{N}$ such that n $\geq N_o$ implies $x_n \leq y_n \leq z_n$.\\
Let N = max($N_o$, $N_1$, $N_2$)\\
Then n $\geq$ N implies: \\
$\beta - \epsilon < x_n \leq y_n \leq z_n < \beta + \epsilon$\\
that is, n $\geq$ N implies $|y_n - \beta| < \epsilon$, so by definition $lim_{n\rightarrow\infty}(y_n) = \beta$\\
\paragraph{3.17 - example}
let $x_n = \frac{cos(sn^2(e^{n^2+\pi}+log(n^4+3n+6)))}{n}$, $n \in \mathbb{N}$\\
Consider the fact that $-1 \leq cosx \leq 1$ for all x $\in \mathbb{R}$\\
Thus, $\frac{-1}{n} \leq x_n \leq \frac{1}{n}$ for all $n \in \mathbb{N}$\\
But the limits of $\frac{-1}{n}$ and $\frac{1}{n}$ = 0, so by the squeeze theorem, $lim_{n\rightarrow\infty}(x_n) = 0$.

(b) Suppose $(x_n)_{n=1}^\infty \in \mathbb{R}^\mathbb{N}$ such that the limit of $x_n$ = 0 and that ($z_n$) is bounded.\\
Claim: $lim_{n\rightarrow\infty}(x_nz_n) = 0$ \\ 
PROOF: let $\epsilon > 0$\\
We want: there exists $N \in \mathbb{N}$ such that n$\geq$ N implies $|x_n - z_n - 0| < \epsilon$.\\
We want: there exists N$\in \mathbb{N}$ such that n $\geq N$ implies $|x_n - 0| = |x_n| < \epsilon$ and there exists $0 < \mu \in \mathbb{R}$ such that $-\mu \leq z_n \leq \mu$ for all $n \in \mathbb{N}$\\
Thus $-\mu|x_n| \leq x_nz_n \leq \mu|x_n|$ for all $n \in \mathbb{N}$\\
Now because $lim_{n\rightarrow\infty}(x_n) = 0$ implies that $lim_{n\rightarrow\infty}(|x_n|) = 0$\\
(showing that $lim_{n\rightarrow\infty}(|x_n|) = 0$):\\
$n \geq N_o$ implies $|x_n - 0| < \epsilon_o$ \\
But then $n \geq N_o$ implies $||x_n| - 0| = ||x_n|| = |x_n| = |x_n - 0| < \epsilon_o$, proving that the limit of $|x_n| = 0$


But then the limit of $-\mu|x_n$ = $(-\mu)lim(|x_n|)$\\
= ($-mu$)0
= 0
and similarly the limit of $-\mu|x_n$ = 0\\
By the squeeze theorem, $lim_{n\rightarrow\infty}(x_nz_n) = 0$

\paragraph{3.18}
We now add \underline{symbols} (not numbers) to $\mathbb{R}$ to obtain the extended real numbers: \\
We add +$\infty$ and define $\infty + x = \infty = x + \infty$\\
and -$\infty$ and define $-\infty + x = -\infty = x + (-\infty)$ for all x $\in \mathbb{R}$\\
Note that $\infty - \infty$ is \underline{NOT} defined. \\ 
A sequence diverges to $\infty$ if for all $\mu \in \mathbb{R}$ there exists $N \in \mathbb{B}$ such that n$\geq N$ implies that $x_n \geq \mu$. 

\paragraph{3.19 : examples}
\subparagraph{a} suppose $(x_n)_{n=1}^\infty$ = (1, -1, 1, -2, 1, -3)\\ 
Then $(x_n)_{n=1}^\infty$ diverges because it is \underline{not} bounded. \\
Note: $lim_{n\rightarrow\infty} x_n \neq \infty$, since for $\mu$ = 2, there does not exist $N \in \mathbb{N}$ such that $n \geq N$ implies $x_n \geq \mu = 2$. Also, $lim_{n\rightarrow -\infty}x_n \neq -\infty$, since $\mu$ can = 0.\\
\subparagraph{b} Suppose $(x_n)_{n=1}^\infty = (-1, -4, -9, -16, -n^2)$, then $lim_{n\rightarrow\infty} x_n = -\infty$ given $\mu \in \mathbb{R}$. Choose $N \in \mathbb{N}$ such that $N > \sqrt{|\mu|}$, then $n \geq N$ implies $n^2 \geq N^2 \geq |\mu|$, so $-n^2 \leq |\mu| \leq \mu$.\\
\subparagraph{c} Suppose that $x_n and y_n \in \mathbb{R}^\mathbb{N}$\\ Suppose that $x_n$ is bounded, and  $y_n$ diverges. Then $lim_{n\rightarrow\infty} x_n + y_n = \infty$.\\ 
Proof: let $\mu \in \mathbb{R}$. We want $N \in \mathbb{N}$ such that n $\geq N$ implies $x_n + y_n \geq \mu$. We know there exists $M > 0$ such that $-M \leq x_n \leq M$ for all n. There exists $N_o \in \mathbb{N} such that n \geq N_o$ implies $y_n \geq \mu$. \\ 
Since $x_n$ is bounded, we can find $M>0$ such that $|x_n| \leq M$ for all $n \geq 1$. Since $y_n$ diverges to infinity, we know $\exists N \in \mathbb{N}$ such that $n \geq N$ implies $y_n \geq \mu + M$. But then $n \geq N$ implies $x_n + y_n \geq y_n - |x_n|$. \\ 
$x_n + y_n \geq (\mu + M) - M = \mu$. By definition, $lim_{n\rightarrow\infty} x_n + y_n = \infty$\\

\paragraph{3.21 The comparison theorem}
Suppose $(x_n)_{n=1}^\infty$ and $(y_n)_{n=1}^\infty$ $\in \mathbb{R}^\mathbb{N}$ $\ni$ $lim x_n = \alpha$ and $lim y_n = \beta$. Suppose also that $\exists N_o \in \mathbb{N} \ni n \geq N_o$ implies $x_n \leq y_n$. Then $\alpha \leq \beta$. \\ 
Proof: we argue by contradiction. Suppose $\beta < \alpha$. Let $\epsilon = \frac{\alpha - \beta}{3} > 0$.\\ 
Since $lim x_n = \alpha$ , we can find $N_1 \ni$ $n \geq N$  implies $|x_n - \alpha| < \epsilon$, ie $\alpha - \epsilon < x_n < \alpha + \epsilon$. \\
Since $lim y_n = \beta$, we can find $N_2 \ni$ $n \geq N_2$ implies $|y_n - \beta| < \epsilon$, ie $\beta - \epsilon < y_n < \beta + \epsilon$. \\
Let N= math ($N_o$, $N_1$, $N_2$). Then $n \geq N$ implies: $\beta - \epsilon < y_n < \beta + \epsilon < \alpha - \epsilon < x_n < \alpha + \epsilon$. This contradicts the hyptothesis that $n \geq N_o$ implies $x_n \leq y_n$. 

\paragraph{3.22: examples}
\subparagraph{(a)}
Suppose that we are given the sequence $(x_n)_{n=1}^\infty \in \mathbb{R}$ and $lim_{n\rightarrow\infty} x_n = \infty$. Then $lim\frac{1}{x_n} = 0$. \\ 
Proof: let $\epsilon > 0$. We want $0 < \frac{1}{x_n} < \epsilon$. We know for large N, $x_n \geq \mu$. 
Since $lim x_n = \infty$, letting $\mu = \frac{2}{\epsilon}$, we can find N such that after $n \geq N$ implies $x_n \geq \mu > 0$, in particular $x_n \neq 0$, so $\frac{1}{x_n}$ exists for all $n \geq N$. also, $x \geq N$ $x_n \geq \frac{2}{\epsilon}$ implies $0 < \frac{1}{x_n} \leq \frac{\epsilon}{2} < \epsilon$. By definition, $lim \frac{1}{x_n} = 0$.

\subparagraph{(b)}
let $x_n = \frac{2n}{n^2 + \pi}$, $n \geq 1$. Then $lim x_n = 0$. We will use the squeeze theorem. Let $y_n = \frac{1}{n} = \frac{2n}{n^2 + n^2} < \frac{2n}{n^2 + \pi} = x_n$ for all $n \geq 2$.\\ 
let $z_n = \frac{2}{n}$. Thus, $n \geq 2$ implies $y_n \leq x_n \leq z_n$. Now $lim y_n = lim \frac{1}{n} = 0 = 2 * 0 = 2lim \frac{1}{n} = lim \frac{2}{n} = lim z_n$. Hence by the squeeze theorem, $lim x_n = 0$. 


\end{document}