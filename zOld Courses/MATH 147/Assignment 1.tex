\documentclass{beamer}     % Specifies the document class 

\title{MATH147 Assignment One}  % Declares the document's title.
\author{Chris Ji 20725415}      % Declares the author's name.

\usepackage{gensymb}            % allows the \degree character to be used
\usepackage[parfill]{parskip}   % prevents random indents

\begin{document}  
\maketitle

\section*{Question 1}
\textbf{(a)}\\
If all the men are innocent, then anyone who is accusing another man must be lying.
In that case, Ed is lying because he is accusing Ted, and Ted is lying because he is accusing one of the other men.
Fred is not accusing anyone, only saying \textit{if} one of the others is guilty, then the other one is aswell. Therefore, Ed and Ted are lying.\\

\textbf{(b)}\\
Let us go through the men one at a time, and assess their statements. Ed accuses Fred, and states that Ted is innocent.
Fred states that if Ed is guilty, so is Ted. Ted states one of the others is guilty. Because Ed is telling the truth, that Fred is guilty, that satisfies Ted's statement. Fred's statement accuses no one, as Ed is not guilty. Therefore, Fred is the only guilty one.\\

\textbf{(c)}\\
Let us assume one by one each man is innocent, and therefore truthful. If Ed is innocent, then Fred is guilty and Ted is innocent. Fred's statement is invalid, then, because he is lying in this case. Ted would then be innocent, for the same reasons as above. If Fred were innocent, his statement gives no information, but Fred's innocence proves Ed's guiltiness. By Fred and Ed's statements, Ted would also be guilty. However, since the guilty lie, this situation can not happen as Ted would be telling the truth in saying at least one of the others is guilty. Ted's innocence is grouped with Ed's innocence, so therefore Fred is guilty if the innocent are truthful and the guilty lie.

\section*{Question 2}
$|3sin$\Theta$ + 4cos$\Theta$$| \leq 5\\
3\sqrt{1-cos^2$\Theta$} + 4cos$\Theta$ \leq 5\\
(3\sqrt{1-cos^2$\Theta$})^2  \leq (5 - 4cos$\Theta$)^2\\
9 - 9cos^2$\Theta$ \leq 25 - 40cos$\Theta$ + 25cos^2$\Theta$\\
0 \leq 16 - 40cos$\Theta$ + 25cos^2$\Theta$\\
0 \leq (5cos$\Theta$ - 4)^2\\

Any number squared will be greater than or equal to 0.
Therefore equality holds when cos$\Theta$ = 4/5\\
Equality holds when $\Theta$ \doteq 36.87\degree

\section*{Question 3}
\log_{a}(bc)\log_{b}(ac)\log_{c}(ab) = \log_{a}(bc) + \log_{b}(ac) + \log_{c}(ab) + 2\\\\
(\frac{\log b + \log c}{\log a})(\frac{\log a + \log c}{\log b})(\frac{\log a + \log b}{\log c})=(\frac{\log b + \log c}{\log a})+(\frac{\log a + \log c}{\log b})+(\frac{\log a + \log b}{\log c})+2\\\\
Let A =  $\log$ a, B = $\log$ b, and C = $\log$ c\\
The above line is equal to \\
\begin{equation}
    (\frac{B+C}{A})(\frac{A+C}{B})(\frac{A+B}{C}) = (\frac{B+C}{A}) + (\frac{A+C}{B}) + (\frac{A+B}{C}) + 2
\end{equation}

Multiplying out the left side of (1) achieves\\
$\frac{A^2B + A^2C + B^2C + AB^2 + AC^2 + BC^2 + 2ABC}{ABC}$

Multiplying for a common denominator, then adding up the right side of (1)\\
$\frac{A^2B + A^2C + B^2C + AB^2 + AC^2 + BC^2}{ABC}$ + $\frac{2ABC}{ABC}$\\\\
$\frac{A^2B + A^2C + B^2C + AB^2 + AC^2 + BC^2 + 2ABC}{ABC}$ = $\frac{A^2B + A^2C + B^2C + AB^2 + AC^2 + BC^2 + 2ABC}{ABC}$

\section*{Question 4}
\textbf{(a)}\\
Left side of the inequality: \\
$\sqrt{a^2+h}$ - a $<$ $\frac{h}{2a}$\\
($\sqrt{a^2+h}$)^2  <  (\frac{h}{2a} + a)^2\\
a^2 + h < \frac{h^2}{4a^2} + h + a^2\\
0 < (\frac{h}{2a})^2\\

Right side of the inequality: \\
$\frac{h}{2a}$ $<$ a - \sqrt{a^2 - h}\\
($\sqrt{a^2 - h}$)^2 < (a - \frac{h}{2a})^2\\
a^2 - h < a^2 - h + \frac{h^2}{4a^2}\\
0 < (\frac{h}{2a})^2\\

\textbf{(b)}\\
$x^3 - y^3 = (x-y)(x^2 + xy + y^2)$\\
Left side of the inequality: \\
$\sqrt[3]{a^3 + h}$ - $a$ $<$ $\frac{h}{3a^2}$\hfill multiply both sides by $((\sqrt[3]{a^3 + h})^2 + a\sqrt[3]{a^3 + h} + a^2)$\\
Using the difference of cubes, the left side simplifies to $a^3 + h - a^3$, which simplifies to h. Then, let $\sqrt[3]{a^3+h}$ be x.\\
h $<$ $\frac{h(x^2 + xa + a^2)}{3a^2}$ \\
$3a^2 < x^2 + xa + a^2$\\
$2a^2 < x^2 + xa$\\
Because $x > a$ ($\sqrt[3]{a^3 + h} > a $), both $x^2$ and $xa$ are greater than $a^2$. Thus, $x^2 + xa > 2a^2$.

Right side of the inequality: \\ 
$\frac{h}{3a^2}$ $<$ $a$ - $\sqrt[3]{a^3 - h}$\hfill multiply both sides by $(a^2 + a\sqrt[3]{a^3 + h} + (\sqrt[3]{a^3 + h})^2)$\\
Using the difference of cubes, the right side simplifies to $a^3 - a^3 + h$, which simplifies to h. Then, let $\sqrt[3]{a^3-h}$ be x.\\
$\frac{h(x^2 + xa + a^2)}{3a^2} < h$\\
$x^2 + xa + a^2 < 3a^2$\\
$x^2 + xa < 2a^2$\\
Because $x < a$ ($\sqrt[3]{a^3 - h} > a $), both $x^2$ and $xa$ are less than $a^2$. Thus, $x^2 + xa < 2a^2$.

\textbf{(c)}\\
$||83 - \sqrt{6891}| - |9 - \sqrt[3]{726}||$
For the left absolute value, a = 83 and h = 2. The root of any square number + a positive number will always be greater than the original number, so the first absolute value should be reversed, to $\sqrt{6891} - 83$\\
For the right absolute value, a = 9 and h = 3. The cube root of any cube - a positive number will always be less than the original number, so the second absolute value is in the right order.\\
For the outside absolute values, we need to compare the middle term for both expressions (the $\frac{h}{2a}$ and the $\frac{h}{3a^2}$, respectively). \\
$\frac{2}{2*83} < \frac{3}{3*9^2}$
Therefore, the equation without absolute values is:\\
$(9-\sqrt[3]{726})-(\sqrt{6891} - 83)$

\section*{Question 5}
\textbf{(a)}\\
$\frac{a+b}{2}$ $\geq$ $(ab)^{1/2}$\\
$\frac{a^2 + 2ab + b^2}{4}$ $\geq$ ab\\
a^2 + 2ab + b^2 $\geq$ 4ab\\
a^2 - 2ab + b^2 $\geq$ 0\\
(a-b)^2 $\geq$ 0\\

\textbf{(b)}\\
For n=1:\\
$\frac{a_1 + a_2}{2}$ $\geq$ ($a$_1a_2)$^{1/2}$\\
If you let $a_1$ = a, and $a_2$ = b, the proof is the same as above. \\

For n=n+1:\\
$\frac{a_1 + ... + a_{2^{n+1}}}{2^{n+1}} \geq (a_1...a_{2^{n+1}})^{1/2^{n+1}}$\\
$\frac{\frac{a_1 + ... + a_{2^{n+1}}} {2^n}} {2} \geq \sqrt{(a_1...a_{2^{n+1}})^{1/2^n}}$\\
$\frac{\frac{a_1+...+a_{2^n}}{2^n} + \frac{a_{2^n+1}+...+a_{2^{n+1}}}{2^n}}{2}$ \geq \sqrt{(a_1...a_{2^n})^{1/2^n}(a_{2^n+1}...a_{2^{n+1}})^{1/2^n}}$\\
By the assumption that $\frac{a_1+...+a_{2^n}}{2^n} \geq \sqrt{(a_1...a_{2^n})^{1/2^n}$},\\
$\frac{\frac{a_1+...+a_{2^n}}{2^n} + \frac{a_{2^n+1}+...+a_{2^{n+1}}}{2^n}}{2}$ \geq \sqrt{\frac{a_1+...+a_{2^n}}{2^n}\frac{a_{2^n+1}+...+a_{2^{n+1}}}{2^n}}$\\
let x = $\frac{a_1+...+a_{2^n}}{2^n}$ and y = $\frac{a_{2^n+1}+...+a_{2^{n+1}}}{2^n}$\\
$\frac{x+y}{2} \geq \sqrt{xy}$\\
$\frac{(x+y)^2}{4} \geq xy$\\
$x^2 - 2xy + y^2 \geq 0$\\
$(x-y)^2 \geq 0$\\

\textbf{(c)}\\
$\frac{a_1+...+a_m}{2^n}\frac{a_{m+1}+...+a_{2^n}}{2^n} \geq (a_1...a_ma_{m+1}...a_{2^n})^{1/2^n}$\\
$\frac{a_1+...+a_m}{2^n} \frac{\frac{2^n-m}{m}(a_1+...+a_m)}{2^n} \geq (a_1...a_m)^{1/2^n}(a_1...a_{m})^{\frac{2^n-m}{2^n}}$\\
$\frac{(a_1+...+a_m)(\frac{2^n-m}{m}+1)}{2^n} \geq (a_1...a_m)^{1/2^n}(a_1...a_{m})^{\frac{2^n-m}{2^n}}$\\
$(\frac{a_1+...+a_m}{m})^{2^n} \geq (a_1...a_m)(a_1...a_{m})^{2^n-m}$\\
$(\frac{a_1+...+a_m}{m})^{2^n} \geq ({a_1...a_m})^{2^n/m}$\\
$\frac{a_1+...+a_m}{m} \geq \sqrt[m]{a_1...a_m}$

\section*{Question 6}
Left side of the inequality: \\
2($\sqrt{k+1}$ - $\sqrt{k}$) $<$ $\frac{1}{\sqrt{k}}$\\
$\sqrt{k+1}$ $<$ $\frac{1}{2\sqrt{k}}$ + $\frac{2k}{2\sqrt{k}}$\\
k+1 $<$ $\frac{(2k+1)^2}{4k}$\\
4k^2 + 4k < 4k^2 + 4k + 1\\
0 < 1\\

Right side of the inequality: \\
$\frac{1}{\sqrt{k}}$ $<$ 2(\sqrt{k} - \sqrt{k-1})\\
$\sqrt{k-1}$ $<$ $\frac{2k}{2\sqrt{k}}$ - $\frac{1}{2\sqrt{k}}$\\
k-1 $<$ $\frac{4k^2 -4k + 1}{4k}$\\
0 $<$ 1\\

If 2($\sqrt{k+1}$ - $\sqrt{k}$) $<$ $\frac{1}{\sqrt{k}}$ $<$ 2(\sqrt{k} - $\sqrt{k-1}$)\\
Then $\frac{1}{\sqrt{n}}$$\sum_{k=1}^{n}$2($\sqrt{k+1}$ - $\sqrt{k}$) $<$ $\frac{1}{\sqrt{n}}$$\sum_{k=1}^{n}$$\frac{1}{\sqrt{k}}$ $<$ $\frac{1}{\sqrt{n}}$$\sum_{k=1}^{n}$2(\sqrt{k} - \sqrt{k-1})\\

Let us simplify the left and right equalities, as the middle term is the sum we want to estimate. \\Consider the left most inequality for n=3. \\
$\frac{2(\sqrt{2} - \sqrt{1}) + 2(\sqrt{3} - \sqrt{2}) + 2(\sqrt{4} - \sqrt{3})}{\sqrt{3}}$\\
As it turns out, for all values of n, the only terms that don't simplify are $\sqrt{n+1}$ and $\sqrt{1}$, which is equal to 1, so the lower bound is $\frac{2(\sqrt{n+1}-1)}{\sqrt{n}}$.\\
Consider the right inequality for n=3.\\
$\frac{2(\sqrt{1} - \sqrt{0}) + 2(\sqrt{2} - \sqrt{1}) + 2(\sqrt{3} - \sqrt{2})}{\sqrt{3}}$\\
This equation simplifies to 2 for all values of n. So the upper bound is 2.

Hence the estimation for $\frac{1}{\sqrt{n}}$$\sum_{k=1}^{n}$$\frac{1}{\sqrt{k}}$ is:\\
$\frac{2(\sqrt{n+1}-1)}{\sqrt{n}}$ $<$ $\frac{1}{\sqrt{n}}$$\sum_{k=1}^{n}$$\frac{1}{\sqrt{k}}$ $<$ 2

\end{document}