\documentclass[10pt,letter]{article}
\usepackage{amsmath}
\usepackage{amssymb}
\usepackage{amsthm}
\usepackage{graphicx}
\usepackage{setspace}
\onehalfspacing
\usepackage{fullpage}
\newtheorem*{remark}{Remark}
\begin{document}

\section*{}

\paragraph{Matrices as Linear Transformations}
Notation: for a vector $x=\begin{bmatrix}x_1\\\vdots\\x_n\end{bmatrix}\subset\mathbb{F}^n$, write $[x]_j=x_j$. For example, for $y\in\mathbb{F}^n$, define the sum of $x$ and $y$ to be $[x+y]_j=[x]_j+[y]_j$. For $c\in\mathbb{F}$ (some scalar), define $cx\in\mathbb{F}^n$ by $[cx]_j=c[x]_j$. \\ 
From last lecture, for $A\in Mm,n(\mathbb{F})$ (A is an mxn matrix), $A(x+y)=Ax+Ay$, and $A(cx)=cAx$. This says that A is a linear transformation. 

\paragraph{Matrix Multiplication}
Suppose $A\in Mm,n(\mathbb{F})$ and $B\in Mn,p(\mathbb{F})$. Then $A:\mathbb{F}^n\rightarrow\mathbb{F}^m$ and $B:\mathbb{F}^p\rightarrow\mathbb{F}^n$. So we can consider the composition $A\circ B:\mathbb{F}^p\rightarrow\mathbb{F}^m$. $(A\circ B)(x)=A(B(x))$. For $A=[a_{ij}]$ and $B=[b_{ij}]$. \\ 
Note: for $x\in\mathbb{F}^n$, $[Ax]_i=[\sum_{j=1}^na_{ij}x_j]$. Similarly, for $x\in\mathbb{F}^p$, $[Bx]_i=[\sum_{k=1}^{p}b_{ik}x_k]$. \\ 
Let $y=Bx$. $[y]_j=[\sum_{k=1}^{p}b_{jk}x_{k}]$, so $[A(Bx)]_i=[Ay]_i=[\sum_{j=1}^{n}a_{ij}y_{j}]=[\sum_{j=1}^na_{ij}(\sum_{k=1}^pb_{jk}x_k)]=\sum_{j=1}^{n}\sum_{k=1}^pa_{ij}b_{jk}x_k$. Define $C\in Mm,p(\mathbb{F})$ by $C_{ik}=\sum_{j=1}^{n}a_{ij}b_{jk}$. Then for $x\in\mathbb{F}^p$, $[Cx]_i=\sum_{k=1}^pc_{ik}x_k=\sum_{k=1}^{p}\sum_{j=1}^{n}a_{ij}b_{jk}x_k=\sum_{j=1}^n\sum_{k=1}^pa_{ij}b_{jk}x_k=[(A\circ B)(x)]_i$. \\ 
Conclusion: For $A\in Mm,n(\mathbb{F})$, $B\in Mn,p(\mathbb{F})$, then the composition $A\circ B: \mathbb{F}^p\rightarrow\mathbb{F}^m$ agrees with the transformation give by the matrix $C\in Mn,p(\mathbb{F})$ given be $C=[c_{ik}]$ where $C_{ik}=\sum_{j=1}^{n}a_{ij}b_{jk}$

\paragraph{Definition}
For $A\in Mm,n$, $B\in Mn,p$, the \underline{matrix product} $AB=[c_{ik}]\in Mm,p$ by $c_{ik}=\sum_{j=1}^na_{ij}b_{jk}$. Note: $AB$ is only defined if the dimensions of $A$ and $B$ match up, that is, $A$ is mxn, and $B$ is nxp, for some values of $m,n,p$. We often write $AB=AxB=A\cdot B=A\circ B$. 

\paragraph{Examples}
$A = \begin{bmatrix} 1 & 2 & 1 \\ 0 & -1 & 3 \end{bmatrix}$ and $B = \begin{bmatrix} 1 & -1 \\ -1 & 2 \\ 3 & 0\end{bmatrix}$. The product AB makes sense, since $A$ is 2x3, and $B$ is 3x2. The resulting matrix is $\begin{bmatrix}(1*1)+(2*-1)+(3*1)=2 & (1*-1)+(2*2)+(1*0)=3 \\ (0*1)+(-1*-1)+(3*3)= 10 & (0*-1)+(-1*2)+(3*0)=-2 \end{bmatrix}$ \\ \\ 

$A = \begin{bmatrix} a & b  \\ c & d \end{bmatrix}$ and $B = \begin{bmatrix} x & y \\ z & w \end{bmatrix}$. $AB = \begin{bmatrix} ax+bz & ay + bw \\ cx+dz & cy+dw\end{bmatrix}$.  \\ \\ 

$A = \begin{bmatrix} a_{11} & \ldots & a_{1n} \\ &\vdots\\ a_{m1} & \ldots & a_{mn} \end{bmatrix}$ and $B = \begin{bmatrix} x_1 \\ \vdots \\ x_n \end{bmatrix}$. $AB = \begin{bmatrix} a_{11}x_1 & \ldots & a_{1n}x_n \\ &\vdots\\ a_{m1}x_1 & \ldots & a_{mn}x_n \end{bmatrix}$. 

\paragraph{Definition}
If A is a square matrix, $A\in Mn,n(\mathbb{F})$, we can take products $A\cdot A$, $A\cdot A\cdot A$, etc. Write $A^0=In, A^1=A, A^2=A\cdot A$, etc. Questions: Can you define $A^{\frac{1}{2}}$? When is there a matrix $B\in Mn,n(\mathbb{F})$ such that $B^2=A$? 



\end{document}