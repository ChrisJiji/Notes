\documentclass[10pt,english]{article}
\usepackage[T1]{fontenc}
\usepackage[latin9]{inputenc}
\usepackage{geometry}
\geometry{verbose,tmargin=1.5in,bmargin=1.5in,lmargin=1.5in,rmargin=1.5in}
\usepackage{amsthm}
\usepackage{amsmath}
\usepackage{amssymb}

\makeatletter
\usepackage{enumitem}
\newlength{\lyxlabelwidth}

\usepackage[T1]{fontenc}
\usepackage{ae,aecompl}

%\usepackage{txfonts}

\usepackage{microtype}

\usepackage{calc}
\usepackage{enumitem}
\setenumerate{leftmargin=!,labelindent=0pt,itemindent=0em,labelwidth=\widthof{\ref{last-item}}}

\makeatother

\usepackage{babel}
\begin{document}
\noindent \begin{center}
\textbf{\large{}MATH 146 - Assignment 1}\\
\textbf{\large{}Chris Ji 20725415}
\par\end{center}{\large \par}
\medskip{}

\begin{enumerate}
\item 
\begin{enumerate}

\item [a.] We will show that row equivalence is reflexive, symmetric, and transitive, and then, by definition, it will be an equivalence relation. 
\begin{enumerate}
    \item [i.] Any matrix is trivially row equivalent to itself, so row equivalence is reflexive. 
    \item [ii.] Let $\Gamma$ be the set of elementary row operations that show that 2 matrixes, $A$ and $B$ are row equivalent. For each elementary row operation in $\Gamma$, we can apply the inverses, in reverse order, to $B$ in order to get A, so row equivalence is symmetric. 
    \item [iii.] Let $A,B,C$ be $m\times n$ matrices such that $A$ is row equivalent to $B$, and $B$ is row equivalent to $C$. Let $\Gamma_1$ be the set of elementary row operations that transforms $A$ to $B$, and $\Gamma_2$ be the set of elementary row operations that transforms $B$ to $C$. By definition, $\Gamma_1$ is finite, and $\Gamma_2$ is finite, you can apply $\Gamma_1$  to $A$, and then $\Gamma_2$ to that product ($B$) to get $C$, hence $A$ is row equivalent to $C$, and row equivalence is transitive. 
\end{enumerate}
\item [b.] We will prove (i) and (ii) separately. \begin{enumerate}
    \item [i.] Suppose $[a]\cap[b]\neq\phi$. Then there exists an element, $c$, such that $c\in[a]$ and $c\in[b]$. By definition, $aRc$ and $bRc$. By symmetry, $cRb$, and so $aRb$. $aRb\Rightarrow bRa$, so $[a]=[b]$. 
    \item [ii.] Since R is reflexive, $aRa$ for any $a\in X$, so $a\in [a]$ for any $a\in X$. Therefore, $X=\cup_{a\in X}[a]$
\end{enumerate}
\item [c.] $1R2$ and $2R3$ imply $1R3$, so $1,2,3$ must be in the same equivalence class. Therefore, the two possible definitions of $R$ are $[1,2,3]$ and $[4]$, and $[1,2,3,4]$. 
\end{enumerate}
\pagebreak
\item \begin{enumerate}
    \item [a.] Setting $[a]\equiv x \text{ mod }n \Rightarrow[a]=x+nk_1$ and $[b]\equiv y\text{ mod }n \Rightarrow [b]=y+nk_2$, where $k_1,k_2\in\mathbb{Z}$. $[a]+[b]=x+nk_1+y+nk_2=x+y+n(k_1+k_2)=x+y\text{ mod }n$. $[a]\cdot[b]=(x+nk_1)(y+nk_2)=xy + xnk_2+ynk_1+n^2k_1k_2=xy+n(xk_2+yk_1+nk_1k_2)=xy \text{ mod }n$. 
    \textbf{Lemma. }\textit{Let $a,b\in\mathbb{F}$. If $a\cdot b =0$, then either $a=0$ or $b=0$.}\\ 
Proof of lemma:     Adding $b$ to $a\cdot b = 0$ we get $a\cdot b + b = b$. Multiplying both sides by $b^{-1}$, we are left with $a + 1 = 1$. Because the additive identity is unique, $a = 0$. If $b^{-1}$ doesn't exist, then $b = 0$, as the multiplicative inverse exists for all $0 \neq b \in \mathbb{F}$.  \\ 
Let $a, b \in \mathbb{Z}_n$ such that $a * b = n$ (When $\sqrt{n} \in \mathbb{Q}$, a = b = $\sqrt{n}$). $a \cdot b = 0$, however by the lemma, if $a \cdot b = 0$, then either a or b is 0. This is not the case by our definition of a and b, therefore $\mathbb{Z}_n$ is not a field when n is not prime.


    \item [b.] $1\in\mathbb{Q}[\sqrt{2}]$, as (in our definition), setting $b=0$ and $a=1$, $1+0\sqrt{2}=1\in\mathbb{Q}[\sqrt{2}]$. Let $q_1=a_1+\sqrt{2}b_1$, and $q_2=a_2+\sqrt{2}b_2$, where $a_1,b_1,a_2,b_2\in\mathbb{Q}$. \\ \\
$q_1+q_2 = a_1+\sqrt{2}b_1+a_2+\sqrt{2}b_2=a_1+a_2+\sqrt{2}(b_1+b_2)$. This is in $\mathbb{Q}[\sqrt{2}]$, as $(a_1+a_2)=a_3\in\mathbb{Q}$, and $(b_1+b_2)=b_3\in\mathbb{Q}$, and so by definition $(a_3+\sqrt{2}b_3)\in\mathbb{Q}[\sqrt{2}]$. \\ \\
$q_1\cdot q_3 = (a_1+\sqrt{2}b_1)(a_2+\sqrt{2}b_2) = (a_1a_2+a_1\sqrt{2}b_2+\sqrt{2}b_1a_2+2b_1b_2)=(a_1a_2+2b_1b_2+\sqrt{2}(a_1b_2+b_1a_2))$. $(a_1a_2+2b_1b_2)=a_3\in\mathbb{Q}$, and $(a_1b_2+b_1a_2)=b_3\in\mathbb{Q}$, and so by definition, $(a_3\cdot\sqrt{2}b_3)\in\mathbb{Q}[\sqrt{2}]$. \\\\
$q_1+(-q_1)=0\Rightarrow (a_1+\sqrt{2}b_1)+(-q_1)=0\Rightarrow -q_1=-a_1+\sqrt{2}(-b_1)$. $(-a_1),(-b_1)\in\mathbb{Q}$ trivially, so $\mathbb{Q}[\sqrt{2}]$ is closed under taking additive inverses.  \\ \\ 
$q_1\cdot q_1^{-1}=1\Rightarrow q_1^{-1}=\frac{1}{q_1}=\frac{1}{a_1+\sqrt{2}{b_1}}=\frac{a-\sqrt{2}b}{a^2-2b^2}$. $a-\sqrt{2}b$ is contained in our field, and $a^2-2b^2$ is a rational number that is non-zero when $a-\sqrt{2}b$ is non-zero, so $q_1^{-1}$ is contained in our field, and so $\mathbb{Q}[\sqrt{2}]$ is closed under the above operations.  
    \item [c.] For some field $\mathbb{F}$ with $2^n$ elements, $2^{n-1}+2^{n-1}=0\Rightarrow 2^{n-1}(1+1)=0\Rightarrow 2^{-1(n-1)}2^{n-1}(1+1)=0\cdot 2^{-1(n-1)}\Rightarrow 1+1=0$. 
    \item [d.] Assume $1+1=0$. Then for every $a\in\mathbb{F}$, $a+a=0$. Consider a field $\mathbb{F}$ with $n=2a+1$ elements, with $n$ being the size of the field and $a\in\mathbb{N}$. Then for all fields with an $n$ amount of elements, $2a=a+a=n-1\Rightarrow 0=n-1$. You can't have a field with only one element, so $n-1>0$.  

\end{enumerate}

\pagebreak
\item \begin{enumerate}
\item [a.] If $f$ is surjective, then for every $x\in X$ there exists $y\in X$ such that $f(y)=x$. But this implies that there exists $z\in X$ such that $f(z)=y$, so $f(f(z))=x$. Because $f$ is a surjection, and $X$ is a finite set, there will be some number of applications of $f$, such that $f(f\ldots(x)=x$. But since $f$ is a surjection, the set of all the values of $f(f\ldots(x)$ will be equal to $X$, and so at no point will $f(x)=f(y)\not\Rightarrow x=y$, or $f$ must be injective if it is surjective. 
\item [b.] We will prove each separately. \begin{enumerate}
    \item [(i.)] When $\sigma$ and $\tau$ are permutations of $[n]$, then it follows that $\sigma\cdot\tau$ is a permutation. Since $S_n$ is the set of all permutations, $\sigma\cdot\tau\in S_n$. 
    \item [(ii.)] Because composition of functions are always associative: $(\sigma\cdot\tau)\cdot\rho=\rho(\sigma(\tau(x)))=\sigma(\tau(\rho(x))=\sigma\cdot(\tau\cdot\rho)$ $\forall x\in[n]$. 
    \item [(iii.)] Let $i:[n]\rightarrow[n]$ be the identity function. By definition, $\sigma\cdot i=i\cdot\sigma=\sigma$. 
    \item [(iv.)] If $\sigma$ is a permutation, then it follows that $\sigma^{-1}$ is a permutation. Since $S_n$ is the set of all possible permutations of $[n]$, they must both be in $S_n$. $\sigma^{-1}$ exists because $\sigma$ is a bijection and unique in $S_n$. 
\end{enumerate}
\end{enumerate}

\pagebreak
\item \begin{enumerate}
\item [a.] Let $A=[a_{ii}],B=[b_{ii}]\in M_{n,n}(\mathbb{F})$. $[a_{ii}]\cdot[b_{ii}]=[\sum_{j=1}^ia_{ij}b_{ji}]$. Then Tr$([a_{ii}]\cdot[b_{ii}])=\sum_{i=1}^n\sum_{j=1}^na_{ij}b_{ji}$. Changing summation order, $\sum_{i=1}^n\sum_{j=1}^na_{ij}b_{ji}=\sum_{j=1}^n\sum_{i=1}^nb_{ji}a_{ij}$. Therefore, $\text{Tr}(A\cdot B)=\text{Tr}(B\cdot A)$. 

\item [b.] If $\sigma$ is a cyclic permutation, then for some $a,b\in\mathbb{N}$, $(A_1\ldots A_a)=(A_{\sigma(b)}\ldots A_{\sigma(m)})$. It follows then that $(A_{a+1}\ldots A_m)=(A_{\sigma(1)}\ldots A_{\sigma(b-1)})$. Because multiplication is associative, $(A_{\sigma(1)}\ldots A_{\sigma(m)})=(A_{\sigma(b)\ldots A_{\sigma(m)}})(A_{\sigma(1)}\ldots A_{\sigma(b-1)})=(A_1\ldots A_m)$. Because tracing is commutative, we can rearrange $(A_{\sigma(1)}\ldots A_{\sigma(m)})$ such that it is equal to $(A_{\sigma(b)\ldots A_{\sigma(m)}})(A_{\sigma(1)}\ldots A_{\sigma(b-1)})$ and therefore equal to $(A_1\ldots A_m)$. Then $\text{Tr}(A_1\ldots A_m)=\text{Tr}(A_{\sigma(1)}\ldots A_{\sigma(m)})$.

\item [c.] Let $A=\begin{bmatrix}1&1\\1000&1\end{bmatrix}, B=\begin{bmatrix}0&1\\0&1\end{bmatrix}, C=\begin{bmatrix}1&1\\1&1\end{bmatrix}$. Then $\text{Tr}(A\cdot B\cdot C) = 1003$, but $\text{Tr}(A\cdot C\cdot B) = 2002$. 
\end{enumerate}

\pagebreak
\item 
$\begin{bmatrix}
2 & -3 & -7 & 5 & 2 & \vrule & -2 \\ 
1 & -2 & -4 & 3 & 1 & \vrule & -2 \\ 
2 & 0 & -4 & 2 & 1 & \vrule & 3 \\ 
1 & -5 & -7 & 6 & 2 & \vrule & -7
\end{bmatrix}$\\\\ 
Switch the second and first rows, then set $(2)=(2)-(1)*2$, $(3)=(3)-(1)*2$, and $(4)=(4)-(1)$.\\   \\
$\begin{bmatrix}
1 & -2 & -4 & 3 & 1 & \vrule & -2 \\ 
0 & 1 & 1 & -1 & 0 & \vrule & 2 \\ 
0 & 4 & 4 & -4 & -1 & \vrule & 7 \\ 
0 & -3 & -3 & 3 & 1 & \vrule & -5
\end{bmatrix}$\\ \\ 
Set $(1)=(1)+(2)*2$, $(3)=(3)-(2)*4$, $(4)=(4)+(2)*3$ \\\\ 
$\begin{bmatrix}
1 & 0 & -2 & 1 & 1 & \vrule & 2 \\ 
0 & 1 & 1 & -1 & 0 & \vrule & 2 \\ 
0 & 0 & 0 & 0 & -1 & \vrule & -1 \\ 
0 & 0 & 0 & 0 & 1 & \vrule & 1 
\end{bmatrix}$\\ \\ 
Set $(3)=(3)*-1$, then $(1)=(1)-(3)$, and $(4)=(4)-(3)$. \\ \\
$\begin{bmatrix}
1 & 0 & -2 & 1 & 0 & \vrule & 1 \\ 
0 & 1 & 1 & -1 & 0 & \vrule & 2 \\ 
0 & 0 & 0 & 0 & 1 & \vrule & 1 \\ 
0 & 0 & 0 & 0 & 0 & \vrule & 0
\end{bmatrix}$ \\ \\ 
Setting $x_3=u$, and $x_4=v$, the solutions to this system are $\{(2u-v+1),(-u+v+2),u,v,1\}:u,v\in\mathbb{R}$. 


\end{enumerate}

\end{document}
