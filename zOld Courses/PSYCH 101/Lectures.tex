\documentclass[10pt,letter]{article}
\usepackage{amsmath}
\usepackage{amssymb}
\usepackage{amsthm}
\usepackage{graphicx}
\usepackage{setspace}
\onehalfspacing
\usepackage{fullpage}
\usepackage{enumitem}
\newtheorem*{remark}{Remark}
\theoremstyle{plain}
\newtheorem*{theorem*}{Theorem}
\newtheorem{theorem}{Theorem}[section]
\newtheorem{corollary}{Corollary}[theorem]
\newtheorem*{lemma*}{Lemma}
\newtheorem{lemma}[theorem]{Lemma}
\theoremstyle{definition}
\newtheorem{definition}{Definition}[section]
\newtheorem*{definition*}{Definition}
\newcommand{\Mod}[1]{\ (\mathrm{mod}\ #1)}

\begin{document}

\section*{Module 1: Memory}
\paragraph{New Terms}
\begin{itemize}
    \item Decay: Forgetting that occurs due to the passage of time 
    \item Interference: Forgetting that occurs when similar pieces of information interfere with each other 
    \item Declarative Memory (explicit memory): Memories that can be transfered by talking (events, facts, etc.) 
    \item Non-declarative memory (implicit memory): Memories that cannot be transfered by talking (motor movement, conditioned responses, etc.)
    \item Chunking: Combining information to create fewer but more meaningful chunks of information
\end{itemize}


\subsection*{How has the understanding of memory evolved?}
\paragraph{Forgetting Curve}
The curve that represents information loss. Hermann Ebbinghaus discovered most information loss occurs relatively quickly. 

\paragraph{Behaviourist Perspective}
To behaviourists, the mind was like an unstudy-able black box. Behaviourists simply studied stimuli, and the organism's behavioural reponse. 
\paragraph{Cognitive Perspective}
To cognitivists, the mind was like a computer. Information was first encoded into memory, then stored, and retrieved. 

\subsection*{How reliable is memory?}
\paragraph{Errors in memory}
Memories must be \textbf{reconstructed} in your mind every time they are used. However, they may undergo omisions (forget), substitutions (substitute), or insertions (add), as memories are not perfect. Some memories may be entirely false: these are called confabulations. When you hear about someone else's experience, then later mistake it as your own, it is called source amnesia. 
\paragraph{Eyewitness Testimony}
A leading questions presupposes the truth about what kind of answer is likely to be correct. They often lead to confabulations in test subjects. 

\subsection*{What is the first process of memory?}
\paragraph{Encoding}
Encoding is converting various types of information into neuronal impulses. Passive encoding is when no effort is invest in remembering information, typically when you are watching tv or reading. Active encoding is when effort is expended, typically when taking notes, or reflecting on information. Structural (structure), phonemic (sound), and semantic (meaning) are some forms of active encoding. When asked these questions, people retained information in that order as well (people asked structural questions retained the least, etc). There are enrichment techniques that can be used to process information deeper: 
\subparagraph{Elaboration}
Elaboration is a form of semantic encoding that aids the recall of new information by connecting it to existing information. For example, using metaphors (like the computer one above). 
\subparagraph{Self referent encoding}
Self referent encoding is a form of semantic encoding that aids the recall of new information by connecting it to oneself. 

\subparagraph{Dual encoding}
Dual encoding is a form of semantic encoding that aids the recall of information by producing redundant codes. E.g. using kermit the frog to remember frog. 


\subsection*{What is the second process of memory?}
\paragraph{Storage}
Research has identified three different memory stores: sensory memory, short-term memory, and long-term memory. These stores are referred to as the three box model of memory storage 
\paragraph{Sensory Memory}
As sensory organs collect information, it is held briefly in a sensory register. It can be thought of overexposing the film when drawing with sparklers. Visual information is usually held for $0.25$ seconds, and auditory information is held for just under 1 second (found by using a study with 3 rows of letters, 3 tones, the tones played for increasing amounts of delay after the letters flashed briefly). When an individual's visual register lasts longer than the typical, this is called eiditic or photographic memory. After the retention interval has elapsed, sensory memory decays. 
\paragraph{Short-term memory}
Short-term memory usually decays in approximately 20 seconds after rehearsal ends (found by using a study with 3 letters, and a three digit number. the 3 digit number was used to count down, until the red light shows, and then the 3 letters are then to be repeated). Inculcation is a form of rehearsal in which you repeat the desired memory over and over, like for remembering a phone number. Short-term memory has been found to hold between 3 and 4 chunks of information. Working memory is an expansion of short-term memory, that has memory and attention packaged together. 
\paragraph{Long-term memory}
When information is present but cannot be retrieved, it is called inaccessible. When information used to be present but has since been lost, it is called unavailable. Information can be stored in networks according to semantic meaning (semantic network). Priming one concept usually activates other related concepts, called spreading activation. Types of long term memory: 
\begin{itemize}
    \item Prospective - memory to perform actions in the future
    \item Retrospective - memory of the past
    \begin{itemize}
        \item Declarative(explicit)- can be transfered 
        \begin{itemize}
            \item episodic (past events)
            \item semantic (facts)
        \end{itemize}
        \item Non-declarative(implicit) - can't be transfered
        \begin{itemize}
            \item procedural (muscle memory)
            \item conditioned responses
        \end{itemize}
    \end{itemize}
\end{itemize}

\subsection*{What is the third process of memory?}
\paragraph{Retrieval}
Retrieval can be in the form of recall: where the respondent has to retrieve information without any cues (e.g. short-answer question), or in the form of recognition: where the respondent has the recognize the target information in the presence of distractor information (e.g. multiple choice). Two different explanations account for the failure to retrieve information: 
\begin{itemize}
    \item Decay: information is lose because of the passage of time. Most indicative of sensory and short-term memory
    \item Interference: similar pieces of information interfere with one another. Proactive interference occurs when existing information interferes with your ability to store new information (remembering maiden name). Retroactive interference occurs when new information interferes with your ability to retrieve old information(cramming for exam). 
\end{itemize}
Although children have excellent memories, they are even more susceptible than adults to techniques that can inadvertently create false memories. 

\section*{Module 2: Stats and Research Designs}
\paragraph{New Terms}
\begin{itemize}
    \item Population: The complete set of scores 
    \item Sample: A subset of scores drawn from the population 
    \item Sampling error: The discrepancy observed between a sample of scores and the population of scores it was drawn from 
    \item Mean: A statistic that describes the central tendency 
    \item Standard deviation: a statistic that describes the variability
    \item Correlation Coefficient: a statistic that describes the direction and magnitude of the relationship of two variables 
\end{itemize}
\subsection*{What is a sample and why do we use them?}
\paragraph{Variables}
Measurement is the process of assigning numbers to represent the things we observe. A set of observations can be constant, or variable. Variables can be quantitative (values reflect having more or less of some attribute), or qualitative (values reflect different categories). 
\paragraph{Populations vs samples}
A population is the complete set of observations. A sample is a subset. The average in a population is called a parameter, and the average in a sample is called a statistic. 
\paragraph{Sampling Error}
Sampling error is defined as the discrepancy between a population parameter and its corresponding sample statistic. It can be found by $|\bar{x}-\mu|$. 

\subsection*{Why is the mean the most common measure of central tendency?}
The mode is usually used for qualitative variables, and the median is used for quantitative variables that are skewed. The mean is used because most phenomena follows a normal curve. 

\subsection*{Why is the standard deviation the most common measure of variability?}
variance etc.

\subsection*{What does the correlation coefficient tell us?}
Positive correlation is when the values are tending towards the same direction, and negative is when they are tending in opposite directions. 


\subsection*{How are descriptive and inferential statistics used in social science?}
Use a t-test. When $p$ is less than $.05$, the test is statistically significant. 

\section*{Module 3: Evolution and Psychology}
\subsection*{How does evolution work?}
\paragraph{Mechanisms of Evolution}
The principle of uniqueness is sometimes referred to as variation. It recognizes that individuals differ from each other. The principle of heredity recognizes that DNA can be passed from parent to offspring, and yet the offspring is different. This is due to recombination (parents DNA is scrambled up) and mutation (random mistakes in copying DNA). Natural selection determines which adaptations are passed onto the next generation. Darwin's finches (large population introduced to new island, they eventually speciated due to limited resources) are an example of adaptive radiation. 

\paragraph{Human Evolution}
Remember that evolution has no goal: it is a mindless process. 


\subsection*{Why should we care about evolution in psychology?}
\paragraph{Four levels of explanation}
\begin{itemize}
    \item Standard psychology
    \begin{itemize}
        \item Proximate (mechanism): how does the behaviour work? 
        \item Ontological (development): how does the behaviour change across the lifespan?
    \end{itemize}
    \item Evolutionary Psychology
    \begin{itemize}
        \item Ultimate (function): what does the behaviour do? 
        \item Phylogenetic (evolution): how did the behaviours change across generations?
    \end{itemize}
\end{itemize}
\paragraph{Why should we care?}
some things that we did in the past might not help in modern society


\subsection*{How is evolution applied in psychology?}
\paragraph{Can our ancestral environment explain behaviour?}
Used the four card problem (two restatements of if p then q). The first one, using numbers and colours, most people struggle with because it is hard to map logic into the common problems of ancestral past. The second one, using age and drinks, simulates social contract, and so is easier to solve. 
\paragraph{Sexual selection}
Intra-sexual selection is a form of sexual selection whereby individuals of one sex compete with each other for sexual access to the other sex, ie. bucks and deers. Inter-sexual selection is when members of one sex choose members of the other sex based on preferred characteristics, ie. peacocks. 
\paragraph{Parental Investment Theory}
Parental investment is any expenditure that benefits an offspring and simultaneously reduces the parent's ability to invest in other aspects of fitness, e.g. mating. Parental investment theory recognizes that there are asymmetries in the degree to which males and females are obligated to invest in offspring. 


\section*{Module 4: Visual Perception}
\paragraph{New Terms}
\begin{itemize}
    \item Sensation: The process of detecting physical energy in the environment using sensory receptors. 
    \item Transduction: The process of converting information about physical energy into neuronal impulses. 
    \item Perception: The brain processes responsible for interpreting sensory information based on neural impulses. 
    \item Bottom-up Processing: A form of processing that begins with individual elements and progresses to whole elements. 
    \item Top-down Processing: A form of processing that begins with whole elements and progresses to individual elements.
\end{itemize}

\subsection*{How do properties of waves contribute to our perception?}
\begin{tabular}{|c|cc|}
\hline
&Light Waves&Sound Waves\\\hline 
Frequency & Hue & Pitch \\ 
Amplitude & Brightness & Loudness \\ 
Complexity & Saturation & Timbre\\
\hline
\end{tabular}

\subsection*{How do we perceive colour?}
\begin{tabular}{|c|cc|}
\hline
Characteristic & Rods & Cones\\\hline 
How many? & 120-125 million & 7-8 million \\ 
Where are most concentrated? & Periphery of retina & Centre of retina (fovea) \\ 
Sensitivity to light? & High sensitivity & Low sensitivity\\ 
Sensitive to colour? & No & Yes\\
\hline
\end{tabular}

\subsection*{How do we identify objects in the world?}
\paragraph{Gestalt Principles}
\begin{itemize}
    \item Figure or Ground: Scenes are divided into either Figure or Background 
    \item Principle of Proximity: Objects positioned closely tend to be grouped together in perception 
    \item Principle of Similarity: Objects resembling each other tend to be grouped in perception 
    \item Principle of Continuity: Objects are perceived as continuous 
    \item Principle of Closure: Objects are perceived as completed forms by filling in gaps
\end{itemize}

\subsection*{How do we know how close or far away an object is?}
Prey animals usually have eyes on opposite sides of their head, to allow more range of vision. Primates have \textbf{steroscopic vision}, with two eyes side by side on the front of the face. This way, there are only minor discrepancies between the images of each eye (\textbf{retinal disparity}). 
\paragraph{Monocular Cues}
\begin{itemize}
    \item \textbf{Interposition} is the tendency to perceive blocked objects as more distant than occluding objects 
    \item \textbf{Linear Perspective} describes the tendency to perceive depth when two lines are observed to converge 
    \item \textbf{Relative Size} describes the tendency to perceive smaller objects as further away than larger objects of the same type. 
    \item \textbf{Texture Gradient} describes the tendency for the units that make up a texture to become both distorted and denser as they recede into the distance. 
    \item \textbf{Visual Acuity} describes the atmosphere affecting far distances 
    \item \textbf{Motion Parallax} is when distant objects tend to move slower when the observer is moving. 
\end{itemize}



\section*{Module 5: Consciousness}
\paragraph{New Terms}
\begin{itemize}
    \item Consciousness: The awareness of the internal and external world. The content of your mind at this moment reflected in the internal monologue you have with yourself.
    \item Dreaming: An altered state of consciousness that occurs during REM sleep. 
    \item Hypnosis: Systematic procedure that induces a heightened state of suggestibility in another person. 
    \item Meditation: Collection of systematic procedures to self-induce heightened awareness and to establish voluntary control of mental processes. 
    \item Homeostasis: Tendency for the nervous system to resist deviations from normal parameters induced by external forces.
    \item Psychoactive Substance: A chemical or drug, either synthesized in a lab or present in natural botanicals, that can alter mental, emotional, or behavioural functioning.
\end{itemize}

\subsection*{What is Consciousness?}
\paragraph{Quantitative changes in Consciousness}
\textbf{Quantitative} changes in consciousness reflect changes in the stream's depth. Stimulants increase consciousness, and depressants decrease consciousness. 

\paragraph{Qualitative changes in Consciousness}
\textbf{Qualitative} changes in consciousness reflect changes in the essence of the stream itself, like water going over a waterfall. The stream's basic properties change temporarily as water travels over the precipice. Dreams and psychoactive substances change the quality of consciousness, as they offer a different experience of consciousness, as opposed to more or less consciousness. 


\subsection*{What are Dreams?}
\textbf{Day residue} is when you seen someone in your dreams that you haven't thought of in ages, or dreamed of something that happened that day.  
\paragraph{Why do we Dream?}
There are four competing theories of why we dream: 
\subparagraph{Wish Fulfillment}
Proposed by Sigmund Freud in 1899, the theory of wish fulfillment is that the brain dreams to maintain integrity of the unconscious by exercising unexpressed desires. Freud proposed that dreams are composed of two types of content: \begin{itemize}
    \item \textbf{Manifest content} is the literal content of the dream. It represents those aspects of the dream that we are aware of. 
    \item \textbf{Latent content} is the symbolic content of the dream. 
\end{itemize}
The study of dream analysis is not empirical, and as such the theory of wish fulfillment should be viewed more as mythology than science. 

\subparagraph{Problem Solving}
Proposed by Rosalind Cartwright, the theory of problem solving is that the brain dreams to reflect waking problems and help us discover creative solutions. Several studies have confirmed that people do in fact dream about their problems. It is unlikely that this theory is correct, as people often dream of problems during the day as a result of day residue, solutions aren't usually found in dreams, and there is no explanation for dream content that is unrelated to problems. 

\subparagraph{Mental Housekeeping}
Proposed by various theorists, the theory of mental housekeeping is that the brain dreams allow the brain to strengthen new connections associated with learning and memory. One way to strengthen these connections is through \textbf{consolidation}, which is where the brain runs signals through some pathways that were formed during learning. Studies show that people who undergo REM sleep perform better on memory tests afterwards. \\ 
All the studies may only show the purpose of REM sleep, rather than dreams. Also, if this theory was correct, how would it explain recurring dreams? And dreams unrelated to learning? 

\subparagraph{Activation Synthesis}
Opposed to the other three, the theory of activation synthesis, dreams are an artifact of spontaneous neuronal activity originating in the brain stem during REM sleep. A structure at the brain stem called the \textbf{pons} becomes active during REM sleep. Pons are responsible for the rapid eye movements associated with REM sleep, but they also work to paralyze the rest of the body muscles. \\ 
This theory does not explain the internal consistency of dreams, and it fails to account for memory consolidation. 


\subsection*{What is hypnosis and meditation?}
\paragraph{Hypnosis}
Franz Mesmer was an Austrian physician working in 18th century Paris who healed people using an elaborate routine called \textbf{animal magnetism}. The following are some theories of hypnosis. 
\subparagraph{Socio-cognitive Theory}
Socio-cognitive, or role-playing, theory suggests that hypnosis is a normal mental state consistent with our current understanding of consciousness. In other words, advocates don't believe any change in consciousness occurs while hypnotized. 
\subparagraph{Dissociative Theory}
The dissociative theory proposes that hypnosis is in fact an altered state of consciousness. \textbf{Dissociation} occurs when mental processes split into separate streams of awareness operating in parallel. It is best described with an example: \textbf{highway hypnosis}. Highway hypnosis is when you forget traveling through a path where you frequently travel, as your consciousness was focused on something else.

\subsection*{What are psychoactive substances?}
\paragraph{Psychoactive Substances}
\begin{itemize}
    \item \textbf{Tolerance}: the more you use a drug, the more you have to use it to achieve the same effect. 
    \item \textbf{Withdrawal Symptoms}: symptoms that occur when a drug is not being used.
    \item \textbf{Homeostasis}: the tendency for the nervous system to resist deviations from normal parameters induced by external forces.
    \item \textbf{Psychoactive Substances}: drugs that have effects on the nervous system and influence thought, mood, or perception. There are four categories: 
    \begin{itemize}
        \item \textbf{Stimulants}: a class of drugs that increase nervous system activity and include nicotine and caffeine. 
        \item \textbf{Depressants}: a class of drugs that decrease nervous system activity and include alcohol, barbiturates. Opiates are sometimes defined as depressants, but some put them into their own class. 
        \item \textbf{Hallucinogens}: a class of drugs that alter perception and include LSD, psilocybin, and mescaline. 
        \item \textbf{Other}: Some drugs don't neatly fall into any of the above categories. Cannabis has analgesic qualities like opioids but also can have hallucinogenic effects. Ecstasy also has hallucinogenic qualities, but also increases nervous system activity.
    \end{itemize}
\end{itemize}


\section*{Module 6: Problem Solving}
\begin{itemize}
    \item Concept: Mental category that groups elements together that share common properties
    \item Schema: Integrated mental network connecting many concepts together based on experience
    \item Algorithm: Set of rules for solving a problem that guarantees a solution even if the user does not understand how or why it works
    \item Heuristic: Rule of thumb used to guide behaviour, decision-making, and problem solving that does not guarantee an optimal solution, but works well under most circumstances
    \item Discounting: Tendency to devalue or discount future rewards relative to more immediate rewards
\end{itemize}
\subsection*{What is risk avoidance?}
We tend to dislike risk and prefer immediate rewards over future rewards. Researches can measure these tendencies using \textbf{discounting tasks}, in which a series of choices are presented one at a time, and one choice is changed over the series until the participant changes their choice. Those with few resources tend to become riskier and live for the present, whereas those with many resources tend to become even more averse to risk and to live for the future, 

\subsection*{What is thought?}
\begin{itemize}
    \item \textbf{Narrower, basic,} and \textbf{broader concepts} convey very precise, optimal, and omit important information respectively. 
    \item \textbf{Stereotypes} are a special category of beliefs that express thoughts about a group of people. 
    \item A \textbf{schema} is an integrated mental network of ideas. \textbf{Social role} (self-explanatory) and \textbf{social script} (what to do in social situations) are both examples of schema. 
    \item \textbf{Culture shock} are when social scripts learned in your home culture don't apply in a new culture
\end{itemize}


\subsection*{What is problem solving?}
\begin{itemize}
    \item \textbf{Formal reasoning problems} are those in which the information is complete and objective. The solutions to formal reasoning problems are usually algorithms. 
    \item \textbf{Informal reasoning problems} are those in which the information is complete and subjective. The solutions usually are a result of \textbf{heuristics}, rules of thumb that aid in decision making. \begin{itemize}
        \item \textbf{Availability heuristic} is a decision strategy that people use that bases a prediction on memorable instances rather than on base-rate probability. 
        \item \textbf{Representative heuristic} is a decision strategy that base decisions on commonality or similarity. 
    \end{itemize}
\end{itemize}

\subsection*{What is decision making?}
\begin{itemize}
    \item \textbf{Utility} is making decisions based on economic gain (or benefit) derived from a given option 
    \item \textbf{Subjective utility} is making decisions based on non-economic gains. 
    \item \textbf{Additive strategy} of decision making is when you "rate" attributes of the objects in the decision, then add them up and choose the one with the highest sum. This is most useful when faced with a few options. 
    \item \textbf{Elimination by aspects} is a decision making process that identifies minimum criteria that must be satisfied, then eliminating all those that don't fit those criteria. This is most useful when faced with a large amount of options, and then using the additive strategy. 
\end{itemize}



\section*{Module 7: Emotion}
\begin{itemize}
    \item \textbf{Affect}: more general term used to describe feelings, especially those that mediate an organism's interaction with stimuli 
    \item \textbf{Emotion}: more specific form of affect. These affective states tend to be high intensity, short in duration, and provoked by specific stimuli or events. 
    \item \textbf{Mood}: more specific form of affect. These affective states tend to be low intensity, long duration, and are generally not tied to specific stimuli or events
    \item \textbf{Motivation}: processes that initiate, guide, and sustain goal directed behaviour. Emotions are one type of motivational system. 
    \item \textbf{Cognitive Appraisal}: Inference used to explain the cause of an event or behaviour
\end{itemize}
\subsection*{What is emotion?}
\textbf{Motivation} is the pursuit of goal-directed behaviour. For instance, hunger can be motivation, behaviour is eating, and the goal is to prevent starvation. \textbf{Drive} and \textbf{incentive} are inverse types of motivation: drive are unpleasant internal states that typically result from deprivation, and incentives are desirable stimuli that motivate us to pursue them. 
\paragraph{Emotion}
\textbf{Emotions} are affective experiences in response to stimuli that induce goal-directed behaviour. Some motivates motivate us to approach stimuli, and some motivate us to avoid stimuli. Emotions also have \textbf{valence}, which is the subjective experience of the affective response. Emotional responses are typically \textbf{adaptive}, which means they promote survival and reproduction. \textbf{Moods} are another type of affective experience that can have valence, but moods are longer and less intense. 
\paragraph{Thee Components of Emotion}
\paragraph{Physiological}
The first component of emotion. \textbf{Autonomic arousal} is what leads to the "feeling" of an emotion. As autonomic arousal increases, sweat begins to accumulate in the sweat glands, and this can be measured using the \textbf{galvanic skin response (GSR)}. 
\paragraph{Cognitive Appraisal}
The second component of emotion. Appraisals are thoughts we generate to explain events. Recognizing when an automatic maladaptive appraisal have been made, and identifying alternative explanations is called a \textbf{reappraisal}.

\paragraph{Behaviour}
This component includes actions performed as a consequence of feeling an emotion. Usually it is difficult to draw connections between specific emotions and actions, except for \textbf{facial expressions}. But, since we can control our facial expressions, it is important to differentiate \textbf{emotional experience} (the above two components), and \textbf{emotional expression} (the component of behaviour). 

\subsection*{Are emotions universal?}
Basically yes. A study by Ekman and Friesen concluded that happiness, sadness, surprise, anger, disgust, and fear are \textbf{primary} (evolved: needing no learning to experience or understand them), and other \textbf{secondary}(learned: emotions might require learning and could vary between cultures). Robert Plutchik's model is a cone, and each slice represents one of eight primary emotions: joy, trust, fear, surprise, sadness, disgust, anger, and interest. All secondary emotions are a blend of primary emotions. Display rules are social rules that dictate how and when an emotion can be expressed. For example, you shouldn't express joy at a funeral, and Japanese smile to disguise embarrassment.

\subsection*{What are the theories of emotion?}
\begin{itemize}
    \item The \textbf{James-Lange theory} of emotion stipulates that autonomic arousal causes the emotional experience. Basically, the stimulus results in autonomic arousal, which results in conscious feeling.
    \item The \textbf{Canon-Bard theory} of emotion states that subcortical regions of the brain stimulate both autonomic activity and emotional experience. Basically, stimulus results in subcortical brain activity, which results in both conscious feeling and autonomic arousal. This is backed by \textbf{facial feedback hypothesis}, which proposes that facial expressions can influence our emotional state. An experiment made people smile (hold pen in teeth), and they found a comic more funny than people who were made to frown. 
    \item The \textbf{Two-Factor theory} proposes that two things must occur for an emotion to be experienced: autonomic arousal, and then cognitive appraisal. 
\end{itemize}



\section*{Module 8: Development}
\begin{itemize}
    \item \textbf{Germinal Stage of Prenatal Development}: Development that occurs during the first two weeks following conception. Begins with the fertilization of the egg with sperm and ends when the placenta begins to function.
    \item \textbf{Embryonic Stage of Prenatal Development}: Development that occurs between two weeks and approximately two months following conception. Stem cells begin to differentiate and organs begin to form.
    \item \textbf{Fetal Stage of Prenatal Development}: Development that occurs between two months following conception and ends at birth. Fetus is capable of movement and organs begin to function.
    \item \textbf{Attachment}: The emotional bond an infant has with their primary caretaker
    \item \textbf{Stranger Anxiety}: The tendency for infants to express anxiety when confronted with novel people
\end{itemize}
\subsection*{What is prenatal development?}
\textbf{Prenatal development} refers to the developmental period preceding infancy, beginning with conception and ending with birth. 
\paragraph{Germinal Stage}
The \textbf{germinal stage} is from conception-2 weeks. The single celled zygote migrates along the fallopian tube on its way to implanting in the uterine wall. At this point, all cells are \textbf{undifferentiated} (yet to specialize into their eventual cell types). By day 5 or 6, they begin to form two masses. The inner mass (embryoblast) becomes the embryo, the outer mass (tyophoblast) forms the placenta. Together they are the blastocyst. 

\paragraph{Embryonic Stage}
The \textbf{embryonic stage} is from 4 weeks to 8 weeks. It commences once the Placenta begins to function. Cell differentiation begins to occur during this stage. Organs also develop, and any mistake in mitosis results in birth defects, and sometimes miscarriage. 

\paragraph{Fetal Stage}
The \textbf{fetal stage} is from 8 weeks to birth. Bones and muscles have formed and been connected. Organs begin to function. 

\paragraph{Prenatal Environment}
The \textbf{prenatal environment} consists of conditions present within the uterus where the embryo/fetus is developing. 
\begin{itemize}
    \item \textbf{Maternal Nutrition}: If the mother consumes too many calories, then the delivery will be more difficult. Too little, and you have an increased risk of schizophrenia.
    \item \textbf{Maternal Drug Use}: \textbf{Fetal Alcohol Syndrome} is something you want to avoid. It results in microcephaly, heart defects, hyperactivity, and mental and motor development. Tobacco use also results in increased risk of miscarriage, prematurity, stillbirth, and SIDS.
    \item \textbf{Maternal Illness}: Mothers with \textbf{genital herpes} must tell their OBGYN, so that a \textbf{caesarean section} can be used to deliver the baby instead. 
    \item \textbf{Teratogens}: a class of chemical agents that are carcinogenic and mutagenic. \textbf{Pregnancy sickness} is the sickness that expectant mothers go through when they eat foods high in teratogens. 
\end{itemize}


\subsection*{What is attachment?}
\begin{itemize}
    \item \textbf{Attachment} is a term used to describe the emotional bond an infant has with her caretaker. 
    \item \textbf{Imprinting} is when a juvenile looks for an follows the first large moving object it encounters. It is \textbf{stimulus independent}(does not follow their mother, only the first large object), and it has a \textbf{critical period} (only newly hatched goslings can imprint, not adult ones). Theorized by Konrad Lorenz.
    \item Theorized by John Bowlby, attachment is the human equivalent of imprinting.
    \item \textbf{Separation Anxiety} is when the infant is separated from the caretaker. Newborns fail to demonstrate \textbf{stranger anxiety}, which supports the fact that they do not get attached to caretakers fast. 
\end{itemize}
\paragraph{Attachment Styles}
\begin{itemize}
    \item \textbf{Secure Attachment}: A healthy and stable emotional bond 
    \item \textbf{Anxious/Ambivalent Attachment}: The infant explores anxiously when mom is present, protests when mom leaves, and is difficult to console after she returns 
    \item \textbf{Avoidant Attachment}: Infant explores comfortably, but fails to protest when mom leaves and is reluctant to greet her when she returns. 
    \item \textbf{Disorganized Attachment}: Infant expresses a fear response to the caretaker herself. 
\end{itemize}

\subsection*{How does attachment develop?}
\begin{itemize}
    \item Factors Associated with Mother: \begin{itemize}
        \item Attentiveness
        \item Sensitiveness
    \end{itemize}
    \item Factors Associated with Infant: \begin{itemize}
        \item Temperament 
    \end{itemize}
    \item Socio-cultural Factors: \begin{itemize}
        \item Different cultures encourage their children to due different things. 
    \end{itemize}
    \item Adult Attachment: \begin{itemize}
        \item \textbf{Securely attached} individuals are comfortable with intimacy and autonomy
        \item \textbf{Preoccupied} individuals are highly dependent and see themselves as unworthy of love 
        \item \textbf{Dismissing} individuals are counter-dependent and see themselves as worthy of love. 
        \item \textbf{Fearful} individuals are fearful of intimacy and socially avoidant
    \end{itemize}
\end{itemize}


\section*{Module 9: Freudian and Humanist Theory}
\begin{itemize}
    \item \textbf{Personality Trait}: A durable disposition to behave consistently across a variety of situations. People have hundreds of these traits, but typically each person has five to ten core traits that they closely identify with. 
    \item \textbf{Personality}: A person's unique constellation of consistent behavioural traits. These become relatively stable following adolescence. 
    \item \textbf{Temperament}: A person's characteristic physiological response to the environment. Temperament is present at birth and provides the foundation on top of which personality is built.
    \item \textbf{Theory of Personality}: A model that attempts to explain how one comes to acquire their tendency to behave consistently. 
    \item \textbf{Personality Inventory}: An instrument used to measure one's tendency to behave consistently. 
\end{itemize}

\subsection*{What is personality?}
\begin{itemize}
    \item \textbf{Personality} is our tendency to behave consistently across situations. \textbf{Personality psychology} recognizes that there is something about us that remains stable over long periods of time that contributes to our actions. \textbf{Social psychology} is interested in how external factors can influence our emotions, decisions, and actions. 
    \item \textbf{Extroversion} is the tendency to be outward focused, whereas \textbf{introversion} is the tendency to be inward focused. 
    \item \textbf{Theories of personality} are models that explain how an individual's tendency to behave consistently over time came to be. \textbf{Informal Theories:}\begin{itemize}
        \item \textbf{Chinese zodiac} proposes that an individual's personality is determined by the year of their birth. 
        \item \textbf{Western zodiac} proposes that an individual's personality is determined by the month of their birth. 
        \item Many east asian cultures believe that personality is determined by \textbf{Blood type}. 
        \item The ancient greeks believed personality was a reflection of the balance between \textbf{four humours}. 
    \end{itemize}
    \textbf{Formal Theories:}
    \begin{itemize}
        \item \textbf{Freud's Theory of Psychanalysis} 
        \item \textbf{Humanists Theory}
    \end{itemize}
\end{itemize}

\subsection*{What is Freud's Theory of Personality?}
\begin{itemize}
    \item Freud was the first to recognize: \textbf{Abnormal Psychology} (anxiety and somatoform disorders), \textbf{Developmental Psychology} (childhood and adulthood psychology are different), and \textbf{Cognitive Psychology} (much of thought is beyond our conscious awareness)
    \item Freud proposed we have 3 components in our mind: \textbf{Id} (Pleasure), \textbf{Ego} (Reality), \textbf{Superego} (Morality). The Id and Ego both aim to satisfy biological urges, and the Superego aims to be good, and controls the Id and Ego through guilt. The Ego is in charge of decision making. This is called \textbf{Intrapsychic Conflict}. The Superego doesn't manifest until 3 to 5 years of age. The Super ego and ego both exist in the conscious, preconscious, and unconscious, but the id only exists in the unconscious mind. 
    \item In response to anxiety generated by unresolved intrapsychic conflict, freud believed we have a large repertoire of unconscious strategies called defence mechanisms: 
    \begin{itemize}
        \item \textbf{Repression}: tendency to keep distressing thoughts and feelings buried in the unconscious. 
        \item \textbf{Projection}: tendency to misattribute one's own thoughts and desires to others
        \item \textbf{Displacement}: tendency to redirect unacceptable impulses away from the original target and towards a substitute target instead. 
        \item \textbf{Sublimation}: tendency to redirect unacceptable impulses away from the original source and towards a more socially acceptable target instead. 
        \item \textbf{Reaction Formation}: tendency to behave in the opposite manner of one's true feelings. Some instances of homophobia and reverse discrimination are examples of this.
        \item \textbf{Regression}: tendency to revert back to immature patterns of behavious such as throwing a temper tantrum
        \item \textbf{Identification}: tendency to reinforce one's self esteem by forming alliances with another person or group
        \item \textbf{Rationalization}: tendency to invent faulty but plausible excuses to justify unacceptable choices
    \end{itemize}
    
    \item \textbf{Psychosexual Stages}\begin{itemize}
        \item \textbf{Oral}: The child primarily receives physical pleasure by way of nursing. 
        \item \textbf{Anal}: Physical pleasure is derived from the expulsion/retention of feces 
        \item \textbf{Phallic}: erotic focus shifts to the genitalia
        \item \textbf{Latency}: children begin to form friendships with others who are not kin
        \item \textbf{Genital}: erotic focus returns to the genitals, but accompanied with interest in sexual relationships. 
    \end{itemize}
\end{itemize}

\subsection*{What are the Carl Jung and Alfred Adler theories of personality?}
\begin{itemize}
    \item Jung proposed we have a \textbf{collective unconscious}, which is shared with all humankind. 
    \item Adler believed the central goal during development was \textbf{striving for superiority} (improve one-self and to master life's challenges). Children \textbf{compensate} by studying and practicing. Some individuals retain exaggerated feelings of inferiority, resulting in an \textbf{inferiority complex}, which Adler argued occurs when parents neglect or pamper their children. 
\end{itemize}

\subsection*{What is the Humanist theory of personality?}
\begin{itemize}
    \item Humanists believed that humans are unique in that they have \textbf{free will} and \textbf{desire for self-improvement}. 
    \item Carl Rogers calls the disparity between self-concept and experience \textbf{incongruence}. He believed that incongruence results from the belief that affection is \textbf{conditional}: that it must be earned. He uses this to explain anxiety. 
    \item Abraham Maslow believed that humans satisfy needs that are on a pyramid, and \textbf{self-actualization} is at the top.
\end{itemize}

\section*{Module 10: Behaviour in Groups}
\subsection*{What is the bystander effect?}
Due to \textbf{diffusion of responsibility}, the responsibility to respond in an emergency is spread equally among all those present. 


\subsection*{What is social loafing and social facilitation?}
\begin{itemize}
    \item \textbf{Social loafing} is when grouping people results in lower efforts invested by individual members. You can reduce social loafing through:
    \begin{itemize}
        \item \textbf{Task Importance}: emphasize task importance
        \item \textbf{Group Cohesion}: cultivate interpersonal relationships between group members
        \item \textbf{Collectivistic Orientation}: encourage collectivistic orientations rather than individualism
        \item \textbf{Specialization}: segment a large task into smaller components
    \end{itemize}
    \item \textbf{Social facilitation} is when presence of others can affect one's performance, for example playing music in front of a crowd. Robert Zajonc proposed this happens due to the fact that people experience an increase in arousal when being observed, which makes things that are practiced easier, and unpracticed things harder. Cockroaches also experience social facilitation. 
\end{itemize}


\subsection*{What is ingroup favoritism and intergroup conflict?}
\begin{itemize}
    \item \textbf{Reciprocal Altruism} or \textbf{Direct Social Exchange} describes cooperation where you are willing to do something for others, and in exchange they are willing to do it for you. 
    \item \textbf{Cheating} is defined as someone who accepts large benefits from others but refuses to pay any cost to return favours.
    \item People playing the prisoners dilemma cooperate more with ingroup members than outgroup members
    \item Conflicts between groups can occur due to competition for finite resources, and this can be solved by finding a superordinate goal
\end{itemize}

\subsection*{What is conformity and obedience?}
\begin{itemize}
    \item 80\% of participants conformed to the group judgement at least once, in an experiment with line lengths. 
    \item In the experiment with electric shocks, participants usually obeyed the experimenter, especially when they were closer to the authority figure. They usually displaced the blame to the authority figure. 
\end{itemize}

\section*{Module 11: Depression, Anxiety, and Schizophrenia}
\subsection*{What is the medical model?}
\begin{itemize}
    \item Mental disorders are defined by three characteristics: \textbf{Deviance} (thoughts and feelings that are unusual in the population or a particular context), \textbf{Distressful} (Subjective feeling that something is wrong), \textbf{Dysfunctional} (individual's ability to work and live is often measurably impaired)
    \item The \textbf{DSM} (Diagnostic and Statistical Manual of Mental Disorders) is ever improving, but can result in over diagnosis and diagnostic labelling. 
\end{itemize}

\subsection*{What is anxiety?}
\begin{itemize}
    \item Anxiety reflects an increase in sympathetic nervous system activity. Although unpleasant, it is useful because it triggers \textbf{Stress Responses}. \begin{itemize}
        \item \textbf{Fight or flight} pathway originates in the hypthalamus, sends neuronal signals to the adrenal gland by the way of the sympathetic division of the autonomic nervous system. Stimulation of the \textbf{Adrenal Medulla} leads to the release of adrenalin and noradrenalin. These dramatically increase immediate energy reserves. This pathway is fast and acute.
        \item \textbf{HPA stress response} originates in the hypothalamus, travels to the Adrenal Cortex by the way of the Pituitary Gland and endocrine system. Stimulation of this pathway secretes \textbf{Cortisol}, which increases the metabolic rate to boost energy. This pathway is slow.
        \item Anxiety Disorders are psychological conditions in which individuals experience excessive apprehension, anxiety, or fear. \begin{itemize}
            \item \textbf{GAD} (generalized anxiety disorder) is characterized by chronic, high-level anxiety. It is insidious and the anxiety eventually plateaus and remains at an elevated level. 
            \item \textbf{Panic Disorder} involves discreet, recurrent, sudden and unexpected attacks of overwhelming anxiety. It is acute. 
            \item \textbf{Phobic Disorder} involves persistent and irrational fear of an object or situation. It is similar to panic disorder, but the phobic response only occurs in the presence of a specific stimulus or situation. 
            \item \textbf{Social Anxiety Disorder} involves a persistent fear of one or more social performance situations. 
        \end{itemize}
    \end{itemize}
\end{itemize}

\subsection*{What are the obsessive compulsive disorder and post traumatic stress disorder?}
\begin{itemize}
    \item OCD used to be categorized as an anxiety disorder, it is now its own category. It is composed of two elements. \textbf{Obsessions} are persistent, uncontrollable intrusions of unwanted thoughts. \textbf{Compulsions} are ritualistic behaviours that relieve anxiety. 
    \item PTSD has a few symptoms: \begin{itemize}
        \item The event being re-experienced in flashbacks or nightmares 
        \item Avoidance of cues related to the event
        \item Numbing of general responsiveness 
        \item Increased autonomic arousal
    \end{itemize}
\end{itemize}


\subsection*{Depression, Bipolar Disorder, and Schizophrenia}
\begin{itemize}
    \item Low mood helps us avoid further social injury. Depression is not merely low mood, but a unipolar mood disorder, characterized by persistent and dramatically reduced mood. 
    \item \textbf{Suicidal Ideation} involves four stages:\begin{enumerate}[label=Stage \arabic*)]
        \item Thoughts of death but no intention
        \item Contemplate suicide but no plan 
        \item Develop plan
        \item Attempt suicide
    \end{enumerate}
    \item \textbf{Clean Drugs} are drugs that affec the targeted neurotransmitter system exclusively without affecting other systems. \textbf{Dirty Drugs} are drugs that affect multiple neurotransmitter systems. 
    \item \textbf{Bipolar Disorder}, or manic-depression, describes abnormal deviations to mood in both directions.
    \item \textbf{Schizophrenia} is a heterogeneous collection of psychotic disorders characterized by severe disruptions to normal thought processes, perception, behaviour, and emotion. It has both negative symptoms and positive symptoms. \begin{itemize}
        \item Negative Symptoms \begin{itemize}
            \item Avolition (motivational deficit)
            \item Alogia (deficits of verbal communication)
            \item Anhedonia (deficits of pleasure)
            \item Flat or Inappropriate affect (emotional deficits)
            \item Asociality (deficits in social relationships)
            \item Cateleptic Stupor (deficits in movement)
            \item Negativism (deficits in responding)
        \end{itemize}
        \item Positive Symptoms \begin{itemize}
            \item Delusions (disruption to thought)
            \item Hallucinations (disruption to perception)
            \item Hyperactivity (excessive movement)
            \item Neologisms (individual invents new words)
            \item Word Salad (individual's language is incoherent)
            \item Clang Speech (disruption to thought/speech)
        \end{itemize} 
        Indications of a good prognosis include \begin{itemize}
            \item More positive and fewer negative symptoms 
            \item Later onset 
            \item rapid descent into the first psychotic episode rather than a slow, gradual descent 
            \item good adjustment prior to the first psychotic episode 
        \end{itemize}
        \item To get schizophrenia, you must both have some of 100 genes, and be exposed to the right environmental component. 
        \item First generation antipsychotics worked in reducing positive symptoms, and second generation antipsychotics reduce negative symptoms in addition to positive symptoms. 
    \end{itemize}
\end{itemize}

\section*{Module 12: Psychological Therapy}
\subsection*{What is psychoanalysis?}
\begin{itemize}
    \item \textbf{Dream Analysis} is when the patient shares their dream, and the psychoanalyst analyzes it. 
    \item \textbf{Free association} is when the patient expresses their thoughts and feelings spontaneously
    \item The psychoanalysis must provide interpretation, the inner significance of the thoughts of the patient. 
    \item The psychoanalysis is working when the patient shows resistance or transference. 
\end{itemize}
\subsection*{What is client-centred therapy?}
\begin{itemize}
    \item As opposed to psychoanalysis therapy, client-centred therapy focuses on creating a therapeutic climate that is supportive, and gives the client unconditional positive regard, genuineness, and empathy. 
    \item Instead of interpreting or providing advice, the therapist is to provide support and clarification
\end{itemize}
\subsection*{What is cognitive therapy?}
\begin{itemize}
    \item Cognitive therapy locates the problem in people's habitual thought patterns. It focuses on reorganizing automated thought patterns. Useful for treating depression. It is often combined with behavioural therapies. 
\end{itemize}
\subsection*{What are behavioural therapies?}
\begin{itemize}
    \item Aims to address maladaptive behaviour rather than the underlying problem that led to the behaviour. 
    \item \textbf{Exposure Therapies} refer to a family of different techniques used to expose the individual to fear provoking stimuli without the possibility of harm. 
    \item \textbf{Systematic Desensitization} has three steps: \begin{enumerate}[label=Step \arabic*)]
        \item train the individual in techniques that enable them to self-induce a state of deep muscle relaxation. Then, build an \textbf{Anxiety Hierarchy}, which is a list of fear invoking stimuli, ordered. 
        \item imagine each scenario on the hierarchy beginning with the least anxiety provoking. 
            \item graduate from imagining these scenarios to actually encountering them, again using the relaxation training. 
    \end{enumerate}
    \item \textbf{Aversion Therapy} does a simlar thing, but instead of replacing an aversive response with a desirable response, it replaces a desirable response with an aversive response. One common method is to use an \textbf{Emetic Drug}, which induces nausea and vomiting. 
    \item \textbf{Social Skills Training} strives to improve interpersonal proficiency in individuals. It applys three principles: \begin{itemize}
        \item Modeling: individual observes socially skilled peers and colleagues while the therapist draws attention to specific elements of the model's behaviour 
        \item Behavioural Rehearsal: the individual then implements and practices these social skills 
        \item Shaping: the therapy gradually increases the complexity of the social situation. 
    \end{itemize}
\end{itemize}

\end{document}