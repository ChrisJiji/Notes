\documentclass[10pt,english]{article}
\usepackage[T1]{fontenc}
\usepackage[latin9]{inputenc}
\usepackage{geometry}
\geometry{verbose,tmargin=1.5in,bmargin=1.5in,lmargin=1.5in,rmargin=1.5in}
\usepackage{amsthm}
\usepackage{amsmath}
\usepackage{amssymb}

\makeatletter
\usepackage{enumitem}
\newlength{\lyxlabelwidth}

\usepackage[T1]{fontenc}
\usepackage{ae,aecompl}

%\usepackage{txfonts}

\usepackage{microtype}

\usepackage{calc}
\usepackage{enumitem}
\setenumerate{leftmargin=!,labelindent=0pt,itemindent=0em,labelwidth=\widthof{\ref{last-item}}}

\makeatother

\usepackage{babel}
\begin{document}
\noindent \begin{center}
\textbf{\large{}MATH 146 - Assignment 1}\\
\textbf{\large{}Chris Ji 20725415}
\par\end{center}{\large \par}
\medskip{}

\begin{enumerate}
\item They are equal when $\cos(x)=\sin(2x)\Rightarrow \cos(x)=2\sin(x)\cos(x)\Rightarrow 1=2\sin(x)\Rightarrow x=\sin^{-1}\left(\frac{1}{2}\right)\Rightarrow x=\frac{\pi}{6}$. Then the area between this curve, $\int_0^{\frac{\pi}{2}}|\cos(x)-\sin(2x)|dx=\int_0^{\frac{\pi}{6}}\cos(x)-\sin(2x)dx+\int_{\frac{\pi}{6}}^\frac{\pi}{2}\sin(2x)-\cos(x)dx$. \\ 
$\int_0^\frac{\pi}{6}\cos(x)-\sin(2x)dx=\sin(x)-\int_0^\frac{\pi}{6}\frac{1}{2}\sin(u)du$ (setting $u=2x\Rightarrow dx=\frac{1}{2}du$). The integral equals $-\frac{1}{2}\cos(u)=-\frac{1}{2}(-\cos(2x))$, so $\int_0^\frac{\pi}{6}(\cos(x)-\sin(2x))dx=(\sin(\frac{\pi}{6})+\frac{1}{2}\cos(2\pi/6))-(\sin(0)+\frac{1}{2}\cos(0))=\frac{1}{4}$. \\ 
$\int_\frac{\pi}{6}^\frac{\pi}{2}\sin(2x)-\cos(x)dx=\int_\frac{\pi}{6}^\frac{\pi}{2}\frac{1}{2}\sin(u)du-\sin(x)$ (setting $u=2x\Rightarrow dx=\frac{1}{2}du$). The integral equals $\frac{1}{2}\cos(u)=\frac{1}{2}(-\cos(2x))$, so $\int_\frac{\pi}{6}^\frac{\pi}{2}(\sin(2x)-\cos(x))dx=(-\frac{cos(2\frac{\pi}{2})}{2}-\sin(\pi/2))-(-\frac{\cos(2\pi/6)}{2}-\sin(\pi/6))=\frac{1}{4}$

\pagebreak
\item \begin{enumerate}
    \item $y=\sqrt{x}=x^2\Rightarrow x=x^4\Rightarrow x=0,1$ (intersection points). Then we can find the area of this solid by $\int_0^1\pi((\sqrt{x})^2-(x^2)^2)=(\pi\int xdx-\pi\int x^4dx)|_0^1=(\pi\frac{x^2}{2}-\pi\frac{x^5}{5})|_0^1=\pi(\frac{1}{2}-\frac{1}{5})-\pi(\frac{0}{2}-\frac{0}{5})=\pi\frac{3}{10}$
    
    \item From above, we know that $\sqrt{x}$ and $x^2$ intersect at $0$ and $1$. Then finding this volume using the shell method we can evaluate the integral $\int_0^12x\pi(\sqrt{x}-x^2)dx=2\pi\left(\int_0^1x\cdot\sqrt{x}dx-\int_0^1x\cdot x^2dx\right)=2\pi(\frac{2x^\frac{5}{2}}{5}-\frac{x^4}{4})|_0^1=\frac{3\pi}{10}$
\end{enumerate}
\pagebreak
\item \begin{enumerate}
    \item MAPLE
    \item MAPLE
    \item If $y(0)=1$, then $y=e^{\frac{x^2}{2}}+1$
    \item 
    \item Note that this is a separable DE, with $f(x)=x$, $g(y)=y-1$. Then we can solve $\int\frac{1}{g(y)}dy=\int f(x)dx\Rightarrow \int \frac{1}{y-1}dy=\int xdx\Rightarrow \ln(|y-1|)+C_1=\frac{x^2}{2}+C_2\Rightarrow y=e^{\frac{x^2}{2}+C}+1$
\end{enumerate}
\pagebreak
\item \begin{enumerate}
    \item Note that the rate of salt into the tank is $0.05\text{kg/liter}\times 8\text{litres/min}+0.04\times 5=0.6\text{kg/min}$, and the rate out is $\frac{s(t)=\text{salt in tank}}{1000}\times 13$. Then $s'(t)=0.6-\frac{s(t)}{1000}kg/min$. (note $s(0)=0$)Then this is a first order linear DE with integrating factor $I(t)=e^{\int\frac{-1}{1000}dt}=e^\frac{t}{1000}$. Then we get $s(t)=\frac{\int 0.6e^\frac{t}{1000}dt}{e^\frac{t}{1000}}=\frac{600e^\frac{t}{1000}+C}{e^\frac{t}{1000}}=600+Ce^\frac{-t}{1000}$. But we know $s(0)=0$, so $0=600+Ce^\frac{-t}{1000}\Rightarrow C=-600$. So there is $s(t)=600-600e^\frac{-t}{1000}$ kg of salt in the tank after $t$ minutes.
    \item $s(30)=600-600e^\frac{-30}{1000}\approx 17.733$kg of salt after half an hour.
    \item We have $\lim_{t\rightarrow\infty}600-600e^\frac{-t}{1000}=600$, as expected in a tank with 0.6kg/L, and 1000L.  
\end{enumerate}

\pagebreak
\item \begin{enumerate}
    \item Note that since $C(t)$ decreases at a rate proportional to $t$, then it can be expressed as $C(t)=Ae^{kt}$, for some constant $A$. If $C(0)=70$, then clearly $A=70$. Then $C(t)=70e^{kt}$. 
    \item If the body eliminates half the drug in 30 hours, then $70e^{k(30)}=\frac{1}{2}70\Rightarrow e^{30k}=\frac{1}{2}\Rightarrow 30k=\ln\left(\frac{1}{2}\right)$, then $k=\frac{1}{30}\ln\left(\frac{1}{2}\right)$. So if we want to find how long it takes to remove 80\%of the drug, we need to find $C(t)=0.2*70=14=70e^{\frac{1}{30}\ln\left(\frac{1}{2}\right)t}\Rightarrow \frac{1}{5}=e^{\frac{1}{30}\ln\left(\frac{1}{2}\right)t}\Rightarrow t=\frac{30\ln(5)}{\ln(2)}\approx 69.658$ hours. 
\end{enumerate}

\pagebreak
\item \begin{enumerate}
    \item We are given $P'=0.01P-0.002P^2=0.0002P(50-P)$, so this follows a logistic growth model of the form $P'=kP(M-P)$, where $k=0.01,M=50$, so the carrying capacity is 50 (since $P$ is measured in 1000s, it's actually 50000).
    \item 50, <50, >50
    \item 
    \item $P>M$: decays to 60. $0<P<M$: grows to 60. $P=0$: stays at 0. As $t\rightarrow\infty$: approaches 50.
\end{enumerate}

\pagebreak
\item \begin{enumerate}
    \item We have $T_0=38$, and $T_a=0$. If $T(20)=20$, then $T(20)=(38-0)e^{20k}+0\Rightarrow 20=38e^{20k}\Rightarrow k=\frac{\left(\ln\frac{10}{19}\right)}{20}$. Then the function $T(t)$, assuming Newton's law of cooling, is $T(t)=38e^{\frac{\ln\left(\frac{10}{19}\right)t}{20}}$
    
    \item $T(t)=10=38e^{\frac{\ln\left(\frac{10}{19}\right)}{20}t}\Rightarrow \ln\frac{10}{38}=\frac{\ln\frac{20}{38}t}{20}\Rightarrow t=\frac{20\ln(\frac{5}{19})}{\ln\frac{10}{19}}\approx41.598$ minutes. 
    
    \item $T(15)=38e^{\frac{\ln\left(\frac{10}{19}\right)15}{20}}=38e^{\frac{3\ln(\frac{10}{19})}{4}}\approx23.481^\circ C$
    
\end{enumerate}

\pagebreak
\item \begin{enumerate}
    \item Note that $f'(x)=\frac{d(x^2)}{dx}+\frac{d\left(\int_0^xtf(t)dt\right)}{dx}\Rightarrow f'(x)=\frac{x^3}{3}+xf(x)$. Then $y'=2x+xy\Rightarrow \frac{x^3}{3}+xf(x)=2x+x(x^2+\int_0^xtf(t)dt)=2x+x^3+x\int_0^xtf(t)dt$
    \item Note that this linear DE has integrating factor $I(x)=e^{-\int xdx}=e^{-\frac{x^2}{2}}$. Then the solution is $y=\frac{\int2xe^{-\frac{x^2}{2}}dx}{e^{-\frac{x^2}{2}}}\Rightarrow y=e^{\frac{x^2}{2}+C}-2$, but since $y(0)=0$, then $C=\ln(2)$, so we can simply to $y=2e^\frac{x^2}{2}-2$. Then if $y=2e^\frac{x^2}{2}-2$
\end{enumerate}

\end{enumerate}

\end{document}
