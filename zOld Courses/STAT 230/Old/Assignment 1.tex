\documentclass[10pt,english]{article}
\usepackage[T1]{fontenc}
\usepackage[latin9]{inputenc}
\usepackage{geometry}
\geometry{verbose,tmargin=1.5in,bmargin=1.5in,lmargin=1.5in,rmargin=1.5in}
\usepackage{amsthm}
\usepackage{amsmath}
\usepackage{amssymb}

\makeatletter
\usepackage{enumitem}
\newlength{\lyxlabelwidth}

\usepackage[T1]{fontenc}
\usepackage{ae,aecompl}

%\usepackage{txfonts}

\usepackage{microtype}

\usepackage{calc}
\usepackage{enumitem}
\setenumerate{leftmargin=!,labelindent=0pt,itemindent=0em,labelwidth=\widthof{\ref{last-item}}}

\makeatother

\usepackage{babel}
\begin{document}
\noindent \begin{center}
\textbf{\large{}STAT 230 - Assignment 1}\\
\textbf{\large{}Chris Ji 20725415}
\par\end{center}{\large \par}
\medskip{}

\begin{enumerate}
\item\begin{enumerate}
    \item [a.] The sample space is $6!=720$ points large. It contains all unique 6 digit numbers containing the numbers from 1,$\ldots$, 6.
    \item [b.] The probability that the first digit is 1 and the last digit is 6 is $\frac{4!}{6!}=\frac{1}{30}\approx3.33\%$. This is because there are $4!$ ways to orient the numbers $\{2\ldots5\}$ in between $1$ and $6$. 
    \item [c.] The probability that the odd numbers occur side by side are $\frac{3!*3!+3!*3!+3!*3!}{6!}=\frac{3}{20}=15\%$. This is because there are $3!$ permutations of the 3 odd numbers, and when they are in a row, there are $3!$ ways to permute the even numbers. Furthermore, they can occur in a row 3 places in the row of 6 numbers.
\end{enumerate}

\pagebreak

\item\begin{enumerate}
    \item [a.] The sample space can be defined by the amount of prizes won: $S=\{0,1,2,3\}$
    \item [b.] The probability that the $3rd$ prize is the only prize is $\frac{9997}{10000}\cdot\frac{9997}{10000}\cdot\frac{3}{10000}=\frac{299820027}{10000^3}\approx0.03\%$. This is because the chances of not winning the first two are $\frac{9997}{10000}$ each, and the chances of winning the third is $\frac{3}{10000}$. 
    \item [c.] The probability of winning at least one prize is $\frac{3+3+3}{10000}=\frac{9}{10000}\approx 0.09\%$. This is because each draw there is a $\frac{3}{10000}$ chance of winning (since I have 3 tickets), and there are 3 draws. 
\end{enumerate}

\pagebreak

\item Since $P(A)=0.3$, $P(B)=0.45$, and $P(A\cap B)=0.1$, then $P(A\cup B)=0.3+0.45-0.1=0.65 = P(D)$. Therefore, $P(\bar D)=1-0.65 = 0.35$.  Since B is independent of C, $P(B\cap C)=P(B)P(C)=0.225$. By De Morgan's laws, $\bar D=\overline{A\cup B}=\bar A\cap \bar B$. Then $P(C|\bar D)=\frac{P(C\cap\bar D)}{P(\bar D)}=\frac{P(C\cap\bar A\cap\bar B)}{P(\bar D)}$. Since $C$ and $A$ are mutually exclusive, $P(C\cap \bar A)=P(C)$, and since $C$ and $B$ are independent, then $P(C\cap\bar B)=P(C)P(\bar B)= 0.5*0.55 = 0.275$. So $P(C|\bar D)=\frac{0.275}{0.35}=0.65$.  
\newpage

\item\begin{enumerate}
    \item [a.] Since $\sum_{\text{all }x}f(x)=1$, and $f(x)$ only takes positive integer values greater than 2, then it is much simpler to calculate $1-(f(2)+f(3)+f(4))$. $f(2)=\frac{1}{2}$, $f(3)=\frac{1}{6}$, $f(4)=\frac{1}{12}$, and so $f(2)+f(3)+f(4)=0.75$, so $P(X\geq5)=1-0.75=0.25$. 
    \item [b.] The cumulative probability function $F(x)$ describes $P(X\leq x)$. Therefore, $F(X)=f(x)+f(x-1)+f(x-2)+\ldots+f(2)$. Let's try a few values of x. $F(2)=f(2)=\frac{1}{2}$, $F(3)=f(3)+f(2)=\frac{1}{6}+\frac{1}{2}=\frac{2}{3}$, $F(4)=f(4)+f(3)+f(2)=\frac{1}{12}+\frac{1}{6}+\frac{1}{2}=\frac{3}{4}$, and $F(5)=f(5)+f(4)+f(3)+f(2)=\frac{1}{20}+\frac{1}{12}+\frac{1}{6}+\frac{1}{2}=\frac{4}{5}$. It is clear from this that $F(x)=\frac{x-1}{x}$. Of course, as with $f(x)$, $F(x)$ only takes in integer values greater than or equal to 2.
\end{enumerate}

\pagebreak

\item\begin{enumerate}
    \item [a.] A possible sample space is just every possible combination of every name assigned to every hash. Because there can be more than one hash value assigned to each key, we must multiply the amount of combinations of the hash values $(5^3)$ by the amount of permutations of the keys $(3!)$. Therefore the size of the sample space is $5^33!=750$.
    \item [b.] The probability that the hash value 01 is not assigned to anyone is $\frac{4}{5}\cdot\frac{4}{5}\cdot\frac{4}{5}=\frac{64}{125}\approx0.512$. This is because at each assignment of hash value, there is a $\frac{4}{5}$ chance of not assigning the hash value 01, and there are 3 keys. 
    \item [c.] The probability of at least one collision is $\frac{3}{5}$. This is because when the first hash value is "picked", there is 0 chance of a collision. When the second is "picked", there is a $\frac{1}{5}$ chance of a collision. When the third hash is "picked" (assuming the second did not result in a collision), there is a $\frac{2}{5}$ chance of collision. For $k$ keys and $n$ hashes, it follows that the chance of at least once collision is $\frac{\sum_{i=1}^{k-1}i}{n}$.
    \item [d.] The probability that at least one of the hash values 01, 02 have been assigned is $\frac{2}{5}+\frac{2}{5}+\frac{2}{5}=\frac{6}{5}$. This is because at each assignment of hash value, there is a $\frac{2}{5}$ chance of assigning either 01, or 02, and there are 3 keys.
\end{enumerate}

\pagebreak

\item 
\end{enumerate}

\end{document}
