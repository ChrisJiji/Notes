\documentclass[10pt,letter]{article}
\usepackage{graphicx}
\usepackage{setspace}
\usepackage{xcolor}
\newcommand\tab[1][5mm]{\hspace*{#1}}
\onehalfspacing
\usepackage{fullpage}
\begin{document}
\definecolor{teal}{RGB}{0,139,139}

\paragraph*{Booleans} 
Any value that isn't \textcolor{blue}{\#f} is considered true. 

\paragraph*{Logical Operators}
\textcolor{teal}{and} produces \textcolor{blue}{\#f} if any of the arguments are \textcolor{blue}{\#f}, \textcolor{blue}{\#t} if there are no arguments, otherwise it produces the last argument. \textcolor{teal}{or} produces either \textcolor{blue}{\#f} or the first non-false argument. 

\paragraph*{Structures}
\begin{itemize}
    \item \textcolor{teal}{struct} can be used instead of \textcolor{teal}{define-struct}
    \item the make- prefix can be omitted (\textcolor{blue}{posn} instead of make-posn)
\end{itemize}
For now, you should include \textcolor{blue}{\#:transparent} in your \textcolor{teal}{struct} definitions (this is discussed in Section 02). \\ e.g. 
(\textcolor{teal}{struct} posn (x y) \textcolor{blue}{\#:transparent})

\paragraph*{member}
(member v lst) produces \textcolor{blue}{\#f} if v does not exist in lst. If v does exist in lst, it produces the tail of lst, starting with the first occurrence of v.

\paragraph*{Implicit local}
You do not need to explicitly use \textcolor{teal}{local}, e.g.  \\ 
(\textcolor{teal}{define} (t-area a b c)\\
\tab(\textcolor{teal}{define}\\ 
\tab\tab{[}(\textcolor{teal}{define} s (/ (+ a b c) 2)){]}\\ 
\tab\tab(sqrt (* s (- s a) (- s b) (- s c)))))\\ 

Is equivalent to:\\ 
(\textcolor{teal}{define} (t-area a b c)\\ 
\tab(\textcolor{teal}{define} s (/ (+ a b c) 2))\\ 
\tab(sqrt (* s (- s a) (- s b) (- s c))))\\ 

\paragraph*{check-expect} More advanced testing methods will be introduced in Section 07, until then, equal? can be used. 

\paragraph*{Racket Modules} \textcolor{teal}{provide} is used in a module to specify the identifiers that are available to clients. \textcolor{teal}{require} is used to identify the module that the current file depends on. 

\paragraph*{C Modules} C uses \#include to include any files needed. 

\paragraph*




\end{document}