\documentclass[10pt,letter]{article}

\usepackage{amsmath}
\usepackage{amssymb}
\usepackage{graphicx}
\usepackage{setspace}
\onehalfspacing
\usepackage{fullpage}
\begin{document}

\title{MATH145 Assignment 1}
\author{Chris Ji 20725415}

\maketitle 

\section*{Problem 1}

\paragraph{(a)} 
Let F = (((P $\vee \neg$Q) $\leftrightarrow$ R) $\vee$ (R $\rightarrow$ ($\neg$P $\leftrightarrow$ Q)))
\begin {center}
\begin{tabular}{|l|l|l|l|l|l|l|l|l|l|}
\hline
P & Q & R & $\neg$Q & P $\vee \neg$Q & (P $\vee \neg$Q) $\leftrightarrow$ R & $\neg$P & $\neg$P $\leftrightarrow$ Q & R $\rightarrow$ ($\neg$P $\leftrightarrow$ Q) & F \\ \hline
1 & 1 & 1 & 0 & 1 & 1 & 0 & 0 & 0 & 1 \\
1 & 1 & 0 & 0 & 1 & 0 & 0 & 0 & 1 & 1 \\
1 & 0 & 1 & 1 & 1 & 1 & 0 & 1 & 1 & 1 \\
1 & 0 & 0 & 1 & 1 & 0 & 0 & 1 & 1 & 1 \\
0 & 1 & 1 & 0 & 0 & 0 & 1 & 1 & 1 & 1 \\
0 & 1 & 0 & 0 & 0 & 1 & 1 & 1 & 1 & 1 \\
0 & 0 & 1 & 1 & 1 & 1 & 1 & 0 & 0 & 1 \\
0 & 0 & 0 & 1 & 1 & 0 & 1 & 0 & 1 & 1 \\
\hline
\end{tabular}
\end{center}

Therefore, $\models$ F.

\paragraph{(b)} 
Let G = ((P $\leftrightarrow \neg$R) $\rightarrow$ (Q $\wedge$ R)) and H = ((P $\rightarrow$ Q) $\leftrightarrow$ (Q $\leftrightarrow$ R))

\begin {center}
\begin{tabular}{|l|l|l|l|l|l|l|l|l|l|}
\hline
P & Q & R & $\neg$R & P $\leftrightarrow \neg$R & Q $\wedge$ R & G & P $\rightarrow$ Q & Q $\leftrightarrow$ R & H \\ \hline
1 & 1 & 1 & 0 & 0 & 1 & 1 & 1 & 1 & 1 \\
1 & 1 & 0 & 1 & 1 & 0 & 0 & 1 & 0 & 0 \\
1 & 0 & 1 & 0 & 0 & 0 & 1 & 0 & 0 & 1 \\
1 & 0 & 0 & 1 & 1 & 0 & 0 & 0 & 1 & 0 \\
0 & 1 & 1 & 0 & 1 & 1 & 1 & 1 & 1 & 1 \\
0 & 1 & 0 & 1 & 0 & 0 & 1 & 1 & 0 & 0 \\
0 & 0 & 1 & 0 & 1 & 0 & 0 & 1 & 0 & 0 \\
0 & 0 & 0 & 1 & 0 & 0 & 1 & 1 & 1 & 1 \\
\hline
\end{tabular}
\end{center}

The G column is not the same as the H column, so G $\not\equiv$ H.

\paragraph{(c)} 

Let I = (P $\leftrightarrow$ (Q $\wedge$ $\neg$R)), J = (R $\rightarrow$ (P $\wedge \neg$Q)), and K = $\neg$((P $\leftrightarrow$ Q) $\rightarrow$ R).

\begin{center}
\begin{tabular}{|l|l|l|l|l|l|l|l|l|l|l|l|}
\hline
P & Q & R & $\neg$R & Q $\wedge$ $\neg$R & I & $\neg$Q & P $\wedge$ $\neg$Q & J & P $\leftrightarrow$ Q & (P $\leftrightarrow$ Q) $\rightarrow$ R & K \\ \hline
1 & 1 & 1 & 0 & 0 & 0 & 0 & 0 & 0 & 1 & 1 & 0 \\
1 & 1 & 0 & 1 & 1 & 1 & 0 & 0 & 1 & 1 & 0 & 1 \\
1 & 0 & 1 & 0 & 0 & 0 & 1 & 1 & 1 & 0 & 1 & 0 \\
1 & 0 & 0 & 1 & 0 & 0 & 1 & 1 & 1 & 0 & 1 & 0 \\
0 & 1 & 1 & 0 & 0 & 1 & 0 & 0 & 0 & 0 & 1 & 0 \\
0 & 1 & 0 & 1 & 1 & 0 & 0 & 0 & 1 & 0 & 1 & 0 \\
0 & 0 & 1 & 0 & 0 & 1 & 1 & 0 & 0 & 1 & 1 & 0 \\
0 & 0 & 0 & 1 & 0 & 1 & 1 & 0 & 1 & 1 & 0 & 1 \\ 
\hline
\end{tabular}
\end {center}

I, J $\models$ K.

\section*{Problem 2}
\paragraph{(a)}
$\forall x(0 < x\rightarrow$ $\exists y,z$ $y>z$ $(y * y = x \wedge z * z = x)$

\paragraph{(b)}
$0 < u \wedge \forall y((1 < y \wedge \exists z$ $x = y * y * z) \rightarrow \exists z$ $y = z + z + z)$

\paragraph{(c)}
$\forall u(u \in w)$ $\exists x(x\in u \rightarrow \exists y\exists z\forall a((y \in u \wedge (y \in z \wedge a \in \neg z))\leftrightarrow y \in w)$\\
Assume w contains some element b, then $y = b$ ($y \in w$). By V30, $y \in u$, and so $b \in u$, a contradiction because this statement should be true $\forall u$. 


\section*{Problem 3}

\paragraph{(a)} $$\exists z(2z = 2y + 1)$$
$$\exists z(z = y + \frac{1}{2})$$
$\frac{1}{2}$ does not exist in $\mathbb{Z}$, but does exist in $\mathbb{R}$, so F is false in $\mathbb{Z}$, and true in $\mathbb{R}$.
\paragraph{(b)} Rearranging the left side of the implication, we get $y = \frac{1-z}{2z}$, meaning that y:z is a bijection, and $\nexists y \forall z$, and $\nexists z \forall y$. Because the first part of the implication is false, F is true by V21.
\paragraph{(c)} For the class of all sets, $\exists y (x\in y \wedge y\in u) \equiv x \in u$. Then the right side of the or statement simplifies to $\forall x (x \in u)$. $\exists u (\exists x$ $x \in u \wedge\forall x (x \in u)) \equiv \exists u(\forall x$ $x\in u)$. This is not true for the class of all sets, as u would have to be the class of all sets to contain all x, and $u \notin u$. 


\section*{Problem 4}

\paragraph{(a)} Suppose that $u \in u$. Then $u = \{u\}$. By the equality axiom, this can't be true, as any element in $u$ can't be in $\{u\}$, as $\{u\}$ only contains $u$.

\paragraph{(b)} Suppose that $u \in v$. Then at minimum, $v = \{u\}$, which we proved in (a) would imply $v \notin u$. Therefore $\nexists u,v$ such that $u \in v$ and $v \in u$.

\paragraph{(c)} Suppose that $u \cup \{u\} = v \cup \{v\}$. Then there exists an arbitrary element $x$ such that $x \in u$ or $x \in \{u\} =  x \in v$ or $x \in \{v\}$. Because $x \in u$ and $x \in v$, by the extension axiom, $u = v$. 

\paragraph{(d)} Suppose that $\{u, \{u, v\}\} = \{x, \{x, y\}\}$, then by the extension axiom each element is the same, so $u = x$, and $\{u, v\} = \{x, y\}$. Again by the extension axiom, $u = x$, and $v = y$.


\section*{Problem 5}

\paragraph{(a)} 
1. $\{(P\rightarrow Q)\rightarrow R, P\rightarrow (Q \wedge\neg R)\} \models \neg(\neg P\wedge R)$ \hfill V5\\
2. $\{(P\rightarrow Q)\rightarrow R, P\rightarrow (Q \wedge\neg R)\} \models P$ \hfill V11 on line 1\\
3. $\{(P\rightarrow Q)\rightarrow R, P\rightarrow (Q \wedge\neg R)\} \models \neg R$ \hfill V12 on line 1\\
4. $\{(P\rightarrow Q)\rightarrow R, P\rightarrow (Q \wedge\neg R)\} \models P \rightarrow (Q \wedge \neg R)$ \hfill V1\\
5. $\{(P\rightarrow Q)\rightarrow R, P\rightarrow (Q \wedge\neg R)\} \models Q \wedge\neg R$ \hfill V23 on line 4\\
6. $\{(P\rightarrow Q)\rightarrow R, P\rightarrow (Q \wedge\neg R)\} \models Q$ \hfill V11 on lines 3 and 5\\
7. $\{(P\rightarrow Q)\rightarrow R, P\rightarrow (Q \wedge\neg R)\} \models (P\rightarrow Q)\rightarrow R$ \hfill V1\\
8. $\{(P\rightarrow Q)\rightarrow R, P\rightarrow (Q \wedge\neg R)\} \models R$ \hfill V23 on lines 2, 6, and 7\\ 

Lines 3 and 8 contradict, so $(P\rightarrow Q)\rightarrow R, P\rightarrow (Q \wedge\neg R) \models \neg P\wedge R$

\paragraph{(b)} $\models\exists x(f(x)Rf(x)\rightarrow xRx)$ \hfill V1\\
$\models\exists x\exists y(yRf(x)\rightarrow xRx)$ \hfill V40\\

\paragraph{(c)} 1. $\forall x\forall y(xRf(y) \leftrightarrow y = x) \models \forall x\forall y(y=x)$\hfill V4\\ 
2. $\forall x\forall y(f(y) = f(x) \rightarrow y = x) \equiv \forall x\forall y(\neg(f(y) = f(x)) \vee y = x)$ \hfill E20 on 1\\ 
3. $\models \forall x\forall y(\neg(f(y) = f(x)) \vee y = x)$ \hfill V12 on 1 and 2\\ 


\end{document}