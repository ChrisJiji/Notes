\documentclass[10pt,letter]{article}
\usepackage{amsmath}
\usepackage{amssymb}
\usepackage{amsthm}
\usepackage{graphicx}
\usepackage{setspace}
\onehalfspacing
\usepackage{fullpage}
\newtheorem*{remark}{Remark}
%\date{\vspace{-5ex}}
\begin{document}

\title{MATH145 Theorems and Definitions}
\author{Chris Ji 20725415}

\section*{2.52: Binomial coefficient}
${n\choose k} = \frac{n!}{k!(n-k)!}$

\section*{2.53: Pascal's Triangle}
${n\choose 0} = {n\choose n} = 1$, ${n \choose k} = {n\choose n-k}$, ${n \choose k} + {n \choose k + 1} = {n+1\choose k+1}$

\section*{2.55: Binomial Theorem}
$(a + b)^{n} = \sum_{k=0}^{n}{n \choose k}a^{k}b^{n-k} = \sum_{k=0}^{n}{n \choose k}a^{n-k}b^{n}$

\section*{3.2: Sum and product of Complex Numbers}
$(a + ib) + (c + id) = {a\choose b} + {c \choose d} = {a+c \choose b+d} = (a+c) + i(b+d)$\\ 
$(a+ib)(c+id) = ac + iad + ibc + i^2bd = (ac - bd) + i(ad + bc)$\\
This requires the product to be commutative and associative and distributive over the sum.

\section*{3.8-3.9: Definitions of Complex numbers}
\textbf{conjugate} if $z = x + iy$, $\bar{z} = x - iy$\\ 
\textbf{length/magnitude}$|z| = \sqrt{x^2+y^2}$\\
\textbf{various identities} $\bar{\bar{z}} = z$\\
$z + \bar{z} = 2Re z$, $z - \bar{z} = 2iIm z$\\ 
$z\bar{z} = |z|^2$, $|\bar{z}| = |z|$\\ 
$\bar{z+w} = \bar{z} + \bar{w}$, $\bar{zw} = \bar{z}\bar{w}$, $|zw| = |z||w|$ 

\section*{3.13: Square Roots of Complex Numbers}
When $z = x + iy$, $\sqrt{z} = \pm (\sqrt{\frac{x+\sqrt{x^2+y^2}}{2}} - i\sqrt{\frac{-x+\sqrt{x^2+y^2}}{2}})$

\section*{3.18-3.20: Polar Form Definition}
The angle (or argument) $z = |z|(\cos\theta(z) + i\sin\theta(z))$\\
If $z \neq 0$ and we have $x = Re(z)$, $y = Im(z)$, $r = |z|$ and $\theta = \theta(z)$, then $x = r\cos\theta$, $y = r\sin\theta$. \\ 
$r = \sqrt{x^2+y^2}$, $tan\theta = \frac{y}{x}$\\ 
$z = re^{i\theta}$, $\bar{z} = re^{-i\theta}$, $z^-1 = \frac{1}{r}e^{-i\theta}$

\section*{5.2: Divisor Theorems}
$a|0$ for all a $\in \mathbb{Z}$ and $0|a \Longleftrightarrow a = 0$\\
$a|1 \Longleftrightarrow a = \pm 1$ and $1 | a$ for all $a \in \mathbb{Z}$\\ 
If $a|b$ and $b|c$ then $a|c$\\ 
If $a|b$ and $b|a$ then $b = \pm a$\\ 
If $a|b$ and $a|c$ then $a|(bx+cy)$ for all $x$, $y \in \mathbb{Z}$

\section*{5.3: Division Algorithm}
Let a, b $\mathbb{Z}$ with $b \neq 0$. Then there exist unique integers q and r such that $a = qb + r$ and $0 \leq r < |b|$. The integers q and r are called the quotient and remainder when a is divided by b. 


\section*{5.5: Common divisor}
Let a, b $\in \mathbb{Z}$. A common divisor of a and b is an integer d such that $d|a$ and $d|b$. 

\section*{5.18: Sieve of Eratosthenes}
circle every number thats prime, then cross off all multiples of that number


\end{document}