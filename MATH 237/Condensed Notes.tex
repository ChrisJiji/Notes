\documentclass[10pt,letter]{article}
\usepackage{amsmath}
\usepackage{amssymb}
\usepackage{amsthm}
\usepackage{graphicx}
\usepackage{setspace}
\onehalfspacing
\usepackage{fullpage}
\newtheorem*{remark}{Remark}
\theoremstyle{plain}
\newtheorem*{theorem*}{Theorem}
\newtheorem{theorem}{Theorem}[section]
\newtheorem{corollary}{Corollary}[theorem]
\newtheorem*{lemma*}{Lemma}
\newtheorem{lemma}[theorem]{Lemma}
\theoremstyle{definition}
\newtheorem{definition}{Definition}[section]
\newtheorem*{definition*}{Definition}

\begin{document}
\paragraph{Graphs of Second Degree Equations}\mbox{}\newline

\begin{tabular}{|p{2cm}|p{4cm}|p{7cm}|}\\
\hline
\textbf{Graph}     & \textbf{Formula} & \textbf{Notes} \\ \hline
\textbf{Circle}    & $(x-h)^2+(y-k)^2=r^2$               & Centre is $(h,k)$ and radius is $r$  \\\hline
\textbf{Parabola}  & $y=ax^2+bx+c$                       & Can find roots with quadratic equation, vertex by completing the square\\\hline   
\textbf{Ellipse}   & $\frac{x^2}{a^2}+\frac{y^2}{b^2}=1$ & $x$ intercepts are found by setting $y=0$, $y$ intercepts are found by setting $x=0$\\\hline
\textbf{Hyperbola} & $\frac{x^2}{a^2}-\frac{y^2}{b^2}=1$ & Find $x$ intercepts by setting $y=0$. Its asymptotes are $y=\pm\left(\frac{b}{a}\right)x$  \\
\hline
\end{tabular}\\ \\
\paragraph{Level Set} Set $f(x,y)=k$ for some constant $k$, draw a set of $f(x,y)$ for a number of $k$.
\paragraph{Cross-section} Similar as above, but keep $x$ or $y$ as a constant, and draw the set of $f(x,c)=z$ or $f(c,y)=z$ for some $c$. 
\paragraph{Limits} 
\subparagraph{Prove it doesn't exist} Replace $y$ with an equation for $y$ (ie. $y=mx$), and then prove that it will be different depending on the choice of $m$.
\subparagraph{Prove it exists} Squeeze theorem. Useful inequalities: \begin{align*}|a|&=\sqrt{a^2}\\|a+b|&\leq|a|+|b|\\2|a||b|&\leq a^2+b^2\end{align*}
\paragraph{Continuity} If a function is a sum, product, quotient, or composite function of any of the following basic functions (constant, coordinates, logarithm, exponential, trigonometric, inverse trigonometric, absolute value), then it is continuous. 
\paragraph{Partial Derivatives} $\frac{\partial f}{\partial x}(a,b)=\lim_{h\rightarrow0}\frac{f(a+h,b)-f(a,b)}{h}$ Otherwise, differentiate using normal derivative rules. 
\paragraph{Tangent Plane}$z=f(a,b)+\frac{\partial f}{\partial x}(a,b)(x-a)+\frac{\partial f}{\partial y}(a,b)(y-b)$
\paragraph{Linear Approximation} $L_{(a,b)}(x,y)=f(a,b)+\frac{\partial f}{\partial x}(a,b)(x-a)+\frac{\partial f}{\partial y}(a,b)(y-b)$, and $f(x,y)\approx L_{(a,b)}(x,y)$ 
\paragraph{Gradient} Suppose that $f:\mathbb{R}^3\rightarrow\mathbb{R}$ has partial derivatives at $\underline{a}$. The \textbf{gradient} of $f$ at $\underline{a}$ is defined by $\nabla f(\underline{a})=(f_x(\underline{a}),f_y(\underline{a}),f_z(\underline{a}))$ 
\paragraph{Linear Approximation in Higher Dimensions} $f(\underline{x})\approx L_{\underline{a}}(\underline{x})=f(\underline{a})+\nabla f(\underline{a})\cdot(\underline{x}-\underline{a})$ for all $\underline{x}$ sufficiently close to $\underline{a}$ 
\paragraph{Differentiability} A function $f:\mathbb{R}^2\rightarrow\mathbb{R}$ is \textbf{differentiable} at $\underline{a}=(a,b)$ if and only if there is a linear function $L(\underline{x})=f(a,b)+c(x-a)+d(y-b)$ such that $\lim_{\underline{x}\rightarrow\underline{a}}\frac{|R_{1,\underline{a}}(\underline{x})|}{||\underline{x}-\underline{a}||}=0$, where $R_{1,\underline{a}}(\underline{x})=f(\underline{x})-L(\underline{x})$. Note that if a function is differentiable with $L(\underline{x})$, then $L(\underline{x})$ is the linear approximation, that is, $c=f_x(a,b)$, and $d=f_y(a,b)$. 



\end{document}