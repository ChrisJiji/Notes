\documentclass[10pt,english]{article}
\usepackage[T1]{fontenc}
\usepackage[latin9]{inputenc}
\usepackage{geometry}
\geometry{verbose,tmargin=1.5in,bmargin=1.5in,lmargin=1.5in,rmargin=1.5in}
\usepackage{amsthm}
\usepackage{amsmath}
\usepackage{amssymb}

\makeatletter
\usepackage{enumitem}
\newlength{\lyxlabelwidth}

\usepackage[T1]{fontenc}
\usepackage{ae,aecompl}

%\usepackage{txfonts}

\usepackage{microtype}

\usepackage{calc}
\usepackage{enumitem}
\setenumerate{leftmargin=!,labelindent=0pt,itemindent=0em,labelwidth=\widthof{\ref{last-item}}}

\makeatother

\usepackage{babel}
\begin{document}
\noindent \begin{center}
\textbf{\large{}PMATH340 - Assignment 1}\\
\textbf{\large{}Chris Ji 20725415}
\par\end{center}{\large \par}
\medskip{}

\begin{enumerate}
\item \begin{enumerate}
    \item \begin{align*}19549&= 14803 + 4746 \\ 14803 &= 3\times4746 + 565 \\ 4746 &= 8\times 565+226\\ 565&=2\times226+113\\226&=2\times113+0\end{align*} Therefore, $\text{gcd}(14803,19549)=113$
    \item Using back substitution, 
    \begin{align*}113&=565-2\times226\\ 
    113&=565 -2\times(4746-8\times565)\\
    &=17\times565-2\times4746\\
    113&=17\times(14803-3\times4746)-2\times4746\\
    &=17\times14803-53\times4746\\
    113&=17\times14803-53\times(19549-14803)\\ 
    &=70\times14803-53\times19549\end{align*}
\end{enumerate}

\pagebreak
\item We will proceed by induction. For $f_1,f_2$, $\text{gcd}(f_1,f_2)$ is clearly $1$. Now we assume that $\text{gcd}(f_{n-1},f_n)$ this is true for all $n$. Consider $\text{gcd}(f_n,f_{n+1})$. This is equal to $\text{gcd}(f_{n},f_{n}+f_{n-1})$ which, by Lemma 1.5 in the notes, is equal to $\text{gcd}(f_{n},f_{n-1})$, which is true by our inductive hypothesis. Thus the gcd of any two consecutive Fibonacci numbers is 1.

\pagebreak
\item Let $n=1$. Then $(1)^7-1=0$, which is a multiple of $7$. So we assume that $n^7-n$ is a multiple of $7$ for all $n$. Now consider $n+1$. $(n+1)^7-(n+1)=n^7+7n^6+21n^5+35n^4+35n^3+21n^2+6n$. Clearly $7 | 7n^6+21n^5+35n^4+35n^3+21n^2$, so by property 2 of divisibility we only need to show $7|n^7+6n$. Note that $n^7+6n=(n^7-n)+n+6n$, and so we need to prove $7|n^7-n+7n$. Clearly $7|7n$, and by inductive hypothesis $7|n^7-n$, and so by property 2 of divisibility, $7|n^7-n+7n$, and so (also by property 2), $7|(n+1)^7-(n+1)$.

\pagebreak
\item Note $n^3+1=(n+1)(n^2-n+1)$. Then clearly $n^3+1$ is prime if and only if one of $n+1$ or $n^2-n+1$ are equal to $1$ (or else $n^3+1$ would have two non-one factors, and so is not prime). If $n+1=1$, then $n=0$, and $n^3+1=(0)^3+1=1$ is not prime. If $n^2-n+1=1$, then $n=0,1$, and so either $n^3+1=1$ is not prime, or $n=1$ and we are done. 

\pagebreak
\item Say $\text{gcd}(7k+2,11k+3)=1$. Then by Corollary 1.7 in the notes, we know there exists $x,y\in\mathbb{Z}$ such that $1=(7k+2)x+(11k+3)y$. Using the Euclidean Algorithm: 
\begin{align*}
11k+3&=(7k+2)+(4k+1)\\ 
7k+2&=(4k+1)+(3k+1)\\
4k+1&=(3k+1)+(k)\\
3k+1&=(3k)+(1)
\end{align*}
Back substituting: 
\begin{align*}
1&=(3k+1)-(3k) \\ 
1&=(3k+1)-3\times[(4k+1)-(3k-1)]\\
&=4\times(3k+1)-3\times(4k+1) \\ 
1&=4\times[(7k+2)-(4k+1)]-3\times(4k+1)\\ 
&=4\times(7k+2)-7\times(4k+1)\\
\end{align*}
We can see $(x,y)=(4,-7)$ is an integer solution to our equation, and so since there is an integer solution to $1=(7k+2)x+(11k+3)y$, the gcd of $7k+2$ and $11k+3$ must be 1. 

\pagebreak
\item Suppose there is some $N$ such that $\sum_{j=1}^N\frac{1}{j}$ is an integer. Define $\alpha=\text{lcm}(1,\ldots,N)$. Note that $\alpha$ must be even, as it will be the combination of the highest powers of the primes in the prime factorizations of $(1,\ldots,N)$, so it will be multiplied by at least one power of 2, since $N>1$. Any number multiplied by an even number will be even. \\ \\
It is easy to see that $\sum_{j=1}^N\frac{1}{j}=\sum_{i=1}^N\frac{\left(\frac{\alpha}{i}\right)}{\alpha}=\frac{1}{\alpha}\sum_{i=1}^N\frac{\alpha}{i}$. Then for $\sum_{j=1}^N\frac{1}{j}$ to be an integer, we need $\alpha|\sum_{i=1}^N\frac{\alpha}{i}$, or $k\times\alpha=\sum_{i=1}^N\frac{\alpha}{i}$, for some $k\in\mathbb{Z}$.  \\ \\
Consider the prime factorization of $\alpha$ and all $i\in\{1,\ldots,N\}$, in particular look at the highest power of 2 in the prime factorization of $\alpha$, say some number $l$ for $l\in\{1,\ldots,N\}$. $\frac{\alpha}{l}$ will be odd, as $\alpha$ will have its power of 2 removed, and hence be the product of a bunch of odd numbers, and so it is odd. Then for all $i\neq l$, $\frac{\alpha}{i}$ will be even, as they will have at least one power of 2. Then $\sum_{i=1}^N\frac{\alpha}{i}$ will be odd, as it will be the sum of some number of even numbers(all the $i\neq l$), plus $\frac{\alpha}{l}$, which is odd. However, $\sum_{i=1}^N\frac{\alpha}{i}=k\times\alpha$, which is even, as $\alpha$ is even and the product of any number with an even number will be even. Then one side is even and one side is odd, contradicting our assumption that $\sum_{j=1}^N\frac{1}{j}$ is an integer. \\ \\ \\ 

One case not discussed is if $l$ appears in the factorization of more than one number in $\{1,\ldots,N\}$. Suppose it does, and let $l=2^n$ for some $n$. Then $2^n$ divides some other number in $\{1,\ldots,N\}$, say $m$. Then $m=2^n\times k$ for some $k\in\mathbb{Z}$. If $k=2$, then $m=2^{n+1}$, contradicting the fact that $2^n$ is our highest power of $2$. If $k>2$, then $m=2^n\times k > 2^{n+1}$. But remember that $2^n$ is our highest power of $2$, and so $m>2^{n+1}$ implies that $m\notin\{1,\ldots,N\}$. Therefore there can only be one highest power of $2$ in the factorization of the numbers $\{1,\ldots,N\}$.


\pagebreak
\item By the Fundamental Theorem of Arithmetic, we can simply list the prime factorizations of the numbers $1,\ldots,25$, and take the highest powers of each prime to get the least common multiple. \\ \\ 
\begin{tabular}{|c|c|}
\hline
Number & Prime factorization \\ \hline
$1$ & $1$    \\
$2$ & $2$    \\      
$3$ & $3$    \\ 
$4$ & $2^2$   \\
$5$ & $5$ \\ 
$6$ & $2*3$ \\ 
$7$ & $7$ \\ 
$8$ & $2^3$ \\ 
$9$ & $3^2$ \\ 
$10$ & $2*5$ \\ 
$11$ & $11$ \\ 
$12$ & $2^2*3$ \\ 
$13$ & $13$ \\ 
$14$ & $2*7$ \\ 
$15$ & $3*5$ \\ 
$16$ & $2^4$ \\ 
$17$ & $17$ \\ 
$18$ & $2*3^2$ \\ 
$19$ & $19$ \\ 
$20$ & $2^2*5$ \\ 
$21$ & $3*7$ \\ 
$22$ & $2*11$ \\ 
$23$ & $23$ \\ 
$24$ & $2^3*3$ \\ 
$25$ & $5^2$ \\ 
\hline
\end{tabular}\\ \\
Comparing the highest powers of each prime, we can see that $2^3*3^2*5^2*7*11*13*17*19*23=13385572200$ is the lowest common multiple of the numbers $1,\ldots,25$.


\pagebreak
\item As from above (question 7), we know the least common multiple of the numbers $1,\ldots,200$, is the product of the highest powers of the unique primes in the prime factorizations of the same numbers. Then, to minimize the least common multiple of the remaining numbers, we want to remove the highest possible numbers that are prime factorized into primes that aren't found in any other number in the sequence. As two consecutive numbers will always feature one even one odd, then we will just choose the highest prime, as the even number that it is consecutive to/is consecutive to it will not have the highest powers of primes in its prime factorization. 199 is the highest prime in this sequence, and so we can remove either $\{198,199\}$ or $\{199,200\}$ to achieve the same minimal least common multiple. 


\pagebreak
\item As $p,q$ are prime numbers, $p^aq^b$ is exactly the prime factorization of $p^aq^b$. For $p^aq^b$ to be perfect, we need $2(p^aq^b)=\sigma(p^aq^b)=\sigma(p^a)\sigma(q^b)$ since $p,q$ are obviously coprime, and so $\sigma$ is multiplicative. $\sigma(p^a)\sigma(q^b)=\frac{p^{a+1}-1}{p-1}\frac{q^{b+1}-1}{q-1}$ by geometric series. $\frac{p^{a+1}-1}{p-1}\frac{q^{b+1}-1}{q-1}<p^aq^b\frac{p}{p-1}\frac{q}{q-1}$. \\ 
Now note that both $\frac{p}{p-1}$ and $\frac{q}{q-1}$ are decreasing functions for reasons discussed in the notes (example 1.15), and so since they are both odd primes the maximum value of each will be $\frac{3}{2}$. Then $\sigma(p^aq^b)<p^aq^b\frac{9}{4}$. It's easy to see from this that $p^aq^b$ will only be perfect if $p,q$ are both equal to $3$, as then $\sigma(p^aq^b)=2(p^aq^b)<\frac{9}{4}p^aq^b$ will be true. If just one of them are greater than $3$ (greater than or equal to $5$), then $2(p^aq^b)<\frac{3}{3-1}\frac{5}{5-1}p^aq^b=\frac{15}{8}p^aq^b$ is clearly false. \\
If $p=q=3$, then we need to solve $2(3^{a+b})=\frac{3^{a+1}-1}{2}\frac{3^{b+1}-1}{2}=\frac{3^{a+b+2}-3^{a+1}-3^{b+1}+1}{4}\Rightarrow 8(3^{a+b})-1=3(3^{a+b+1}-3^a-3^b)\Rightarrow-1=3^{a+b}-3^{a+1}-3^{b+1}$. This is clearly false, as any sum (or difference) of powers of 3 (RHS) will be divisible by $3$, but $-1$ (LHS) is not divisible by $3$.\\
Therefore $p^aq^b$ is not perfect if $p,q\geq3$, and so not perfect for all odd primes. 



\end{enumerate}

\end{document}
